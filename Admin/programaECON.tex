\documentclass[11pt]{article}
\usepackage[utf8]{inputenc}
\usepackage[spanish]{babel}
\usepackage{amsmath}
\usepackage{subcaption}
\usepackage{caption}
\usepackage{amssymb}
\usepackage{graphicx}
\usepackage{natbib}
\usepackage{anysize}
\usepackage{hyperref}
\usepackage{color,soul}
\usepackage[usenames,dvipsnames,svgnames,table]{xcolor}
\usepackage{multicol}
\usepackage{cancel}
\usepackage{xfrac}
\usepackage{mathrsfs}
\usepackage{amsthm}
\usepackage{multirow}
\usepackage{enumitem}
\usepackage{pgf}
\usepackage{pgfpages}
\usepackage{multicol}

%%%% Diseño de pagina
%\renewcommand{\baselinestretch}{1.5}
\setlength{\parskip}{\baselineskip}
\marginsize{1cm}{1cm}{0cm}{0.5cm} 
\setlength{\parindent}{0pt}
%\setlength{\columnseprule}{0.4pt}

\pgfpagesdeclarelayout{boxed}
{
  \edef\pgfpageoptionborder{0pt}
}
{
  \pgfpagesphysicalpageoptions
  {%
    logical pages=1,%
  }
  \pgfpageslogicalpageoptions{1}
  {
    border code=\pgfsetlinewidth{0.5pt}\pgfstroke,%
    border shrink=\pgfpageoptionborder,%
    resized width=.95\pgfphysicalwidth,%
    resized height=.95\pgfphysicalheight,%
    center=\pgfpoint{.5\pgfphysicalwidth}{.5\pgfphysicalheight}%
  }%
}

%%%% Funciones
\newcommand{\hlc}[2][yellow]{ {\sethlcolor{#1} \hl{#2}} }
\newcommand*{\rom}[1]{\expandafter\romannumeral #1}
\newcommand{\Rom}[1]{\uppercase\expandafter{\romannumeral #1\relax}}
\newcommand{\Importante}[2]{{\color{#1}#2}}
\newcommand{\importante}[2]{{\color{#1}\underline{#2}}}
\newcommand{\plim}{\text{Plím }}
\newcommand{\sumi}{\sum_{i=1}^n}
\newcommand{\encabezado}{\vspace*{-2.5cm}
\begin{minipage}{0.4\linewidth}
\centering
    \includegraphics[width=0.6\textwidth]{img/logo_econ.png}
\end{minipage}
\begin{minipage}{0.6\linewidth}
\begin{center}
\textbf{\large TALLER USOS DE \LaTeX\ }\\
\textbf{\large ECON--1298 - Sección 1}\\
\textbf{\large Julián Enrique Chitiva Bocanegra}\\
\textbf{\href{mailto:je.chitiva10@uniandes.edu.co}{je.chitiva10@uniandes.edu.co}}\\
\textbf{2019-2}\\
\end{center}
\end{minipage}
}


\pgfpagesuselayout{boxed}

\begin{document}
\vspace*{2cm}
\encabezado


\begin{enumerate}
\item \textbf{Horario atención a estudiantes, correos electrónicos y nombres
  de los profesores complementarios}\\~\\
\textbf{\underline{Clase magistral}}

Profesor: Julián Enrique Chitiva Bocanegra (\href{mailto:je.chitiva10@uniandes.edu.co}{je.chitiva10@uniandes.edu.co}) \\
Horario de atención a estudiantes: Jueves 7:00 am - 8:00 am \\
Lugar de atención a estudiantes: W-911.\\

 \item \textbf{Introducción y descripción general del curso }
 El curso Taller de \LaTeX\ busca que los estudiantes adquieran una comprensión general de lenguaje \TeX y su utilidad para la producción de documentación técnica y científica. \LaTeX\ es el estándar para la comunicación y publicación de documentos científicos. El énfasis será sobre las partes de un documento en \TeX, manejo de escritura matemática, tablas, gráficos y conexión con otros softwares como R y Stata, además de la exportación a otros formatos como Word.

El curso busca que los estudiantes entiendan la estructura general de programación en \TeX y la apliquen a sus trabajos requeridos en otras clases. Al final se espera que los estudiantes se sientan cómodos utilizando \TeX para escribir documentos y hacer presentaciones.


\item\textbf{Objetivos de la materia}

\begin{itemize}
    \item Familiarizarse con el lenguaje \TeX.
    \item Motivar el uso de \LaTeX\ a partir de casos prácticos y aplicaciones.
    \item Adquirir herramientas para la escritura de documentos científicos.
    \item Dominio de paquetes y comandos frecuentemente usados en \LaTeX
    \item Aprender formas de automatizar tareas en \LaTeX
    \item Tener los conocimientos necesarios para que puedan continuar de manera autónoma su
    aprendizaje de paquetes y aplicaciones de \LaTeX
\end{itemize}

\item \textbf{Organización del curso}

\begin{multicols}{2}
 \setlength{\columnseprule}{0.8pt} 
\begin{enumerate}[start=1,label={\bfseries \arabic*.},leftmargin=1cm]
    \item \textbf{Introducción a \LaTeX\ . }
        \begin{itemize}
            \item ¿Qué es \LaTeX\ ? ¿Por qué \LaTeX\ ?
            \item Instalación de \LaTeX\  
            \item Compiladores de \LaTeX\  \textit{off-line} y \textit{on-line}  (\href{www.overleaf.com}{overleaf}).
            \item Tipos de \textit{documentclass} estándar.
            \item Estructura básica de un documento en \LaTeX\ .
        \end{itemize}
    \item \textbf{Escritura matemática.}
        \begin{itemize}
            \item Símbolos matemáticos y letras griegas
            \item Escritura de fórmulas y proposiciones
            \item Ecuaciones de más de una linea
            \item Alinear ecuaciones
        \end{itemize}
    \item \textbf{\textit{mini páginas} y Tablas.}
        \begin{itemize}
            \item Uso de minipages
            \item Tabular
            \item Table
            \item Ajustar tablas al tamaño requerido.
            \item Diseño de tablas booktabs, multirows, multicolumn
        \end{itemize}
    \item \textbf{Listas.} 
        \begin{itemize}
            \item Listas no numeradas
            \item Listas numeradas
            \item Cambiar diseño de listas (enumerate y enumitem).
        \end{itemize}
    \item \textbf{Diseño de página.}
        \begin{itemize}
            \item múltiples columnas (multicol)
            \item Espaciado horizontal (hspace)
            \item Espaciado vertical (vspace)
            \item Páginas horizontales
            \item Numeración de páginas
            \item Salto de página
        \end{itemize}
    \item \textbf{Gráficos e imágenes.}
        \begin{itemize}
            \item Insertar imagenes
            \item Cambiar tamaño de imagenes
            \item Paquetes subfigure y subcaption
            \item Tikz
            \begin{itemize}
                \item Ecuaciones
                \item Árboles
                \item Redes
            \end{itemize}
            \item Colores
            \item Gráficos con datos desde \LaTeX\ .
        \end{itemize}
    \item \textbf{Beamer y presentaciones.}
        \begin{itemize}
            \item Estructura básica
            \item Formatos de presentación
            \item Diseño de Encabezado y pie de página
        \end{itemize}
    \item \textbf{Diseño de exámenes y talleres.}
        \begin{itemize}
            \item Paquete exam
            \item Escritura de soluciones
            \item Tablas de calificación
            \item Preguntas bono
            \item Preguntas tituladas
        \end{itemize}
    \item \textbf{Manejo de referencias y bibliografía.}
        \begin{itemize}
            \item Archivos bibtex
            \item Natbib
            \item Estilos de citación
        \end{itemize}
    \item \textbf{Conexión con otros softwares.}
        \begin{itemize}
            \item Stata 
            \item R
            \item Excel
            \item (Pandoc) Word
            \item (Pandoc) Html
        \end{itemize}
    \item \textbf{Manejo de inputs.}
        \begin{itemize}
            \item Inputs desde \LaTeX
            \item Inputs desde otros archivos
        \end{itemize}
    \item \textbf{Creación de paquetes.}
        \begin{itemize}
            \item Diseño
            \item Programación de funciones
            \item Publicación
        \end{itemize}
    \item \textbf{Journals de Economía.}
        \begin{itemize}
            \item American Economic Association
            \item Econometrica
            \item The Quarterly Journal of Economics
            \item Otros
        \end{itemize}
    \item \textbf{Loops en \LaTeX\ .}
        \begin{itemize}
            \item Sobre el documento
            \item Sobre gráficos
        \end{itemize}
    
\end{enumerate}
\end{multicols}


\item \textbf{Metodología}

El curso se realizará en salas habilitadas para el uso de computadores y estarán divididas en dos partes; durante la primera parte el profesor introducirá técnicas, comandos y conceptos relacionados con la utilización de \LaTeX\ y utilizará ejemplos que lo ilustren. Durante la segunda parte los estudiantes deberán trabajar individualmente en un ejercicio, el cual deberán entregar al finalizar la clase.

Durante el curso, los estudiantes podrán trabajar en la escritura de un documento del tema de su elección. 

El curso tiene 14 módulos, los cuales buscan iniciar al estudiante en el uso de \TeX\ y presentar algunas aplicaciones básicas y avanzadas, que motiven su continuo aprendizaje y utilización de \LaTeX\ durante el curso y después de haberlo terminado.


\item \textbf{Competencias }
  
  Al final del curso, los estudiantes podrán:
\begin{itemize}
    \item Entender los archivos de ayuda sobre paquetes de \LaTeX.
    \item Escribir documentos y presentaciones.
    \item Manejar y diseñar de tablas.
    \item Importar y crear gráficos.
    \item Escribir funciones básicas.
    \item Importar datos de otros programas como R y Stata.
    \item Construir paquetes.
    \item Exportar/Importar a/de otros formatos como Word.
\end{itemize}

\item \textbf{Criterios de evaluación (Porcentajes de cada evaluación)}
\begin{itemize}
    \item Actividades de clase: $14*2.5\%=35\%$
    
    Todas las clases se evaluará los contenidos de la misma mediante una actividad corta. \\
    
    \item Talleres prácticos: $5\times 10\%=50\%$
    
    Se evaluará el contenidos acumulado de la clase mediante una actividad que contenga los elementos vistos hasta el momento. Las fechas de publicación y entrega de estas son:\\
    
    
    \begin{tabular}{ccc}
         Taller & Publicación & Entrega \\\hline
         1 & 22/08  & 29/08\\
         2 & 5/09 & 12/09\\
         3 & 26/09 & 3/10\\
         4 & 24/10 & 31/11\\
         5 & 14/11 & 21/11\\
    \end{tabular}
    
    \item Actividad final: $10\%$ (28/11)
    
    Se evaluará todo el contenido de la clase mediante una actividad. 
    
    \item Participación: $5\%$
\end{itemize}

Las actividades que se entreguen tarde o en un formato diferente al solicitado tendrán una penalidad de 1 sobre la nota definitiva. En caso de inasistencia, las actividades de clase no tendrán supletorio y la nota será el promedio entre la anterior y la siguiente. 

Según los artículos 62 y 63 del \href{https://secretariageneral.uniandes.edu.co/images/documents/Reglamento_Pregrado_web_2017.pdf}{{Reglamento general de estudiantes de pregrado}}, el estudiante tendrá \textbf{cuatro} días hábiles después de la entrega de la evaluación calificada para presentar un reclamo. El profesor responderá al reclamo en los \textbf{cinco} días hábiles siguientes. Si el estudiante considera que la respuesta no concuerda con los criterios de evaluación podrá solicitar un segundo calificador al Consejo de la Facultad en los \textbf{cuatro} días hábiles posteriores a la recepción de la decisión del profesor.

\item\textbf{Sistema de aproximación de notas definitiva}
El sistema de notas definitivas es el siguiente: las notas totales con decimales en 0 o en .5 no se modificarán. Las notas totales con decimales entre .25 a .49 y entre .75 a .99, se aproximarán a la nota definitiva siguiente. Las notas con decimales entre .01 a .24 y entre .51 a .74, se aproximarán a la nota definitiva anterior. Para que la nota definitiva se aproxime del rango 2.75-2.99 a 3.00, el estudiante deberá pasar por lo menos el examen final o uno de los dos parciales. De lo contrario, la nota definitiva quedará en 2.5.

\item \textbf{Fechas importantes}

\begin{itemize}
    \item Lunes 5 de agosto: inicio de clases.
    \item Jueves 26 de septiembre: Día del estudiante (no hay clases en la tarde).
    \item  Lunes 30 de septiembre a sábado 5 de octubre: semana de receso.
    \item  Viernes 4 de octubre: límite para subir las notas del 30\% a Banner (6pm).
    \item Viernes 11 de octubre: límite para que los estudiantes retiren materias.
    \item Sábado 30 de noviembre: último día de clases.
\end{itemize}

%\item \textbf{Bibliografía}
\nocite{*}

\bibliographystyle{Admin/apalike-es}
\bibliography{Admin/biblio}

\end{enumerate}


\end{document}