\documentclass[12pt]{article}
\usepackage[utf8]{inputenc}
\usepackage[spanish]{babel}
\usepackage{amsmath}
\usepackage{amssymb}
\usepackage{subcaption}
\usepackage{caption}
\usepackage{graphicx}
\usepackage{anysize}
\usepackage{hyperref}
\usepackage{color,soul}
\usepackage[usenames,dvipsnames,svgnames,table]{xcolor}
\usepackage{multicol}
\usepackage{cancel}
\usepackage{xfrac}
\usepackage{mathrsfs}
\usepackage{amsthm}
\usepackage{multirow}
\usepackage{enumitem}
\usepackage{pgf}
\usepackage{pgfpages}
\usepackage{multicol}

%%%% Diseño de pagina
%\renewcommand{\baselinestretch}{1.5}
\setlength{\parskip}{\baselineskip}
\marginsize{1cm}{1cm}{0cm}{0.5cm} 
\setlength{\parindent}{0pt}
%\setlength{\columnseprule}{0.4pt}

\pgfpagesdeclarelayout{boxed}
{
  \edef\pgfpageoptionborder{0pt}
}
{
  \pgfpagesphysicalpageoptions
  {%
    logical pages=1,%
  }
  \pgfpageslogicalpageoptions{1}
  {
    border code=\pgfsetlinewidth{0.5pt}\pgfstroke,%
    border shrink=\pgfpageoptionborder,%
    resized width=.95\pgfphysicalwidth,%
    resized height=.95\pgfphysicalheight,%
    center=\pgfpoint{.5\pgfphysicalwidth}{.5\pgfphysicalheight}%
  }%
}

%%%% Funciones
\newcommand{\hlc}[2][yellow]{ {\sethlcolor{#1} \hl{#2}} }
\newcommand*{\rom}[1]{\expandafter\romannumeral #1}
\newcommand{\Rom}[1]{\uppercase\expandafter{\romannumeral #1\relax}}
\newcommand{\Importante}[2]{{\color{#1}#2}}
\newcommand{\importante}[2]{{\color{#1}\underline{#2}}}
\newcommand{\plim}{\text{Plím }}
\newcommand{\sumi}{\sum_{i=1}^n}
\newcommand{\encabezado}{\vspace*{-2.5cm}
\begin{multicols}{2}

    \includegraphics[width=0.2\textwidth]{img/logo_econ.png}
    
\columnbreak

\begin{flushright}
\textbf{\large Taller de \LaTeX\ }\\
\textbf{\large Julián Chitiva Bocanegra}\\
\textbf{\small Propuesta}\\
\end{flushright}

\end{multicols}\vspace*{-1cm}
\noindent\rule{\linewidth}{1pt}}


\pgfpagesuselayout{boxed}


\begin{document}
\vspace*{0.8cm}
\encabezado

\begin{enumerate}[label={\bfseries\arabic*.}]
    \item \textbf{Descripción del Curso}\\
El curso Taller de \LaTeX\ busca que los estudiantes adquieran una comprensión general de lenguaje \TeX y su utilidad para la producción de documentación técnica y científica. \LaTeX\ es el estándar para la comunicación y publicación de documentos científicos. El énfasis será sobre las partes de un documento en \TeX, manejo de escritura matemática, tablas, gráficos y conexión con otros softwares como R y Stata, además de la exportación a otros formatos como Word.

El curso busca que los estudiantes entiendan la estructura general de programación en \TeX y la apliquen a sus trabajos requeridos en otras clases. Al final se espera que los estudiantes se sientan cómodos utilizando \TeX para escribir documentos y hacer presentaciones.

\item \textbf{Objetivos de la materia}

\begin{itemize}
    \item Familiarizar a los estudiantes en el lenguaje \TeX
    \item Motivar el uso de \LaTeX\ a partir de la presentación de casos prácticos
    \item Proporcionar herramientas para la escritura de documentos científicos
    \item Enseñar a los estudiantes paquetes y comandos frecuentemente usados en \LaTeX
    \item Enseñar a los estudiantes formas de automatizar tareas en \LaTeX
    \item Proporcionar herramientas para que los estudiantes puedan continuar de manera autónoma su
    aprendizaje de paquetes y aplicaciones de \LaTeX
\end{itemize}

\item \textbf{Competencias}\\
Al final del curso, los estudiantes podrán:
\begin{itemize}
    \item Entender los archivos de ayuda sobre paquetes de \LaTeX
    \item Escribir documentos y presentaciones
    \item Manejar y diseñar de tablas
    \item Importar y crear gráficos
    \item Escribir funciones básicas
    \item Importar datos de otros programas como R y Stata
    \item Construcción de paquetes
    \item Exportar/Importar a/de otros formatos como Word
\end{itemize}

\item \textbf{Metodología}\\

El curso se realizará en salas habilitadas para el uso de computadores y estarán divididas en dos partes; durante la primera parte el profesor introducirá técnicas, comandos y conceptos relacionados con la utilización de \LaTeX\ y utilizará ejemplos que lo ilustren; durante la segunda parte los estudiantes deberán trabajar individualmente en un ejercicio, el cual deberán entregar al finalizar la clase.

Durante el curso, los estudiantes podrán trabajar en la escritura de un documento del tema de su elección. 

El curso tiene 14 módulos, los cuales buscan iniciar al estudiante en el uso de \TeX y presentar algunas aplicaciones básicas y avanzadas, que motiven su continuo aprendizaje y utilización de \LaTeX\ durante el curso y después de haberlo terminado.

\item \textbf{Contenido y cronograma}\\

\begin{enumerate}[start=1,label={\large\bfseries Módulo\ \arabic*:},leftmargin=2cm]
    \item \textbf{Introducción a \LaTeX\ }
    \begin{enumerate}[start=1,label={\bfseries Semana\ \arabic*:},leftmargin=1cm,rightmargin=.1\linewidth]
        \item
        \begin{itemize}
            \item ¿Qué es \LaTeX\ ? ¿Por qué \LaTeX\ ?
            \item Instalación de \LaTeX\  
            \item Compiladores de \LaTeX\  \textit{off-line} y \textit{on-line}  (\href{www.overleaf.com}{overleaf}).
            \item Tipos de \textit{documentclass} estándar.
            \item Estructura básica de un documento en \LaTeX\ .
        \end{itemize}
    \end{enumerate}
    \item Escritura matemática
    \begin{enumerate}[start=2,label={\bfseries Semana\ \arabic*:},leftmargin=1cm,rightmargin=.1\linewidth]
        \item
        \begin{itemize}
            \item Símbolos matemáticos y letras griegas
            \item Escritura de fórmulas y proposiciones
            \item Ecuaciones de más de una linea
            \item Alinear ecuaciones
        \end{itemize}
    \end{enumerate}
    \item \textit{mini páginas} y Tablas
    \begin{enumerate}[start=3,label={\bfseries Semana\ \arabic*:},leftmargin=1cm,rightmargin=.1\linewidth]
        \item
        \begin{itemize}
            \item Uso de minipages
            \item Tabular
            \item Table
            \item Ajustar tablas al tamaño requerido.
            \item Diseño de tablas booktabs, multirows, multicols
        \end{itemize}
    \end{enumerate}
    \item Listas 
    \begin{enumerate}[start=4,label={\bfseries Semana\ \arabic*:},leftmargin=1cm,rightmargin=.1\linewidth]
        \item
        \begin{itemize}
            \item Listas no numeradas
            \item Listas numeradas
            \item Cambiar diseño de listas (enumerate y enumitem).
        \end{itemize}
    \end{enumerate}
    \item Diseño de página
    \begin{enumerate}[start=5,label={\bfseries Semana\ \arabic*:},leftmargin=1cm,rightmargin=.1\linewidth]
        \item
        \begin{itemize}
            \item múltiples columnas (multicolumn)
            \item Espaciado horizontal (hspace)
            \item Espaciado vertical (vspace)
            \item Páginas horizontales
            \item Numeración de páginas
            \item Salto de página
        \end{itemize}
    \end{enumerate}
    \item Gráficos e imágenes
    \begin{enumerate}[start=6,label={\bfseries Semana\ \arabic*:},leftmargin=1cm,rightmargin=.1\linewidth]
        \item
        \begin{itemize}
            \item Insertar imagenes
            \item Cambiar tamaño de imagenes
            \item Paquetes subfigure y subcaption
            \item Tikz
            \begin{itemize}
                \item Ecuaciones
                \item Árboles
                \item Redes
            \end{itemize}
            \item Colores
            \item Gráficos con datos desde \LaTeX\ .
        \end{itemize}
    \end{enumerate}
    \item Beamer
    \begin{enumerate}[start=7,label={\bfseries Semana\ \arabic*:},leftmargin=1cm,rightmargin=.1\linewidth]
        \item
        \begin{itemize}
            \item Estructura básica
            \item Formatos de presentación
            \item Diseño de Encabezado y pie de página
        \end{itemize}
    \end{enumerate}
    \item Diseño de exámenes y talleres 
    \begin{enumerate}[start=8,label={\bfseries Semana\ \arabic*:},leftmargin=1cm,rightmargin=.1\linewidth]
        \item
        \begin{itemize}
            \item Paquete exam
            \item Escritura de soluciones
            \item Tablas de calificación
            \item Preguntas bono
            \item Preguntas tituladas
        \end{itemize}
    \end{enumerate}
    \item Manejo de referencias y bibliografía
     \begin{enumerate}[start=9,label={\bfseries Semana\ \arabic*:},leftmargin=1cm,rightmargin=.1\linewidth]
        \item
        \begin{itemize}
            \item Archivos bibtex
            \item Natbib
            \item Estilos de citación
        \end{itemize}
    \end{enumerate}
    \item Conexión con otros softwares
    \begin{enumerate}[start=10,label={\bfseries Semana\ \arabic*:},leftmargin=1cm,rightmargin=.1\linewidth]
        \item
        \begin{itemize}
            \item Stata 
            \item R
            \item Excel
            \item (Pandoc) Word
            \item (Pandoc) Html
        \end{itemize}
    \end{enumerate}
    \item Manejo de inputs 
    \begin{enumerate}[start=11,label={\bfseries Semana\ \arabic*:},leftmargin=1cm,rightmargin=.1\linewidth]
        \item
        \begin{itemize}
            \item Inputs desde \LaTeX
            \item Inputs desde otros archivos
        \end{itemize}
    \end{enumerate}
    \item Creación de paquetes
    \begin{enumerate}[start=12,label={\bfseries Semana\ \arabic*:},leftmargin=1cm,rightmargin=.1\linewidth]
        \item
        \begin{itemize}
            \item Diseño
            \item Programación de funciones
            \item Publicación
        \end{itemize}
    \end{enumerate}
    \item Journals de Economía
    \begin{enumerate}[start=13,label={\bfseries Semana\ \arabic*:},leftmargin=1cm,rightmargin=.1\linewidth]
        \item
        \begin{itemize}
            \item American Economic Association
            \item Econometrica
            \item The Quarterly Journal of Economics
            \item Otros
        \end{itemize}
    \end{enumerate}
    \item Loops en \LaTeX
    \begin{enumerate}[start=14,label={\bfseries Semana\ \arabic*:},leftmargin=1cm,rightmargin=.1\linewidth]
        \item
        \begin{itemize}
            \item Sobre el documento
            \item Sobre gráficos
        \end{itemize}
    \end{enumerate}
    
    
\end{enumerate}

\end{enumerate}

\end{document}
