\documentclass[dvipsnames,xcolor=x11names, handout]{beamer}
\usepackage[spanish]{babel}
\usepackage[utf8]{inputenc}
\usepackage[all]{xy}
\usepackage{afterpage}
\usepackage{tikz}
\usepackage{cancel}
\usepackage{verbatim}
\usepackage{tabu}
\usepackage{xfrac}
\usepackage{mathrsfs}
\usepackage{amsthm}
\usepackage{amssymb}
\usepackage{bbm}
\usepackage{enumerate}
\usepackage{booktabs}
\usepackage{relsize}
\usepackage{hyperref}
\usepackage{float}
\usepackage{longtable}
\usepackage{amsmath}
\usepackage{multirow}
\usepackage{multicol}
\usepackage{colortbl}
\usepackage{adjustbox}
\usepackage{xfrac}
\usepackage{bm}
\usepackage{keystroke}
\usepackage{wrapfig}
\usepackage{graphicx}
\usepackage{csvsimple}
\usepackage{otros/pgf-pie}
\usetikzlibrary{decorations.pathmorphing, patterns,shapes}
\usetikzlibrary{positioning}
\usepackage{pgfplots}
\pgfplotsset{compat=1.12}
\usepackage{pgfplotstable}


\PassOptionsToPackage{demo}{graphicx}

\newcommand{\hlc}[2][yellow]{ {\sethlcolor{#1} \hl{#2}} }

\newcommand*{\rom}[1]{\expandafter\romannumeral #1}
\newcommand{\Rom}[1]{\uppercase\expandafter{\romannumeral #1\relax}}

\newcommand{\Importante}[2]{{\color{#1}#2}}
\newcommand{\importante}[2]{{\color{#1}\underline{#2}}}

\renewcommand{\baselinestretch}{1}
\setlength{\parskip}{\baselineskip}

\usetheme{Boadilla}
\definecolor{colorClase}{RGB}{148,178,0}
\usecolortheme[named=colorClase]{structure}
\usepackage{natbib}


\theoremstyle{plain}
  \newtheorem{teorema}{Teorema}
  \newtheorem{proposicion}{Proposición}
  \newtheorem{corolario}{Corolario}
  \newtheorem{lema}[teorema]{Lema}
\theoremstyle{definition}
  \newtheorem{definicion}{Definici\'on}
  \newtheorem{ejemplo}{Ejemplo}
  
  
\setbeamertemplate{caption}[numbered]
 
\title{Taller usos de \LaTeX}
\subtitle{Diseño de exámenes y talleres.}
\setbeamersize{text margin left=25pt,text margin right=25pt}

\author[Julián Chitiva Bocanegra]{Julián Enrique Chitiva Bocanegra}
\institute[Uniandes] 
{Universidad de los Andes\\ Facultad de Economía}
\titlegraphic{\includegraphics[width=0.8cm]{img/uniandes_logo.png}
}

\pgfdeclareimage[height=.8cm]{university-logo}{img/uniandes_logo.png}
\logo{\pgfuseimage{university-logo}}

\date{\today}
\subject{}
\usepackage{caption}
\usepackage{subcaption}

\begin{document}

\begin{frame}
  \titlepage
\end{frame}

\begin{frame}{Contenido.}
  \tableofcontents\footnote{\href{http://www-math.mit.edu/~psh/exam/examdoc.pdf}{Aquí la documentación de exam}}
\end{frame}

\section{Estructura básica.}
\begin{frame}{Contenido.}
  \tableofcontents[currentsection]
\end{frame}

\begin{frame}[fragile]{Estructura básica.}
\begin{itemize}[<+->]
    \item La estructura básica de un examen es la misma que la de un \verb!article!, solo basta usar el \verb!\documentclass[addpoints,12pt]{exam}!.
    \item Las preguntas van dentro de un ambiente llamado \verb!questions!.
    \item Las partes de las preguntas van dentro de un ambiente llamado \verb!parts!.
    \item Es recomendable ponerle un encabezado al examen (Instrucciones y espacio para poner el nombre del estudiante).
    \item También puede cambiar la numeración de las preguntas
\begin{verbatim}
\renewcommand{\thequestion}{\Roman{question}}
\renewcommand{\thepartno}{\Alph{partno}}
\end{verbatim}
\item Si quieren saltarse la identación hay que usar \verb!\uplevel!.
\end{itemize}
\end{frame}

\begin{frame}[fragile]{Instrucciones.}
    Una de las maneras\footnote{Puede crear su propio diseño.} de solicitar la información del estudiante y dar las instrucciones del examen es:
\begin{verbatim}
\begin{center}
\fbox{\fbox{\parbox{5.5in}{\centering
Instrucciones básicas para el desarrollo del examen}}}
\end{center}
\vspace{0.1in}
\makebox[\textwidth]{Información 1:\enspace\hrulefill}
\vspace{0.2in}
\makebox[\textwidth]{Información 2:\enspace\hrulefill}
\begin{center}
Este examen tiene \numquestions\ preguntas,
para un total de \numpoints\
puntos y \numbonuspoints\ puntos bono.
\end{center}
\end{verbatim}
\end{frame}

\begin{frame}[fragile]{Preguntas y subpreguntas.}
\begin{itemize}[<+->]
    \item Para insertar una pregunta dentro del ambiente \verb!questions!:
    \begin{verbatim}
        \question[numPuntos] La pregunta
    \end{verbatim}
    \item Para insertar una subpregunta dentro del ambiente \verb!parts!:
    \begin{verbatim}
        \part[numPuntos] La subpregunta
    \end{verbatim}
    \item También hay \verb!subparts! y  \verb!subsubparts! 
    \item Ojo con el número de puntos:
    \begin{itemize}[<+->]
        \item Si solo ponen puntos en las subpreguntas en la tabla de calificación se suman a la pregunta.
        \item La forma de poner medios puntos es con \verb!\half!.
        \item Para cambiar el nombre de los puntos toca insertar el siguiente comando
        \verb!\pointpoints{nombreSingular}{nombrePlural}!
        \item También se pueden cambiar todos los nombres a \% (por ejemplo) usando \verb!\pointname{\%}! 
    \end{itemize}
\end{itemize}
\end{frame}

\begin{frame}[fragile]{Encabezado y pie de página}
Para diseñar los encabezados y pies de página hay que poner 
\begin{verbatim}
\pagestyle{headandfoot}
\runningheadrule
\firstpageheader{Izq}{Centro}{Der}
\runningheader{Izq}{Centro \thepage\ of \numpages}{Der}
\firstpagefooter{Izq}{Centro}{Der}
\runningfooter{Izq}{Centro}{Der}
\end{verbatim}
\end{frame}

\section{Escritura de soluciones.}
\begin{frame}{Contenido.}
  \tableofcontents[currentsection]
\end{frame}

\begin{frame}[fragile]{Escritura de soluciones.}
La forma de escribir soluciones es muy sencilla.

\begin{itemize}[<+->]
    \item Después de la pregunta, parte o subparte hay que poner el ambiente \verb!solution!.
    \item Se puede cambiar el nombre de la solución con el comando 
    \verb!\renewcommand{\solutiontitle}{\textbf{Nombre de la solución:}}!
    \item Por \textit{default} las soluciones no aparecen, pero si se pone en el \verb!documentclass! la opción \verb!answers! se pueden visualizar.
\end{itemize}
    
\end{frame}

\section{Otros tipos de pregunta}
\begin{frame}{Contenido.}
  \tableofcontents[currentsection]
\end{frame}

\begin{frame}[fragile]{Selección múltiple 1 (con letras)}
    \begin{itemize}[<+->]
        \item Toca usar el ambiente \verb!choices!
        \item Cada opción se coloca como \verb!\choice!
        \item La opción correcta como \verb!\CorrectChoice!
    \end{itemize}
\end{frame}
\begin{frame}[fragile]{Selección múltiple (con checkbox)}
    \begin{itemize}[<+->]
        \item Toca usar el ambiente \verb!checkboxes!
        \item Cada opción se coloca como \verb!\choice!
        \item La opción correcta como \verb!\CorrectChoice!
    \end{itemize}
\end{frame}
\begin{frame}[fragile]{Completar}
    \begin{itemize}[<+->]
        \item Se usan en el ambiente \verb!questions!
        \item 
        \begin{scriptsize}
        \verb!\question Complete en la linea \fillin[Respuesta][largoLinea].!
        \end{scriptsize}
    \end{itemize}
\end{frame}

\section{Tabla de calificación.}
\begin{frame}{Contenido.}
  \tableofcontents[currentsection]
\end{frame}

\begin{frame}[fragile]{Tabla de calificación}
La forma de insertar las tablas de calificación es:
\begin{verbatim}
    \gradetable[orientacion][questions]
\end{verbatim}
\end{frame}

\section{Preguntas tituladas.}
\begin{frame}{Contenido.}
  \tableofcontents[currentsection]
\end{frame}

\begin{frame}[fragile]{Preguntas tituladas}
\begin{itemize}[<+->]
    \item La estructura es la misma que \verb!questions!. 
    \item Se puede poner de la siguiente manera:
    \begin{itemize}[<+->]
        \item \verb!\titledquestion{Pregunta \thequestion: Título de la pregunta}!
        \item \verb!\titledquestion{Título de la pregunta}!
    \end{itemize}
\end{itemize}
    
\end{frame}

\section{Preguntas bono.}
\begin{frame}{Contenido.}
  \tableofcontents[currentsection]
\end{frame}
\begin{frame}[fragile]{Preguntas bono.}
\begin{itemize}[<+->]
\item La estructura de las preguntas bono es la misma que para las preguntas normales, pero hay que cambiar 
\vfill
\begin{minipage}{0.35\linewidth}
\begin{verbatim}
\bonusquestion
\bonustitledquestion
\bonuspart
\bonussubpart
\bonussubsubpart
\end{verbatim}
\end{minipage}
\begin{minipage}{0.3\linewidth}
\begin{center}
En lugar de 
\end{center}
\end{minipage}
\begin{minipage}{0.25\linewidth}
\begin{verbatim}
\question
\titledquestion
\part
\subpart
\subsubpart
\end{verbatim}
\end{minipage}
\vfill
    \item Para cambiar el nombre de los puntos de las preguntas bono:
    \begin{scriptsize}
    \verb!\bonuspointpoints{nombreSingular (bono)}{nombrePlural (bono)}!
    \end{scriptsize}
\end{itemize}
    
\end{frame}


\end{document}