\documentclass[dvipsnames,xcolor=x11names]{beamer}
\usepackage[spanish]{babel}
\usepackage[utf8]{inputenc}
\usepackage[all]{xy}
\usepackage{afterpage}
\usepackage{tikz}
\usepackage{cancel}
\usepackage{verbatim}
\usepackage{tabu}
\usepackage{xfrac}
\usepackage{mathrsfs}
\usepackage{amsthm}
\usepackage{amssymb}
\usepackage{bbm}
\usepackage{enumerate}
\usepackage{booktabs}
\usepackage{relsize}
\usepackage{hyperref}
\usepackage{float}
\usepackage{longtable}
\usepackage{amsmath}
\usepackage{multirow}
\usepackage{multicol}
\usepackage{colortbl}
\usepackage{adjustbox}
\usepackage{xfrac}
\usepackage{bm}
\usepackage{keystroke}
\usepackage{wrapfig}
\usepackage{graphicx}
\usepackage{csvsimple}
\usepackage{otros/pgf-pie}
\usetikzlibrary{decorations.pathmorphing, patterns,shapes}
\usetikzlibrary{positioning}
\usepackage{pgfplots}
\pgfplotsset{compat=1.12}
\usepackage{pgfplotstable}


\PassOptionsToPackage{demo}{graphicx}

\newcommand{\hlc}[2][yellow]{ {\sethlcolor{#1} \hl{#2}} }

\newcommand*{\rom}[1]{\expandafter\romannumeral #1}
\newcommand{\Rom}[1]{\uppercase\expandafter{\romannumeral #1\relax}}

\newcommand{\Importante}[2]{{\color{#1}#2}}
\newcommand{\importante}[2]{{\color{#1}\underline{#2}}}

\renewcommand{\baselinestretch}{1}
\setlength{\parskip}{\baselineskip}


% There are many different themes available for Beamer. A comprehensive
% list with examples is given here:
% http://deic.uab.es/~iblanes/beamer_gallery/index_by_theme.html
% You can uncomment the themes below if you would like to use a different
% one:
%%%{{{
% \usetheme{AnnArbor}
% \usetheme{Antibes}
% \usetheme{Bergen}
%\usetheme{Berkeley}
% \usetheme{Berlin}
\usetheme{Boadilla}
% \usetheme{boxes}
% \usetheme{CambridgeUS}
% \usetheme{Copenhagen}
% \usetheme{Darmstadt}
% \usetheme{default}
% \usetheme{Frankfurt}
% \usetheme{Goettingen}
% \usetheme{Hannover}
% \usetheme{Ilmenau}
% \usetheme{JuanLesPins}
%\usetheme{Luebeck}
% \usetheme{Madrid}
%\usetheme{Malmoe}
% \usetheme{Marburg}
%\usetheme{Montpellier}
%\usetheme{PaloAlto}
%\usetheme{Pittsburgh}
% \usetheme{Rochester}
%\usetheme{Singapore}
% \usetheme{Szeged}
%\usetheme{Warsaw}
%%%}}}

%%%{{{
% \usecolortheme{albatross}
%  \usecolortheme{beaver}
% \usecolortheme{beetle}
% \usecolortheme{crane}
 \usecolortheme{default}
% \usecolortheme{dolphin}
% \usecolortheme{dove}
% \usecolortheme{fly}
% \usecolortheme{lily}
% \usecolortheme{monarca}
% \usecolortheme{orchid}
% \usecolortheme{rose}
% \usecolortheme{seagull}
% \usecolortheme{seahorse}
% \usecolortheme{sidebartab}
% \usecolortheme{spruce}
% \usecolortheme{structure}
% \usecolortheme{whale}
% \usecolortheme{wolverine}
%%%}}}

\definecolor{colorClase}{RGB}{112,39,61}
%\usecolortheme[named=colorClase]{structure}
\usepackage{natbib}


\theoremstyle{plain}
  \newtheorem{teorema}{Teorema}
  \newtheorem{proposicion}{Proposición}
  \newtheorem{corolario}{Corolario}
  \newtheorem{lema}[teorema]{Lema}
\theoremstyle{definition}
  \newtheorem{definicion}{Definici\'on}
  \newtheorem{ejemplo}{Ejemplo}
  
  
\setbeamertemplate{caption}[numbered]
 
\title{Taller usos de \LaTeX \\ \small{Beamer} \vspace*{-0.2cm}}
\subtitle{Beamer}
\setbeamersize{text margin left=25pt,text margin right=25pt}

\author[Julián Chitiva Bocanegra]{Julián Enrique Chitiva Bocanegra}
\institute[Uniandes] 
{Universidad de los Andes\\ Facultad de Economía}
\titlegraphic{\includegraphics[width=0.8cm]{img/uniandes_logo.png}
}

\pgfdeclareimage[height=.8cm]{university-logo}{img/uniandes_logo.png}
\logo{\pgfuseimage{university-logo}}

\date{\today}
\subject{}
\usepackage{caption}
\usepackage{subcaption}

\begin{document}

\begin{frame}
  \titlepage
\end{frame}

\begin{frame}{Contenido de esta linda presentaci\'on}
  \tableofcontents
\end{frame}

\begin{frame}{Lo mismo \dots}
  \dots pero el contenido aparece con pausas
  \tableofcontents[pausesections]
\end{frame}

% Section and subsections will appear in the presentation overview
% and table of contents.
\section{Primera secci\'on: frames, blocks, columns} %%%{{{
\begin{frame}{Contenido de esta linda presentaci\'on}
  \tableofcontents[currentsection]
  % You might wish to add the option [pausesections]
\end{frame}
%%% -------------------------------------------------------{{{
\begin{frame}[fragile]{Hacer una diapositiva}{\dots es f\'acil}
   % [fragile] allows the use of \verb+...+
   Para hacer una diapositiva, se usa
   \verb+\begin{frame}{Titulo de la diapositiva}{Subtitulo de la diapositiva}+\\
      \verb+    ... Contenido ... +\\
   \verb+\end{frame}+
   
   \begin{verbatim}
       jvhbjkjhvjghbkjk
       \begin{enumerate}
           \item 
       \end{enumerate}
   \end{verbatim}
   \bigskip
   \bigskip

   Gran parte de este archivo est\'a basada en 
   \href{http://www.math-linux.com/latex-26/article/how-to-make-a-presentation-with-latex-introduction-to-beamer}{http://www.math-linux.com/latex-26/article/how-to-make-a-presentation-with-latex-introduction-to-beamer}
\end{frame}
%%% -------------------------------------------------------}}}

%%% -------------------------------------------------------{{{
\begin{frame}[plain,fragile]{Hacer una diapositiva con m\'as espacio disponible}{\dots es f\'acil}
   \verb+\begin{frame}+{\color{red}\verb+[plain]+}\verb+{Titulo de la diapositiva}{Subtitulo de la diapositiva}+\\
      \verb+    ... Contenido ... +\\
   \verb+\end{frame}+
\end{frame}
%%% -------------------------------------------------------}}}


%%% -------------------------------------------------------{{{
\begin{frame}{Definiciones, teoremas, etc.}

   Cada charla debe contener una definici\'on \dots
   \begin{definicion}[Definición 1]
      Esta es una definición\dots
   \end{definicion}
   \dots y un teorema.
   \begin{teorema}
      Este es un teorema \dots
   \end{teorema}
   \begin{proof}
      Obvio.
   \end{proof}
\end{frame}
%%% -------------------------------------------------------}}}

%%% -------------------------------------------------------{{{
\begin{frame}[fragile]{Bloques}
   Se pueden producir bloques en distintos colores as\'i:

   \begin{block}{T\'itulo del bloque}
      \verb+\begin{block} ... \end{block}+
      produce un bloque normal.
   \end{block}

   \begin{alertblock}{T\'itulo del bloque}
      \verb+\begin{alertblock} ... \end{alertblock}+
      produce un bloque en un color llamativo.
   \end{alertblock}

   \begin{exampleblock}{T\'itulo del bloque}
      \verb+\begin{exampleblock} ... \end{exampleblock}+
      produce un bloque para ejemplos.
   \end{exampleblock}

\end{frame}
%%% -------------------------------------------------------}}}

%%% -------------------------------------------------------{{{
\begin{frame}[fragile]{Columnas btc}
   \begin{columns}
      \fbox{
      \begin{column}[b]{4.4cm}
	 \texttt{\textbackslash begin\{column\}[c]\{5cm\}\\
	 \dots\\
	 \dots\\
	 \dots\\
	 \dots\\
	 \dots\\
	 \textbackslash end\{column\}
	 }
      \end{column}
      }
      \fbox{
      \begin{column}[t]{1.4cm}
	 Puse las columnas en una framebox para que se vean mejor.
      \end{column}
      }
      \fbox{
      \begin{column}[c]{4.9cm}
	 \begin{itemize}
	    \item
	    La letra en [.] puede ser c,t,b (center, top, bottom). Se refiera a alineamento horizontal.\\
	    \item
	 En \{.\} se espec\'ifica el ancho de la columna.
	 Puede ser un valor absoluto como \texttt{5 cm}, o un m\'ultiplo de algo, por ejemplo \texttt{0.2\textbackslash textwidth}.
	 \end{itemize}
      \end{column}
      }
   \end{columns}
\end{frame}
%%% -------------------------------------------------------}}}

%%% -------------------------------------------------------{{{
\begin{frame}[fragile]{Columnas bbb}
   \begin{columns}
      \fbox{
      \begin{column}[b]{4.4cm}
	 \texttt{\textbackslash begin\{column\}[c]\{5cm\}\\
	 \dots\\
	 \dots\\
	 \dots\\
	 \dots\\
	 \dots\\
	 \textbackslash end\{column\}
	 }
      \end{column}
      }
      \fbox{
      \begin{column}[b]{1.4cm}
	 Puse las columnas en una framebox para que se vean mejor.
      \end{column}
      }
      \fbox{
      \begin{column}[b]{4.9cm}
	 \begin{itemize}
	    \item
	    La letra en [.] puede ser c,t,b (center, top, bottom). Se refiera a alineamento horizontal.\\
	    \item
	 En \{.\} se espec\'ifica el ancho de la columna.
	 Puede ser un valor absoluto como \texttt{5 cm}, o un m\'ultiplo de algo, por ejemplo \texttt{0.2\textbackslash textwidth}.
	 \end{itemize}
      \end{column}
      }
   \end{columns}
\end{frame}
%%% -------------------------------------------------------}}}

%%%}}}

% You can reveal the parts of a slide one at a time
% with the \pause command:

\pgfdeclareimage[height=.8cm]{university-logo}{img/logo_econ.png}

\section{Segunda secci\'on: overlays, pause}

\subsection{Subsecciones}

%%% -------------------------------------------------------{{{
\begin{frame}[fragile]{Subsecciones}
   \begin{itemize}
      \item
      Observe que a esta secci\'on damos m\'as estructura usando subsecciones.

      \item
      Observe que hemos cambiado el logo.
   \end{itemize}
\end{frame}
%%% -------------------------------------------------------}}}

\subsection{Pausas}

%%% -------------------------------------------------------{{{
\begin{frame}{Pause con \texttt{\textbackslash pause}}
   Cada charla debe tener una definici\'on \dots
   \begin{definicion}
      Esta es una definición
   \end{definicion}
   \pause
   \dots y un teorema.
   \begin{teorema}
      Este es un teorema
   \end{teorema}
   \pause
   \begin{proof}
      Obvio.
   \end{proof}

\end{frame}
%%% -------------------------------------------------------}}}

%%% -------------------------------------------------------{{{
\begin{frame}[fragile]{Enumeraciones}
   \begin{columns}
      \begin{column}{.5\textwidth}
	 \verb+\begin{itemize}+\\
	 \verb+  \item<2-> a partir de 2+\\
	 \verb+  \item<3-> a partir de 3+\\
	 \verb+  \item<4-> a partir de 4+\\
	 \verb+ \item<3> solo en 3+\\
	 \verb+ \item<-4> solo hasta 4+\\
	 \verb+ \item<5-> a partir de 5+\\
	 \verb+ \item<2-4> en 2 hasta 4+\\
	 \verb+\end{itemize}+
      \end{column}
      \begin{column}{.5\textwidth}
      \begin{itemize}
	 \item<2-> a partir de diapositiva 2
	 \item<3-> a partir de diapositiva 3
	 \item<4-> a partir de diapositiva 4
	 \item<3> solo en diapositiva 3
	 \item<-4> solo hasta diapositiva 4
	 \item<5-> a partir de diapositiva 5
	 \item<2-4> en diapositivas 2 hasta 4
      \end{itemize}
      \end{column}
   \end{columns}
   \bigskip

   \centerline{\bf\color{red}
   \alt<1>{Slide 1}{\alt<2>{Slide 2}{\alt<3>{Slide 3}{\alt<4>{Slide 4}{Slide 5}}%
   }}
   }
\end{frame}
%%% -------------------------------------------------------}}}

%%% -------------------------------------------------------{{{
\begin{frame}[fragile]{Enumeraciones}

   \begin{columns}
      \begin{column}{.5\textwidth}
	 \verb! \begin{itemize}!{\color{red}\verb![<+->]!}\\
	 \verb+    \item X +\\
	 \verb+    \item XX +\\
	 \verb+    \item XXX +\\
	 \verb+    \item XXXX +\\
	 \verb+    \item XXXXX +\\
	 \verb+ \end{itemize} +\\
      \end{column}
      \begin{column}{.5\textwidth}
      \begin{itemize}[<+->]
	 \item X 
	 \item XX
	 \item XXX
	 \item XXXX
	 \item XXXXX 
      \end{itemize}
      \end{column}
   \end{columns}
   \bigskip

\end{frame}
%%% -------------------------------------------------------}}}

\subsection{\texttt{\textbackslash only, \textbackslash uncover, \textbackslash alt, \textbackslash invisible }}
%%% -------------------------------------------------------{{{
\begin{frame}{\texttt{\textbackslash only, \textbackslash uncover }}
   \uncover<2->
   {a partir de la diapositiva 2\\}
   \uncover<3-4>
   {desde la diapositiva 3 hasta la 4\\}
   \uncover<4>{solo en la diapositiva 4, puede ser con \texttt{\textbackslash uncover<4>} o  \texttt{\textbackslash only<4>} \\}
   \uncover<3->{desde la diapositiva 3 hasta el final\\}
   \only<1>{Solo en la primera diapositiva\\}
\end{frame}
%%% -------------------------------------------------------}}}

%%% -------------------------------------------------------{{{
\begin{frame}{\texttt{\textbackslash alt, \textbackslash invisible}}

   \alt<1-2>{En las diapositivas 1 hasta 2 hay este texto.\\}{{\color{red}En las dem\'as no.\\}}

   \invisible<2-3>{Este texto es invisible en las diapositivas 2 hasta 3.\\ }
   \smallskip

   \only<4>{Este texto solo est\'a en la diapositiva 4.}

\end{frame}
%%% -------------------------------------------------------}}}

\subsection{Tablas}
%%% -------------------------------------------------------{{{
\begin{frame}{Tablas}
   \begin{tabular}{l|l}
      A & B\\\hline
      C& D\\
   \end{tabular}

\end{frame}
%%% -------------------------------------------------------}}}

\section{Colores}
%%% -------------------------------------------------------{{{
\begin{frame}[fragile]{\texttt{\textbackslash alert}}

   \verb+\alert<1>{Esto} \alert<2>{es} \alert<3>{rojo.}+\\
   \verb+Esto es \alert<2-3>{rojo.}+
   \verb+Esto es \alert<1,3>{rojo.}+
   \bigskip

   \alert<1>{Esto} \alert<2>{es} \alert<3>{rojo.}\\
   Esto es \alert<2-3>{rojo.}\\
   Esto es \alert<1,3>{rojo.}
   \bigskip
   \pause

   \'Util para escoger colores:
   \begin{itemize}
      \item
      \href{http://latexcolor.com/}{\texttt{http://latexcolor.com/}}

      \item \href{http://mirrors.ucr.ac.cr/CTAN/macros/latex/contrib/xcolor/xcolor.pdf}{xcolor}

   \end{itemize}

\end{frame}
%%% -------------------------------------------------------}}}

%%% -------------------------------------------------------{{{

\begin{frame}[fragile]{\color{magenta}Colores} 

   \definecolor{darkred}{rgb}{0.6,0,0}
   Colores ...\\
   \color<4>{red}{Rojo\\}
   \color<2>{green}{Verde\\}
   \color<3>{blue}{Azul\\}
   \color<5>{magenta}{Magenta\\}
   \color<6>{darkred}{darkred\\}

\end{frame}
%%% -------------------------------------------------------}}}

%%% -------------------------------------------------------{{{
\begin{frame}{Tablas}
   \definecolor{MyBlue}{rgb}{0.25,0.5,0.75}
   \colorlet{bbbbb}{blue}
   \colorlet{bbbbg}{blue!80!green}
   \colorlet{bbbgg}{blue!60!green}
   \colorlet{bbggg}{blue!40!green}
   \colorlet{bgggg}{blue!20!green}
   \colorlet{ggggg}{green}
   \colorlet{SecondBlue}{MyBlue!30!gray}
   \begin{tabular}{l|l}
      Fecha & Expositores  \\\hline\hline
      \color<1>{red}A &
      \only<1>{\cellcolor{bbbbg}}
      C\\
      \color<1>{red}B & \only<2>{\cellcolor{bbbgg}}
      \color<1>{red}D \pause \\ 
      \hline
   \end{tabular}

\end{frame}
%%% -------------------------------------------------------}}}


\section{Misc}

%%% -------------------------------------------------------{{{
\begin{frame}{Enfatisar palabras clave}
   \begin{definicion}
      Esta es una definición con palabras clave
   \end{definicion}
   \pause
   Mejor:
   \begin{definicion}
      Esta es una definición con \textbf{palabras clave}
   \end{definicion}

\end{frame}
%%% -------------------------------------------------------}}}

%%% -------------------------------------------------------{{{
\begin{frame}[fragile]{}
   \definecolor{defcolor}{rgb}{0,0.6,0.6}    %% blue-green
   \definecolor{bittersweet}{rgb}{1,0.44,0.37} %% bittersweet
   \newcommand{\define}[1]{{\bf\color{defcolor}#1}}
   \newcommand{\myemph}[1]{{\sc\color{bittersweet}#1}}
   \begin{itemize}
      \item Usar colores para enfatizar, pero no hay que exagerar.\pause
      \item Ser consistente en el uso de colores/fuentes.
	 \'Util puede ser definir comandos\\
	 \verb+\newcommand{\define}[1]{{\bf\color{defcolor}#1}}+\\
	 \verb+\newcommand{\myemph}[1]{{\sc\color{defcolor}#1}}+.\\
	 \smallskip
	 \pause

	 Lo importante en esta diapositiva es recordarles que
	 \myemph{no dejen la preparac\'ion de su charla para el \'ultimo momento.}
	 Procuren terminar con la preparaci\'on de las diapositivas por los menso tres d\'ias antes de su charla. Rev\'isela el d\'ia siguiente.
	 \myemph{Hay que ensayar la charla varias veces.}
	 \smallskip

	 Una \define{charla buena} es una en la cual el p\'ublico no se pierde.

      \item Cuidado con los colores: En la proyecci\'on con video beam pueden ser muy distintos de como aparacen en la pantalla del computador.

   \end{itemize}

\end{frame}
\end{document}