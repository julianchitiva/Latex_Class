\documentclass[dvipsnames,xcolor, handout]{beamer}
\usepackage[spanish]{babel}
\usepackage[utf8]{inputenc}
\usepackage[all]{xy}
\usepackage{afterpage}
\usepackage{tikz}
\usepackage{cancel}
\usepackage{verbatim}
\usepackage{xfrac}
\usepackage{mathrsfs}
\usepackage{amsthm}
\usepackage{amssymb}
\usepackage{bbm}
\usepackage{enumerate}
\usepackage{booktabs}
\usepackage{relsize}
\usepackage{hyperref}
\usepackage{float}
\usetikzlibrary{decorations.pathmorphing, patterns,shapes}
\usetikzlibrary{positioning}
\usepackage{pgfplots}
\pgfplotsset{compat=1.12}
\PassOptionsToPackage{demo}{graphicx}

\usepackage{array}

\usepackage{amsmath}
\usepackage{multirow}
\usepackage{colortbl}
\usepackage{multicol}
\usepackage{adjustbox}
\usepackage{xfrac}
\usepackage{bm}

\newcommand{\hlc}[2][yellow]{ {\sethlcolor{#1} \hl{#2}} }

\newcommand*{\rom}[1]{\expandafter\romannumeral #1}
\newcommand{\Rom}[1]{\uppercase\expandafter{\romannumeral #1\relax}}

\newcommand{\Importante}[2]{{\color{#1}#2}}
\newcommand{\importante}[2]{{\color{#1}\underline{#2}}}

\renewcommand{\baselinestretch}{1}
\setlength{\parskip}{\baselineskip}


\def\Put(#1,#2)#3{\leavevmode\makebox(0,0){\put(#1,#2){#3}}}

 \usetheme{Boadilla}

%\usecolortheme{crane}
\definecolor{colorClase}{rgb}{0.576,0.302, 0.067}
\usecolortheme[named=colorClase]{structure}
\usepackage{natbib}


\theoremstyle{plain}
  \newtheorem{teorema}{Teorema}
  \newtheorem{proposicion}{Proposición}
  \newtheorem{corolario}{Corolario}
  \newtheorem{lema}[teorema]{Lema}
\theoremstyle{definition}
  \newtheorem{definicion}{Definici\'on}
  \newtheorem{ejemplo}{Ejemplo}
  
  
\makeatletter
\setbeamertemplate{footline}
{
  \leavevmode%
  \hbox{%
  \begin{beamercolorbox}[wd=.4\paperwidth,ht=2.25ex,dp=1ex,center]{author in head/foot}%
    \usebeamerfont{author in head/foot}\insertshortauthor \hspace*{1em}(\insertshortinstitute)
  \end{beamercolorbox}%
  \begin{beamercolorbox}[wd=.5\paperwidth,ht=2.25ex,dp=1ex,center]{title in head/foot}%
    \usebeamerfont{title in head/foot}\insertsection 
  \end{beamercolorbox}%
  \begin{beamercolorbox}[wd=.1\paperwidth,ht=2.25ex,dp=1ex,center]{date in head/foot}%
    \usebeamerfont{date in head/foot}
    \insertframenumber{} / \inserttotalframenumber\hspace*{2ex} 
  \end{beamercolorbox}}%
  \vskip0pt%
}
\makeatother
  
\title{Taller usos de \LaTeX \\ \small{Escritura matemática} \vspace*{-0.2cm}}

\setbeamersize{text margin left=25pt,text margin right=25pt}

\author[Julián Chitiva Bocanegra]{Julián Enrique Chitiva Bocanegra}
\institute[Uniandes] 
{Universidad de los Andes\\ Facultad de Economía}
\titlegraphic{\includegraphics[width=0.8cm]{img/uniandes_logo.png}
}

\date{\today}

\subject{}
\usepackage[figurename=]{caption}
\begin{document}
\begin{frame}
  \titlepage
\end{frame}

\begin{frame}{Contenido.}
  \tableofcontents%[hideallsubsections]%, currentsection]
\end{frame}

\section{Estructura básica.}
\begin{frame}{Contenido.}
  \tableofcontents[currentsection]
\end{frame}

\begin{frame}[fragile]{Ambiente matemático.}
    Para escribir funciones matemáticas es necesario importar los siguientes paquetes:
    \begin{itemize}
        \item \verb!amsmath!
        \item \verb!amssymb!
        \item \verb!amsfonts!
        \item \verb!mathrsfs!
    \end{itemize}
\end{frame}
\begin{frame}{Ambiente matemático.}
 \LaTeX es famoso por su calidad tipográfica a la hora de escribir formulas matemáticas. \\~\\
 
 Dependiendo de lo que necesite escribir, hay varias formas de insertar ecuaciones:
 \begin{itemize}
     \item Dentro de una línea.
     \item Con un salto de línea.
     \item Con un salto de línea y la ecuación numerada.
 \end{itemize}
\end{frame}

\begin{frame}[fragile]{Ecuaciones dentro de una línea.}
    La forma de escribir ecuaciones dentro de una linea es escribiéndola entre signos peso: \verb! $ formula $! 
    
    \begin{ejemplo}
    Esta es una ecuación \verb!$a+b$! (Código)
    
    Esta es una ecuación $a+b$ (Resultado)
    
    \end{ejemplo}
\end{frame}

\begin{frame}[fragile]{Ecuaciones con un salto de línea.}
    La forma de escribir ecuaciones centradas con un salto de linea es escribiéndola entre doble signos peso: \verb! $$ formula $$! 
    
    \begin{ejemplo}
    Esta es una ecuación \verb!$$a+b$$! (Código)
    
    Esta es una ecuación $$a+b$$ (Resultado)
    
    \end{ejemplo}
\end{frame}

\begin{frame}[fragile]{Ecuaciones con un salto de línea numeradas.}
    La forma de escribir ecuaciones centradas con un salto de linea y que \LaTeX\ las numere automáticamente es escribiéndola dentro del ambiente \verb!equation!:
    \begin{verbatim}
    \begin{equation}
        formula
        \label{eq:etiqueta formula}
    \end{equation}  
    \end{verbatim}
    
\end{frame}

\begin{frame}[fragile]{Ecuaciones con un salto de línea numeradas.}

    \begin{ejemplo}
    \begin{multicols}{2}
    \setlength{\columnseprule}{0.8pt} 
    (Código) 
    \begin{verbatim}
    Esta es una ecuación
    \begin{equation}
        a+b
        \label{eq:ejemplo}
    \end{equation} 
    Esta es otra ecuación
    \begin{equation*}
        c+d
    \end{equation*} 
    \end{verbatim}
    \columnbreak
    
     (Resultado) \\
     Esta es una ecuación 
     \begin{equation}
        a+b
        \label{eq:ejemplo}
    \end{equation}  
    \vspace{0.5cm} Esta es otra ecuación
    \vspace*{-0.4cm}\begin{equation*}
        c+d
    \end{equation*} 
    \end{multicols}
    \end{ejemplo}
    
\end{frame}

\begin{frame}[fragile]{Ecuaciones con un salto de línea numeradas: referencias.}

Estas etiquetas son muy útiles para hacer referencias dentro del texto. 

    \begin{ejemplo}
    (Código) 
    \begin{verbatim}
    Este es un ejemplo de cómo referenciar
    la ecuación \ref{eq:ejemplo}.
    \end{verbatim}
    
     (Resultado)\\
     Este es un ejemplo de cómo referenciar
     la ecuación \ref{eq:ejemplo}.  
    \end{ejemplo}
    
    
\end{frame}

\section{Funciones básicas: Símbolos matemáticos y letras griegas.}
\begin{frame}{Contenido.}
  \tableofcontents[currentsection]
\end{frame}

\begin{frame}[fragile]{Reglas básicas.}
\begin{itemize}
    \item Los números y letras latinas se escriben normalmente.
    \begin{itemize}
        \item \verb! $f(x) = 5a+b$!
        \item $f(x) = 5a+b$
    \end{itemize}\pause
    \item Superíndices:
    \begin{itemize}
        \item \verb! $a^1, a^1+b, a^{1+b}$!
        \item  $a^1, a^1+b, a^{1+b}$
    \end{itemize}\pause
    \item Subíndices:
    \begin{itemize}
        \item \verb! $a_1, a_1+b, a_{1+b}$!
        \item  $a_1, a_1+b, a_{1+b}$
    \end{itemize}
\end{itemize}
\end{frame}

\begin{frame}[fragile]{Funciones.}
\begin{itemize}
    \item Fracciones
    \begin{itemize}
        \item \verb!$\frac{a}{b}=c$!
        \item $\frac{a}{b}=c$
    \end{itemize}\pause
    \item Raíces
    \begin{itemize}
        \item \verb!$\sqrt{3}$!: $\sqrt{3}$
        \item \verb!$\sqrt[4]{3+x^2}$!: $\sqrt[4]{3+x^2}$
    \end{itemize}\pause
    \item El símbolo de multiplicación se escribe \verb!\times!
    \begin{itemize}
        \item \verb!$a\times b$!: $a\times b$
    \end{itemize}\pause
    \item Funciones comunes en matemáticas se pone un backslash antes del nombre
    \begin{itemize}
        \item \verb!$\sin x, \cosh a, \ln_2 a, \log 1$!
        \item $\sin x, \tanh a, \ln_2 a, \log 1$
    \end{itemize}
\end{itemize}
\end{frame}

\begin{frame}[fragile]{Operadores.}
\begin{itemize}
\item La mayoría de operadores también están definidos como las funciones
    \begin{itemize}
        \item \verb!$\sum_{i=1}^{10} a_i$!: $\sum_{i=1}^{10} a_i$
        \item \verb!$\prod_{i=1}^{\infty} a_i$!: $\prod_{i=1}^{\infty} a_i$
        \item \verb!$\int_a^b x \mathrm{d}x$!: $\int_a^b x\mathrm{d}x$
        \item \verb!$\partial x$!: $\partial x$
    \end{itemize}\pause
    \item A veces es importante usar \verb!\limits! para los limites de los operadores
    \begin{itemize}
        \item \verb!$\sum_{i=1}^{10} a_i $!: $\sum_{i=1}^{10} a_i $
        \verb!$\sum\limits_{i=1}^{10} a_i$!: $\sum\limits_{i=1}^{10} a_i$
    \end{itemize}
\end{itemize}
\end{frame}

\begin{frame}[fragile]{Paréntesis y corchetes.}
\begin{itemize}
    \item Los paréntesis y corchetes se ingresan normalmente
    \begin{itemize}
        \item \verb!$y=2(x+b)$!: $y=2(x+b)$
        \item \verb!$$y=2[\frac{x}{b}]$$!: $$y=2[\frac{x}{b}]$$
    \end{itemize}\pause
    El problema es que no se ajustan al contenido\pause
    \item Para ajustar los paréntesis y corchetes toca definirlos así:
    \begin{itemize}
        \item \verb!$$y=2\left[\frac{x}{b}\right]$$!: $$y=2\left[\frac{x}{b}\right]$$
    \item Siempre tengo que abrir con \verb!\left! y cerrar con \verb!\right!
    \item Puedo combinar como quiera los paréntesis: \verb!. ( ) [ ]!
    \end{itemize}
\end{itemize}
\end{frame}

\begin{frame}[fragile]{Letras griegas y texto.}
\begin{itemize}
    \item Las letras griegas se escriben poniendo backslash antes del nombre
    \begin{itemize}
        \item \verb!$\alpha, \beta, \gamma$!: $\alpha, \beta, \gamma$
    \end{itemize}\pause
    \item Cuando están en mayúscula siguen la misma estructura anterior, poniendo backslash antes del nombre
    \begin{itemize}
        \item \verb!$\Psi, \Xi, \Gamma$!: $\Psi, \Xi, \Gamma$
    \end{itemize}\pause
    
    \item Para escribir texto dentro de una ecuación hay que escribirlo dentro del ambiente \verb!\text!, de lo contrario \LaTeX\ considerará las letras como variables.
    \begin{itemize}
        \item \verb!$A_{inicial}$!: $A_{inicial}$
        \item \verb!$A_{\text{inicial}}$!: $A_{\text{inicial}}$
    \end{itemize}
\end{itemize}
\end{frame}

\begin{frame}[fragile]{Conjuntos.}
\begin{itemize}
    \item Como las llaves hacen parte de la estructura básica de \LaTeX, los conjuntos se deben escribir de la siguiente manera:
    \begin{itemize}
        \item \verb!$$A=\{1,2,3,4,5\}$$!: $$A=\{1,2,3,4,5\}$$
    \end{itemize}\pause
    \item También podemos poner puntos suspensivos
    \begin{itemize}
        \item \verb!$$A=\{1,2,\dots,10\}$$!: $$A=\{1,2,\dots,10\}$$
    \end{itemize}
    Ver \verb!\cdots, \vdots, \ddots!
\end{itemize}
\end{frame}

\begin{frame}[fragile]{Vectores y matrices.}
    \begin{itemize}
        \item La estructura básica de una matriz es:
        \begin{itemize}
            \item \verb!$$\begin{matrix}a & b \\ c & d \end{matrix}$$!: $$\begin{matrix}a & b \\ c & d \end{matrix}$$
        \end{itemize}
        
       
    \end{itemize}
\end{frame}

\begin{frame}[fragile]{Vectores y matrices.}
    \begin{itemize}
        \item Vemos que el ambiente \verb!matrix! no pone nada alrededor de la matriz. Si queremos algún paréntesis usamos los siguientes ambientes:
        \begin{itemize}
            \item \verb!pmatrix!: Paréntesis $\begin{pmatrix}a & b \\ c & d \end{pmatrix}$\pause
            \item \verb!bmatrix!: Corchetes $\begin{bmatrix}a & b \\ c & d \end{bmatrix}$\pause
            \item \verb!Bmatrix!: Llaves $\begin{Bmatrix}a & b \\ c & d \end{Bmatrix}$\pause
            \item \verb!vmatrix!: $\begin{vmatrix}a & b \\ c & d \end{vmatrix}$\pause
            \item \verb!Vmatrix!: $\begin{Vmatrix}a & b \\ c & d \end{Vmatrix}$
        \end{itemize}
    \end{itemize}
\end{frame}

\begin{frame}[fragile]{Acentos.}
    \begin{itemize}
        \item \LaTeX\ permite acentos en el ambiente matemático
        \begin{itemize}
            \item Prima: \verb!$a^\prime$!: $a^\prime$\pause
            \item Gorro: \verb!$\hat{a}$!: $\hat{a}$
            \item Gorro ancho: \verb!$\widehat{aaa}$!: $\widehat{aaa}$\pause
            \item Punto: \verb!$\dot{a}$!: $\dot{a}$\pause
            \item Barra: \verb!$\bar{a}$!: $\bar{a}$
            \item Barra ancha: \verb!$\overline{aaa}$!: $\overline{aaa}$
            \item Subrayar:  \verb!$\underline{aaa}$!: $\underline{aaa}$\pause
            \item Tilde: \verb!$\tilde{a}$!: $\tilde{a}$
            \item Tilde ancha: \verb!$\widetilde{aaa}$!: $\widetilde{aaa}$\pause
            \item Flechas:
            \verb!$\vec{aaa}$!: $\vec{aaa}$\\
            \hspace{1.3cm}\verb!$\overrightarrow{aaa}$!: $\overrightarrow{aaa}$\\
            \hspace{1.3cm}\verb!$\overleftarrow{aaa}$!: $\overleftarrow{aaa}$
        \end{itemize}
    \end{itemize}
\end{frame}

\begin{frame}[fragile]{Fuentes matemáticas.}
\begin{itemize}
    \item \LaTeX permite ciertas fuentes especiales
    \begin{itemize}
        \item Para los conjuntos especiales
        \begin{itemize}
            \item \verb!$\mathbb{N,Z,R,C}$!: $\mathbb{N,Z,R,C}$
        \end{itemize}\pause
        \item Cursiva
        \begin{itemize}
            \item \verb!$\mathit{N,Z,R,C}$!: $\mathit{N,Z,R,C}$
        \end{itemize}\pause
    \item Negrita
        \begin{itemize}
            \item \verb!$\mathbf{N,Z,R,C}$!: $\mathbf{N,Z,R,C}$
        \end{itemize}\pause
    \item Caligráfica
        \begin{itemize}
            \item \verb!$\mathcal{N,Z,R,C}$!: $\mathcal{N,Z,R,C}$
            \item \verb!$\mathscr{N,Z,R,C,L}$!: $\mathscr{N,Z,R,C,L}$
        \end{itemize}
    \end{itemize}
\end{itemize}    
\end{frame}

\section{Escritura de fórmulas y proposiciones complejas.}

\begin{frame}{Contenido.}
  \tableofcontents[currentsection]
\end{frame}

\begin{frame}[fragile]{Ecuaciones de más de una línea.}
%split
\LaTeX permite partir las ecuaciones en múltiples lineas
\begin{verbatim}
\begin{multline}
    p(x) = 3x^6 + 14x^5y + 590x^4y^2 + 19x^3y^3\\ 
    - 12x^2y^4 - 12xy^5 + 2y^6 - a^3b^3
\end{multline}
\end{verbatim}

\begin{multline}
p(x) = 3x^6 + 14x^5y + 590x^4y^2 + 19x^3y^3\\ 
- 12x^2y^4 - 12xy^5 + 2y^6 - a^3b^3
\end{multline}
    
\end{frame}

\begin{frame}[fragile]{Alinear ecuaciones.}
%align y eqnarray
    Hay dos formas de alinear ecuaciones:
    \begin{itemize}
    \begin{multicols}{2}
    \setlength{\columnseprule}{1pt} 
        \item Align:\\
        
\begin{verbatim}
\begin{align}
    y&=a(x+b)\\
    &=ax+ab \nonumber
\end{align}
\end{verbatim}
        \begin{align}
                y&=a(x+b)\\
                &=ax+ab \nonumber
            \end{align}
            \pause
        \columnbreak
        
        \item Eqnarray:\\
        
\vspace*{-0.2cm}\begin{verbatim}
\begin{eqnarray}
    y&=&a(x+b)\\
    &=&ax+ab
\end{eqnarray}
\end{verbatim}
\vspace*{-0.7cm}\begin{eqnarray}
                y&=&a(x+b)\\
                &=&ax+ab
            \end{eqnarray}
    \end{multicols}
    \end{itemize}
\end{frame}

\begin{frame}[fragile]{Funciones a trozos.}
%cases
\LaTeX permite escribir funciones a trozos de una manera sencilla usando el ambiente \verb!cases!:
\begin{multicols}{2}
\begin{verbatim}
\begin{equation*}
f(x)=\begin{cases}
2 & \text{si } x<2 \\
\ln x & \text{si } x\geq2
\end{cases}
\end{equation*}
\end{verbatim}

\begin{equation*}
f(x)=\begin{cases}
2 & \text{si } x<2 \\
\ln x & \text{si } x\geq2
\end{cases}
\end{equation*}
\end{multicols}
\end{frame}

\begin{frame}[fragile]{Estética de las ecuaciones}
    \LaTeX permite ponerle elementos a las ecuaciones para que se vean mejor o simplemente se entiendan más fácilmente por el lector.
    \begin{itemize}
        \item Fracciones en diagonal: importar el paquete \verb!xfrac! \\
        \verb!$\sfrac{2}{5}$!: $\sfrac{2}{5}$\pause
        \item Overset: \\
        \verb!$x_n\overset{n\to\infty}{\rightarrow}10$!: $x_n\overset{n\to \infty}{\rightarrow}10$
        \item Underset:\\
        \verb!$x_n\underset{n\to\infty}{\rightarrow}10$!: $x_n\underset{n\to \infty}{\rightarrow}10$\pause
        \item Overbrace:\\
        \small{\verb!$\overbrace{x_n\rightarrow 10}^{\text{esto es un limite}}$!}: $\overbrace{x_n\rightarrow 10}^{\text{esto es un limite}}$
        \item Underbrace\\
        \small{\verb!$\underbrace{x_n\rightarrow 10}_{\text{esto es un limite}}$!}: $\underbrace{x_n\rightarrow 10}_{\text{esto es un limite}}$
    \end{itemize}
\end{frame}

\section{Ejercicio.}
\begin{frame}{Contenido.}
  \tableofcontents[currentsection]
\end{frame}

\begin{frame}{Ejercicio. (Actividad 1: nombre.tex y nombre.pdf)}
Transcriba las siguientes ecuaciones: 
\begin{enumerate}
    \item $$\forall x,y,z \in V,\ (x+y)+z=x+(y+z)$$
    \item $$\underset{u(t)}{\text{máx }}\int\limits e^{-\rho t}\left[P(x(t))-C(u(t))\right]\mathrm{d}t$$
    \item $$u_C(t)=\begin{cases}2-\left(t_C-t_J\right)^2& \text{si } t_C<t_J\\ -\left(t_C-t_J\right)^2 &\text{de lo contrario}\end{cases}$$
    \item $$A=\begin{pmatrix}
    a_{11}&a_{12}&\dots& a_{1n}\\ 
    a_{21}& a_{22}& \dots &a_{2n}\\ 
    \vdots & \vdots & \ddots & \vdots\\
    a_{n1}& a_{n2}& \dots & a_{nn}
    \end{pmatrix}$$
\end{enumerate}

    
\end{frame}




{
% all template changes are local to this group.
    \setbeamertemplate{navigation symbols}{}
    \setbeamercolor{background canvas}{bg=colorClase}
    \begin{frame}[plain, noframenumbering]
    \vfill
    \begin{center}
    \begin{Huge}
        %\textcolor{white}{Gracias!}
    \end{Huge}
    \end{center}
    \vfill
     \end{frame}
}

\end{document}