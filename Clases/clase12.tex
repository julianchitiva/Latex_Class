\documentclass[dvipsnames,xcolor=x11names, handout]{beamer}
\usepackage[spanish]{babel}
\usepackage[utf8]{inputenc}
\usepackage[all]{xy}
\usepackage{afterpage}
\usepackage{tikz}
\usepackage{cancel}
\usepackage{verbatim}
\usepackage{tabu}
\usepackage{xfrac}
\usepackage{mathrsfs}
\usepackage{amsthm}
\usepackage{amssymb}
\usepackage{bbm}
\usepackage{enumerate}
\usepackage{booktabs}
\usepackage{relsize}
\usepackage{hyperref}
\usepackage{float}
\usepackage{longtable}
\usepackage{amsmath}
\usepackage{multirow}
\usepackage{multicol}
\usepackage{colortbl}
\usepackage{adjustbox}
\usepackage{xfrac}
\usepackage{bm}
\usepackage{keystroke}
\usepackage{wrapfig}
\usepackage{graphicx}
\usepackage{csvsimple}
\usepackage{color, soul}
\usepackage{otros/pgf-pie}
\usetikzlibrary{decorations.pathmorphing, patterns,shapes}
\usetikzlibrary{positioning}
\usepackage{pgfplots}
\pgfplotsset{compat=1.12}
\usepackage{pgfplotstable}
\usepackage{natbib}


\PassOptionsToPackage{demo}{graphicx}

\newcommand{\hlc}[2][yellow]{ {\sethlcolor{#1} \hl{#2}} }

\newcommand*{\rom}[1]{\expandafter\romannumeral #1}
\newcommand{\Rom}[1]{\uppercase\expandafter{\romannumeral #1\relax}}

\newcommand{\Importante}[2]{{\color{#1}#2}}
\newcommand{\importante}[2]{{\color{#1}\underline{#2}}}

\renewcommand{\baselinestretch}{1}
\setlength{\parskip}{\baselineskip}

\usetheme{Boadilla}
\definecolor{colorClase}{RGB}{2,70,56}
\usecolortheme[named=colorClase]{structure}
\usepackage{natbib}


\theoremstyle{plain}
  \newtheorem{teorema}{Teorema}
  \newtheorem{proposicion}{Proposición}
  \newtheorem{corolario}{Corolario}
  \newtheorem{lema}[teorema]{Lema}
\theoremstyle{definition}
  \newtheorem{definicion}{Definici\'on}
  \newtheorem{ejemplo}{Ejemplo}
  
  
\setbeamertemplate{caption}[numbered]
 
\title{Taller usos de \LaTeX}
\subtitle{Manejo de proyectos ``grandes''.}
\setbeamersize{text margin left=25pt,text margin right=25pt}

\author[Julián Chitiva Bocanegra]{Julián Enrique Chitiva Bocanegra}
\institute[Uniandes] 
{Universidad de los Andes\\ Facultad de Economía}
\titlegraphic{\includegraphics[width=0.8cm]{img/uniandes_logo.png}
}

% \pgfdeclareimage[height=.8cm]{university-logo}{img/uniandes_logo.png}
% \logo{\pgfuseimage{university-logo}}

\date{\today}
\subject{}
\usepackage{caption}
\usepackage{subcaption}

\begin{document}
\begin{frame}
  \titlepage
\end{frame}

\begin{frame}{Contenido.}
  \tableofcontents%[hideallsubsections]%, currentsection]
\end{frame}
\section{Manejo de \itshape inputs.}
\begin{frame}{Contenido.}
  \tableofcontents[currentsection]
\end{frame}

\begin{frame}[fragile]{Manejo de \itshape inputs \normalfont ``.tex''.}
\begin{itemize}
    \item La forma de incluir archivos ``.tex'' dentro de otro archivo ``.tex'' es mediante el comando \verb!\input{archivoSecundario}! o \verb!\include{archivoSecundario}!
    \begin{itemize}
    \item \verb!\input! es como si escribieramos el contenido del ``.tex'' en el nuevo
    \item \verb!\include! es igual que \verb!\input! pero insertando un \verb!\clearpage! y crea un archivo ``.aux'' para cada uno de los archivos incluidos.
    \end{itemize}
    \item \verb!\include! permite incluir o excluir solo algunos archivos usando \verb!\includeonly{<lista*>}! y \verb!\excludeonly{<lista*>}!
\end{itemize}
\end{frame}

\begin{frame}[fragile]{Manejo de \itshape inputs \normalfont ``.pdf''.}
\begin{itemize}
    \item Toca importar el paquete \verb!pdfpages!
    \begin{itemize}[<+->]
        \item Para insertar todo el pdf \verb!\includepdf[pages=-]{archivo.pdf}!
        \item Para insertar solo páginas 1 y 2 del pdf \verb!\includepdf[pages={1,2}]{archivo.pdf}!
    \end{itemize}
\end{itemize}

\end{frame}

\section{Creación de paquetes.}
\begin{frame}{Contenido.}
  \tableofcontents[currentsection]
\end{frame}
\begin{frame}[fragile]{Creación de paquetes.}
Crear paquetes es muy sencillo. \pause Solo es necesario poner 
    
\begin{verbatim}
\ProvidesPackage{nombrePaquete}
\end{verbatim}

En la primera linea de un archivo ``.sty''.\pause

Después se pone todo el código correspondiente a las funciones y paquetes que necesitemos para el desarrollo de nuestro paquete (Yo lo uso solo para poder poner todo el preámbulo en otro archivo y que todo sea mas ordenado). 
\end{frame}

\begin{frame}[fragile]{Programación de funciones.}
\begin{itemize}[<+->]
    \item Hemos visto que el comando \verb!\renewcommand! sirve para sobrescribir la programación de un comando o valor predeterminado en \LaTeX\ .
    \item Para crear nuevos comandos el comando es \verb!\newcommmand! Este nos permite definir el número de parámetros necesarios para la función y sus valores por defecto.\pause
    \begin{ejemplo}
    \begin{enumerate}[<+->]
    \item \verb!\newcommand*{\rom}[1]{\expandafter\romannumeral #1}!
    \item \scriptsize{\verb!\newcommand{\Rom}[1]{\uppercase\expandafter{\romannumeral #1\relax}}!}
    \item \verb!\newcommand{\Importante}[2][red]{{\color{#1}#2}}!
    \end{enumerate}
    \end{ejemplo}
\end{itemize}
\end{frame}

\section{Ventajas y desventajas.}
\begin{frame}{Contenido.}
  \tableofcontents[currentsection]
\end{frame}
{\renewcommand{\columnseprule}{1pt}
\begin{frame}{Ventajas.}
\begin{multicols}{2}
\begin{enumerate}
\item ORDEN! 
\item Hay archivos más cortos y es más fácil buscar errores. 
\end{enumerate}
    
\columnbreak
    
\begin{enumerate}
\item Pueden ser muchos archivos y buscar las cosas puede ser más difícil. 
\end{enumerate}
\end{multicols}

%\hlc{blue}{hola}
{\sethlcolor{blue} \hl{hola}}
\end{frame}}
\end{document}