\documentclass[dvipsnames,xcolor=x11names]{beamer}
\usepackage[spanish]{babel}
\usepackage[utf8]{inputenc}
\usepackage[all]{xy}
\usepackage{afterpage}
\usepackage{tikz}
\usepackage{cancel}
\usepackage{verbatim}
\usepackage{tabu}
\usepackage{xfrac}
\usepackage{mathrsfs}
\usepackage{amsthm}
\usepackage{amssymb}
\usepackage{bbm}
\usepackage{enumerate}
\usepackage{booktabs}
\usepackage{relsize}
\usepackage{hyperref}
\usepackage{float}
\usepackage{longtable}
\usepackage{amsmath}
\usepackage{multirow}
\usepackage{multicol}
\usepackage{colortbl}
\usepackage{adjustbox}
\usepackage{xfrac}
\usepackage{bm}
\usepackage{keystroke}
\usepackage{wrapfig}
\usepackage{graphicx}
\usepackage{csvsimple}
\usepackage{otros/pgf-pie}
\usetikzlibrary{decorations.pathmorphing, patterns,shapes}
\usetikzlibrary{positioning}
\usepackage{pgfplots}
\pgfplotsset{compat=1.12}
\usepackage{pgfplotstable}
\usepackage{natbib}


\PassOptionsToPackage{demo}{graphicx}

\newcommand{\hlc}[2][yellow]{ {\sethlcolor{#1} \hl{#2}} }

\newcommand*{\rom}[1]{\expandafter\romannumeral #1}
\newcommand{\Rom}[1]{\uppercase\expandafter{\romannumeral #1\relax}}

\newcommand{\Importante}[2]{{\color{#1}#2}}
\newcommand{\importante}[2]{{\color{#1}\underline{#2}}}

\renewcommand{\baselinestretch}{1}
\setlength{\parskip}{\baselineskip}

\usetheme{Boadilla}
\definecolor{colorClase}{RGB}{108,29,69}
\usecolortheme[named=colorClase]{structure}
\usepackage{natbib}


\theoremstyle{plain}
  \newtheorem{teorema}{Teorema}
  \newtheorem{proposicion}{Proposición}
  \newtheorem{corolario}{Corolario}
  \newtheorem{lema}[teorema]{Lema}
\theoremstyle{definition}
  \newtheorem{definicion}{Definici\'on}
  \newtheorem{ejemplo}{Ejemplo}
  
  
\setbeamertemplate{caption}[numbered]
 
\title{Taller usos de \LaTeX}
\subtitle{Manejo de bibliografía.}
\setbeamersize{text margin left=25pt,text margin right=25pt}

\author[Julián Chitiva Bocanegra]{Julián Enrique Chitiva Bocanegra}
\institute[Uniandes] 
{Universidad de los Andes\\ Facultad de Economía}
\titlegraphic{\includegraphics[width=0.8cm]{img/uniandes_logo.png}
}

\pgfdeclareimage[height=.8cm]{university-logo}{img/uniandes_logo.png}
\logo{\pgfuseimage{university-logo}}

\date{\today}
\subject{}
\usepackage{caption}
\usepackage{subcaption}

\begin{document}

\begin{frame}
  \titlepage
\end{frame}

\begin{frame}{Contenido.}
  \tableofcontents%[hideallsubsections]%, currentsection]
\end{frame}
\section{\textquestiondown Cómo construir la bibliografía?}
\begin{frame}{Contenido.}
  \tableofcontents[currentsection]
\end{frame}
\begin{frame}[fragile]{\textquestiondown Cómo construir la bibliografía?}
Primero es importante usar el paquete \verb!natbib! que permite insertar la bibliografía de manera sencilla. \\~\\

La bibliografía se construye llenando campos necesarios de acuerdo a lo que uno quiera citar:

\begin{block}{Ejemplo:}
\begin{verbatim}
@article{greenwade93,
    author  = "George D. Greenwade",
    title   = "The {C}omprehensive
    {T}ex {A}rchive {N}etwork ({CTAN})",
    year    = "1993",
    journal = "TUGBoat",
    volume  = "14",
    number  = "3",
    pages   = "342--351"
}
\end{verbatim}
\end{block}

Pueden encontrar los campos necesarios en el siguiente \href{https://en.wikibooks.org/wiki/LaTeX/Bibliography_Management}{\textcolor{colorClase}{link}}
\end{frame}
\begin{frame}[fragile]{\textquestiondown Cómo construir la bibliografía?}

Pueden encontrar los campos necesarios para los tipos de elementos a referenciar en el siguiente \href{https://en.wikibooks.org/wiki/LaTeX/Bibliography_Management}{\textcolor{colorClase}{link}}\\~\\

La forma más fácil de construir estas referencias es buscandolas en internet.
\end{frame}

\section{\textquestiondown Dónde buscar bibliografía?}
\begin{frame}{Contenido.}
  \tableofcontents[currentsection]
\end{frame}

\begin{frame}{\textquestiondown Dónde buscar bibliografía?}
    \begin{enumerate}[<+->]
    \item \href{https://scholar.google.com.co}{google scholar}
    \item \href{https://google.com.co}{google} Nombre del artículo o libro .bib
    \end{enumerate}
\end{frame}

\section{\textquestiondown Cómo citar en \LaTeX\ ?}
\begin{frame}{Contenido.}
  \tableofcontents[currentsection]
\end{frame}

\begin{frame}[fragile]{\textquestiondown Cómo citar en \LaTeX\ ?}

Como estamos usando el paquete \verb!natbib! la forma más sencilla es usando el comando \verb!\cite{nombreReferencia}!

\begin{tabular}{l|l}
\verb!\cite{greenwade93}! &  \cite{greenwade93}\\
\verb!\citet{greenwade93}! &\citet{greenwade93}\\
\verb!\citep{greenwade93}! & \citep{greenwade93}	\\	
\verb!\citet*{greenwade93}! &\citet*{greenwade93}\\
\verb!\citep*{greenwade93}! & \citep*{greenwade93}\\
\verb!\citeauthor{greenwade93}! & \citeauthor{greenwade93}\\
\verb!\citeauthor*{greenwade93}! & \citeauthor*{greenwade93}\\
\verb!\citeyear{greenwade93}! & \citeyear{greenwade93}\\
\verb!\citeyearpar{greenwade93}! &\citeyearpar{greenwade93}\\
\verb!\citealt{greenwade93}! & \citealt{greenwade93}\\
\verb!\citealp{greenwade93}! & \citealp{greenwade93}\\
\verb!\citetext{priv.\ comm.}! & \citetext{priv.\ comm.}
\end{tabular}
    
\end{frame}

\section{\textquestiondown Cómo mostrar la bibliografía \LaTeX\ ?}
\begin{frame}{Contenido.}
  \tableofcontents[currentsection]
\end{frame}
\begin{frame}[fragile]{\textquestiondown Cómo mostrar la bibliografía \LaTeX\ ?}
 Para que la bibliografía \textbf{aparezca} es importante poner los comandos
 \begin{verbatim}
  \bibliographystyle{estilo}
  \bibliography{archivo .bib}
 \end{verbatim}
 \vspace*{-0.5cm}
 Es importante saber que solo lo que hayan citado dentro del texto aparecerá en las referencias. Si quieren que les aparezcan referencias no citadas deben usar \verb!\nocite{nombreReferencia}! para que les aparezca solo esa referencia o \verb!\nocite{*}! para que aparezcan todas las referencias en el \verb!.bib!. \pause
 
 Los posibles estilos de citación\footnote{Pueden ver cómo son cada una de estas bibliografías en el siguiente \href{https://www.overleaf.com/learn/latex/Bibtex_bibliography_styles#Further_reading}{\textcolor{colorClase}{link}}} son: \verb!abbrv!, \verb!acm!, \verb!alpha!, \verb!apalike!, \verb!ieeetr!, \verb!plain!, \verb!siam!,\verb!unsrt!.
 
 
\end{frame}


\section{Bibliografía}
\begin{frame}
      \bibliographystyle{apalike}
  \bibliography{Admin/biblio}
\end{frame}
\end{document}