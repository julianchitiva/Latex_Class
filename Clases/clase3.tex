\documentclass[dvipsnames,xcolor, handout]{beamer}
\usepackage[spanish]{babel}
\usepackage[utf8]{inputenc}
\usepackage[all]{xy}
\usepackage{afterpage}
\usepackage{tikz}
\usepackage{cancel}
\usepackage{verbatim}
\usepackage{tabu}
\usepackage{xfrac}
\usepackage{mathrsfs}
\usepackage{amsthm}
\usepackage{amssymb}
\usepackage{bbm}
\usepackage{enumerate}
\usepackage{booktabs}
\usepackage{relsize}
\usepackage{hyperref}
\usepackage{float}
\usepackage{longtable}
%\usetikzlibrary{decorations.pathmorphing, patterns,shapes}
%\usetikzlibrary{positioning}
%\usepackage{pgfplots}
%\pgfplotsset{compat=1.12}
%\PassOptionsToPackage{demo}{graphicx}

%\usepackage{array}

\usepackage{amsmath}
\usepackage{multirow}
\usepackage{multicol}
\usepackage{colortbl}
\usepackage{adjustbox}
\usepackage{xfrac}
\usepackage{bm}
\usepackage{keystroke}

\newcommand{\hlc}[2][yellow]{ {\sethlcolor{#1} \hl{#2}} }

\newcommand*{\rom}[1]{\expandafter\romannumeral #1}
\newcommand{\Rom}[1]{\uppercase\expandafter{\romannumeral #1\relax}}

\newcommand{\Importante}[2]{{\color{#1}#2}}
\newcommand{\importante}[2]{{\color{#1}\underline{#2}}}

\renewcommand{\baselinestretch}{1}
\setlength{\parskip}{\baselineskip}


\def\Put(#1,#2)#3{\leavevmode\makebox(0,0){\put(#1,#2){#3}}}

 \usetheme{Boadilla}

%\usecolortheme{crane}
\definecolor{colorClase}{rgb}{0,0.188,0.529}
\usecolortheme[named=colorClase]{structure}
\usepackage{natbib}


\theoremstyle{plain}
  \newtheorem{teorema}{Teorema}
  \newtheorem{proposicion}{Proposición}
  \newtheorem{corolario}{Corolario}
  \newtheorem{lema}[teorema]{Lema}
\theoremstyle{definition}
  \newtheorem{definicion}{Definici\'on}
  \newtheorem{ejemplo}{Ejemplo}
  
  
\makeatletter
\setbeamertemplate{footline}
{
  \leavevmode%
  \hbox{%
  \begin{beamercolorbox}[wd=.4\paperwidth,ht=2.25ex,dp=1ex,center]{author in head/foot}%
    \usebeamerfont{author in head/foot}\insertshortauthor \hspace*{1em}(\insertshortinstitute)
  \end{beamercolorbox}%
  \begin{beamercolorbox}[wd=.5\paperwidth,ht=2.25ex,dp=1ex,center]{title in head/foot}%
    \usebeamerfont{title in head/foot}\insertsection 
  \end{beamercolorbox}%
  \begin{beamercolorbox}[wd=.1\paperwidth,ht=2.25ex,dp=1ex,center]{date in head/foot}%
    \usebeamerfont{date in head/foot}
    \insertframenumber{} / \inserttotalframenumber\hspace*{2ex} 
  \end{beamercolorbox}}%
  \vskip0pt%
}
\makeatother
\setbeamertemplate{caption}[numbered]
 
\title{Taller usos de \LaTeX \\ \small{\textit{Minipage} y Tablas} \vspace*{-0.2cm}}

\setbeamersize{text margin left=25pt,text margin right=25pt}

\author[Julián Chitiva Bocanegra]{Julián Enrique Chitiva Bocanegra}
\institute[Uniandes] 
{Universidad de los Andes\\ Facultad de Economía}
\titlegraphic{\includegraphics[width=0.8cm]{img/uniandes_logo.png}
}

\date{\today}

\subject{}
\usepackage[figurename=]{caption}
\begin{document}

\begin{frame}
  \titlepage
\end{frame}

\begin{frame}{Contenido.}
\begin{multicols}{2}
  \tableofcontents
\end{multicols}
\end{frame}

\section{\itshape Minipages}
\begin{frame}{Contenido.}
\begin{multicols}{2}
  \tableofcontents[currentsection]
\end{multicols}
\end{frame}

\begin{frame}[fragile]{\protect\textit{Minipages}}
La idea de un \textit{minipage} es poder construir una página dentro de una página. Normalmente se usa para:
\begin{itemize}
    \item Poner dos tablas una al lado de otra.
    \item Poner dos figuras una al lado de otra.
    \item Poner texto al lado de una tabla o una figura.
\end{itemize}
\end{frame}

\begin{frame}[fragile]{Sintaxis de un \protect\textit{minipage}}
\begin{verbatim}
\begin{minipage}[ajuste]{ancho del minipage}
Texto ... \\
Imágenes ... \\
Tablas ... \\
\end{minipage} 
\end{verbatim}
\begin{itemize}
    \item Ajuste: puede ser \verb!c! centrado, \verb!t! arriba o \verb!b! abajo 
    \item Ancho: puede ser absoluto o relativo
    \begin{itemize}
        \item Absoluto: \verb!6cm, 2in, 2pt, 3ex, 5em!
        \item Relativo: representa una proporción de una medida predeterminada de \LaTeX\ como \verb!\textwidth!$=$\the\textwidth\  o \verb!\linewidth!$=$\the\linewidth
    \end{itemize}
\end{itemize}
\end{frame}

\begin{frame}[fragile]{Medidas en \LaTeX\ }
\begin{tabular}{ccc}\hline
    Unidad & Definición & Valor en pt \\ \hline
    pt & un punto es $\pm$ 0.35 mm	& 1	\\
    mm	& milímetro	& 2.84 \\
    cm	& a centímetro	& 28.4 \\
    in	& pulgada	& 72.27 \\
    ex	& $\pm$ la altura de una `x' en la fuente actual &	- \\
    em	& $\pm$ el ancho de una `M' en la fuente actual	& -
\end{tabular}
    
    Tambien pueden definir sus longitudes como 
    \vspace*{-0.8cm}\begin{verbatim}
    \setlength{\mylength}{longitud}
\end{verbatim}
\end{frame}

\begin{frame}{Ejemplos}
    \begin{minipage}{0.55\linewidth}
     Es un ejemplo de dos \textit{minipage} pegados
    \end{minipage}
    \begin{minipage}{0.35\linewidth}
    \textcolor{blue}{\rule{\textwidth}{\textwidth}}
    \end{minipage}
\end{frame}

\begin{frame}[fragile]{Ejemplos}
\begin{verbatim}
    \begin{minipage}{0.55\linewidth}
     Es un ejemplo de dos \textit{minipage} pegados
    \end{minipage}
    \begin{minipage}{0.35\linewidth}
    \textcolor{blue}{\rule{\textwidth}{\textwidth}}
    \end{minipage}
\end{verbatim}
\end{frame}

\begin{frame}{Ejemplos}
    \begin{minipage}{0.49\linewidth}
    \textcolor{blue}{\rule{\textwidth}{\textwidth}}
    \end{minipage}
    \begin{minipage}{0.49\linewidth}
    \textcolor{red}{\rule{\textwidth}{
    \textwidth}}
    \end{minipage}
\end{frame}

\begin{frame}[fragile]{Ejemplos}
\begin{verbatim}
    \begin{minipage}{0.49\linewidth}
    \textcolor{blue}{\rule{\textwidth}{\textwidth}}
    \end{minipage}
    \begin{minipage}{0.49\linewidth}
    \textcolor{red}{\rule{\textwidth}{
    \textwidth}}
    \end{minipage}
\end{verbatim}
\end{frame}

\begin{frame}{Ejemplos}
    \begin{minipage}{0.35\linewidth}
     Es un ejemplo de tres \textit{minipage} pegados
    \end{minipage}
    \begin{minipage}{0.3\linewidth}
    \textcolor{blue}{\rule{\textwidth}{\textwidth}}
    \end{minipage}
    \begin{minipage}{0.3\linewidth}
    \textcolor{red}{\rule{\textwidth}{
    \textwidth}}
    \end{minipage}
\end{frame}

\begin{frame}[fragile]{Ejemplos}
\begin{verbatim}
    \begin{minipage}{0.35\linewidth}
     Es un ejemplo de tres \textit{minipage} pegados
    \end{minipage}
    \begin{minipage}{0.3\linewidth}
    \textcolor{blue}{\rule{\textwidth}{\textwidth}}
    \end{minipage}
    \begin{minipage}{0.3\linewidth}
    \textcolor{red}{\rule{\textwidth}{
    \textwidth}}
    \end{minipage}
\end{verbatim}
\end{frame}

\begin{frame}[fragile]{Ejemplos}
    \begin{minipage}{0.45\linewidth}
     Es un ejemplo de cuatro \textit{minipage}. Si se da\Enter entre el \verb!\end{minipage}! y el \verb!\begin{minipage}! se pondrá en la linea de abajo.
    \end{minipage}
    \begin{minipage}{0.4\linewidth}
    \hfill\textcolor{blue}{\rule{0.8\textwidth}{0.6\textwidth}}
    \end{minipage}\pause
    
    \begin{minipage}{0.25\linewidth}
    \textcolor{orange}{\rule{\textwidth}{0.6\textwidth}}
    \end{minipage}
    \begin{minipage}{0.7\linewidth}
    \textcolor{red}{\rule{\textwidth}{1cm}}
    \end{minipage}
\end{frame}

\begin{frame}[fragile]{Ejemplos}
\begin{small}
\begin{verbatim}
\begin{minipage}{0.45\linewidth}
    Es un ejemplo de cuatro \textit{minipage}. Si se da
    \Enter entre el \end{minipage} y el \begin{minipage}
    se pondrá en la linea de abajo.
\end{minipage}
\begin{minipage}{0.4\linewidth}
    \hfill\textcolor{blue}{\rule{0.8\textwidth}{0.6\textwidth}}
\end{minipage}

\begin{minipage}{0.25\linewidth}
    \textcolor{orange}{\rule{\textwidth}{0.6\textwidth}}
\end{minipage}
\begin{minipage}{0.7\linewidth}
    \textcolor{red}{\rule{\textwidth}{1cm}}
\end{minipage}
\end{verbatim}
\end{small}
\end{frame}

\section{\itshape Tabular}
\begin{frame}{Contenido.}
\begin{multicols}{2}
  \tableofcontents[currentsection]
\end{multicols}
\end{frame}

\begin{frame}[fragile]{Estructura básica}
\verb!Tabular! es el ambiente predeterminado de \LaTeX\ para crear tablas. 

\begin{minipage}{0.57\linewidth}
\begin{verbatim}
\begin{tabular}{|c|| l |||r|}
    \hline
    Cent. & Izq. & Der. \\
    d & e & f\\
    \hline
\end{tabular}
\end{verbatim}
\end{minipage}
\begin{minipage}{0.1\linewidth}
    $\rightarrow$
\end{minipage}
\begin{minipage}{0.3\linewidth}
    \begin{tabular}[b]{|c|| l |||r|}
    \hline
    Cent. & Izq. & Der. \\
    d & e & f\\
    \hline
\end{tabular}
\end{minipage}
    
\end{frame}

\begin{frame}[fragile]{Estructura básica}
Hay una forma ``rápida'' de crear una tabla con muchas columnas.

\begin{minipage}{0.57\linewidth}
\begin{verbatim}
\begin{tabular}{|*3{c|}}
    \hline
    Cent. & Izq. & Der. \\
    d & e & f\\
    \hline
\end{tabular}
\end{verbatim}
\end{minipage}
\begin{minipage}{0.1\linewidth}
    $\rightarrow$
\end{minipage}
\begin{minipage}{0.3\linewidth}
    \begin{tabular}[b]{|*3{c|}}
    \hline
    Cent. & Izq. & Der. \\
    d & e & f\\
    \hline
\end{tabular}
\end{minipage}
    
\end{frame}

\section{Table}
\begin{frame}{Contenido.}
\begin{multicols}{2}
  \tableofcontents[currentsection]
\end{multicols}
\end{frame}

\begin{frame}[fragile]{Estructura básica}
\verb!Table! es el ambiente que permite referenciar y ponerle títulos a las tablas. 

\begin{minipage}{0.57\linewidth}
\begin{verbatim}
\begin{table}[]
    \centering
    \begin{tabular}{c|c}
        a & b \\
        c & d
    \end{tabular}
    \caption{Caption}
    \label{tab:my_label}
\end{table}
\end{verbatim}
\end{minipage}
\begin{minipage}{0.1\linewidth}
    $\rightarrow$
\end{minipage}
\begin{minipage}{0.3\linewidth}
    \begin{table}[]
    \centering
    \begin{tabular}{c|c}
        a & b \\
        c & d
    \end{tabular}
    \caption{Título de la tabla}
    \label{tab:my_label}
\end{table}
\end{minipage}
\begin{small}

    En el Cuadro \ref{tab:my_label} podemos ver un ejemplo del ambiente \verb!Table!. 
    En los corchetes se puede poner \vspace*{-0.5cm} \begin{verbatim}
    h, h!, H, t, b
\end{verbatim}
    \vspace*{-0.5cm}(para poner \verb!H! se debe importar el paquete \verb!float!)
\end{small}
\end{frame}

\begin{frame}[fragile]{Referencias de tablas.}
\begin{itemize}
    \item Si se quiere poner la lista de todas las tablas en el documento hay que usar el paquete \verb!\listoftables!
    \item si se quiere cambiar los nombres de las tablas en el preámbulo hay que poner \verb!\renewcommand\spanishtablename{Otro nombre}!
\end{itemize}
\end{frame}

\section{Diseño de tablas}
\begin{frame}{Contenido.}
\begin{multicols}{2}
  \tableofcontents[currentsection]
\end{multicols}
\end{frame}

\subsection{Celdas de ancho fijo.}
\begin{frame}{Contenido.}
\begin{multicols}{2}
  \tableofcontents[currentsubsection]
\end{multicols}
\end{frame}

\begin{frame}[fragile]{Celdas de ancho fijo.}
Si el texto dentro de las celdas es muy largo, \LaTeX\ sigue escribiendo hasta salirse de la hoja.

\begin{small}
\begin{tabular}{|c|l|r|}
\hline
Alinea con la linea de arriba & 
Alinea con la linea de la mitad &
Alinea con la linea de abajo \\
\hline
\end{tabular}
\end{small}\pause

\begin{minipage}{0.5\linewidth}
\begin{scriptsize}
\begin{verbatim}
\begin{tabular}{|p{1cm}|m{1cm}|b{1cm}|}
\hline
Alinea con la linea de arriba & 
Alinea con la linea de la mitad &
Alinea con la linea de abajo \\
\hline
\end{tabular}
\end{verbatim}
\end{scriptsize}
\end{minipage}
\begin{minipage}{0.05\linewidth}
    $\rightarrow$
\end{minipage}
\begin{minipage}{0.4\linewidth}
\begin{scriptsize}
\begin{tabular}{|p{1cm}|m{1cm}|b{1cm}|}
\hline
Alinea con la linea de arriba & 
Alinea con la linea de la mitad &
Alinea con la linea de abajo \\
\hline
\end{tabular}
    \end{scriptsize}
\end{minipage}
\end{frame}

\subsection{Tablas de ancho fijo.}
\begin{frame}{Contenido.}
\begin{multicols}{2}
  \tableofcontents[currentsubsection]
\end{multicols}
\end{frame}

\begin{frame}[fragile]{Tablas de ancho fijo.}
    Si queremos tablas de ancho fijo tenemos que usar el paquete \verb!tabu!. La sintaxis es la siguiente:\\\pause
\begin{verbatim}
\begin{tabu} to 0.8\textwidth { | X[l] | X[c] | X[r] | }
\hline
aaaaa & bbbbb & ccccc \\
\hline
ddddd  & eeeee  & fffff  \\
\hline
\end{tabu}
\end{verbatim}\pause

    \begin{tabu} to 0.8\textwidth { | X[l] | X[c] | X[r] | }
    \hline
    aaaaa & bbbbb & ccccc \\
    \hline
    ddddd  & eeeee  & fffff  \\
    \hline
    \end{tabu}
\end{frame}

\begin{frame}[fragile]{Tablas de ancho fijo.}
    Si queremos tablas de ancho fijo tenemos que usar el paquete \verb!tabu!. \\
\begin{verbatim}
\begin{tabu} to 0.4\textwidth { | X[l] | X[c] | X[r] | }
\hline
aaaaa & bbbbb & ccccc \\
\hline
ddddd  & eeeee  & fffff  \\
\hline
\end{tabu}
\end{verbatim}\pause

    \begin{tabu} to 0.4\textwidth { | X[l] | X[c] | X[r] | }
    \hline
    aaaaa & bbbbb & ccccc \\
    \hline
    ddddd  & eeeee  & fffff  \\
    \hline
    \end{tabu}
\end{frame}

\subsection{Escalar tablas.}
\begin{frame}{Contenido.}
\begin{multicols}{2}
  \tableofcontents[currentsubsection]
\end{multicols}
\end{frame}

\begin{frame}[fragile]{Escalar tablas.}
Hay dos maneras de escalar tablas:
\begin{enumerate}
    \item \verb!\scalebox{Escala}{Contenido}!: Permite escalar todo el contenido proporcionalmente
    \item \verb!\resizebox{Ancho}{Alto}{Contenido}!: Permite escalar el ancho y alto por separado, también permite mantener la escala.
\end{enumerate}

Para usar estas funciones hay que importar el paquete \verb!graphicx!
\end{frame}

\begin{frame}[fragile]{Escalar tablas: scalebox.}
\begin{center}
\footnotesize
\begin{tabular}{|p{1cm}|m{1cm}|b{1cm}|}
\hline
Alinea con la linea de arriba & 
Alinea con la linea de la mitad &
Alinea con la linea de abajo \\
\hline
\end{tabular}
\end{center}
\begin{minipage}{0.6\linewidth}
\footnotesize
\begin{verbatim}
\scalebox{0.7}{
\begin{tabular}{|p{1cm}|m{1cm}|b{1cm}|}
\hline
Alinea con la linea de arriba & 
Alinea con la linea de la mitad &
Alinea con la linea de abajo \\
\hline
\end{tabular}
}
\end{verbatim}
\end{minipage}\pause
\begin{minipage}{0.39\linewidth}
\footnotesize
\centering
\scalebox{0.7}{
\begin{tabular}{|p{1cm}|m{1cm}|b{1cm}|}
\hline
Alinea con la linea de arriba & 
Alinea con la linea de la mitad &
Alinea con la linea de abajo \\
\hline
\end{tabular}
}
\end{minipage}

\end{frame}

\begin{frame}[fragile]{Escalar tablas: scalebox.}
\begin{center}
\footnotesize
\begin{tabular}{|p{1cm}|m{1cm}|b{1cm}|}
\hline
Alinea con la linea de arriba & 
Alinea con la linea de la mitad &
Alinea con la linea de abajo \\
\hline
\end{tabular}
\end{center}
\begin{minipage}{0.6\linewidth}
\footnotesize
\begin{verbatim}
\scalebox{0.4}{
\begin{tabular}{|p{1cm}|m{1cm}|b{1cm}|}
\hline
Alinea con la linea de arriba & 
Alinea con la linea de la mitad &
Alinea con la linea de abajo \\
\hline
\end{tabular}
}
\end{verbatim}
\end{minipage}\pause
\begin{minipage}{0.39\linewidth}
\footnotesize
\centering
\scalebox{0.4}{
\begin{tabular}{|p{1cm}|m{1cm}|b{1cm}|}
\hline
Alinea con la linea de arriba & 
Alinea con la linea de la mitad &
Alinea con la linea de abajo \\
\hline
\end{tabular}
}
\end{minipage}
\end{frame}

\begin{frame}[fragile]{Escalar tablas: resizebox.}
\begin{minipage}{0.6\linewidth}
\footnotesize
\begin{verbatim}
\resizebox{\textwidth}{!}{
\begin{tabular}{|p{1cm}|m{1cm}|b{1cm}|}
\hline
Alinea con la linea de arriba & 
Alinea con la linea de la mitad &
Alinea con la linea de abajo \\
\hline
\end{tabular}
}
\end{verbatim}
\end{minipage}\pause

\begin{minipage}{0.49\linewidth}
\footnotesize
\centering
\begin{tabular}{|p{1cm}|m{1cm}|b{1cm}|}
\hline
Alinea con la linea de arriba & 
Alinea con la linea de la mitad &
Alinea con la linea de abajo \\
\hline
\end{tabular}
\end{minipage}
\begin{minipage}{0.49\linewidth}
\footnotesize
\centering
\resizebox{\textwidth}{!}{
\begin{tabular}{|p{1cm}|m{1cm}|b{1cm}|}
\hline
Alinea con la linea de arriba & 
Alinea con la linea de la mitad &
Alinea con la linea de abajo \\
\hline
\end{tabular}
}
\end{minipage}
\end{frame}

\begin{frame}[fragile]{Escalar tablas: resizebox.}
\begin{minipage}{0.6\linewidth}
\footnotesize
\begin{verbatim}
\resizebox{!}{1cm}{
\begin{tabular}{|p{1cm}|m{1cm}|b{1cm}|}
\hline
Alinea con la linea de arriba & 
Alinea con la linea de la mitad &
Alinea con la linea de abajo \\
\hline
\end{tabular}
}
\end{verbatim}
\end{minipage}\pause

\begin{minipage}{0.49\linewidth}
\footnotesize
\centering
\begin{tabular}{|p{1cm}|m{1cm}|b{1cm}|}
\hline
Alinea con la linea de arriba & 
Alinea con la linea de la mitad &
Alinea con la linea de abajo \\
\hline
\end{tabular}
\end{minipage}
\begin{minipage}{0.49\linewidth}
\footnotesize
\centering
\resizebox{!}{1cm}{
\begin{tabular}{|p{1cm}|m{1cm}|b{1cm}|}
\hline
Alinea con la linea de arriba & 
Alinea con la linea de la mitad &
Alinea con la linea de abajo \\
\hline
\end{tabular}
}
\end{minipage}
\end{frame}

\begin{frame}[fragile]{Escalar tablas: resizebox.}
\begin{minipage}{0.6\linewidth}
\footnotesize
\begin{verbatim}
\resizebox{5cm}{1cm}{
\begin{tabular}{|p{1cm}|m{1cm}|b{1cm}|}
\hline
Alinea con la linea de arriba & 
Alinea con la linea de la mitad &
Alinea con la linea de abajo \\
\hline
\end{tabular}
}
\end{verbatim}
\end{minipage}\pause

\begin{minipage}{0.49\linewidth}
\footnotesize
\centering
\begin{tabular}{|p{1cm}|m{1cm}|b{1cm}|}
\hline
Alinea con la linea de arriba & 
Alinea con la linea de la mitad &
Alinea con la linea de abajo \\
\hline
\end{tabular}
\end{minipage}
\begin{minipage}{0.49\linewidth}
\footnotesize
\centering
\resizebox{5cm}{1cm}{
\begin{tabular}{|p{1cm}|m{1cm}|b{1cm}|}
\hline
Alinea con la linea de arriba & 
Alinea con la linea de la mitad &
Alinea con la linea de abajo \\
\hline
\end{tabular}
}
\end{minipage}
\end{frame}

\subsection{Tablas de más de una página.}
\begin{frame}{Contenido.}
\begin{multicols}{2}
  \tableofcontents[currentsubsection]
\end{multicols}
\end{frame}

\begin{frame}[fragile]{Tablas de más de una página.}
    Para poder escribir tablas de más de una página hay que cargar el paquete \verb!longtable!. La sintaxis es la siguiente:\\\pause
\begin{multicols}{2}
\begin{tiny}
\begin{verbatim}
\begin{longtable}[c]{| c | c |}
 \caption{Nombre de la tabla.\label{etiqueta}}\\
 %Diseño del encabezado
 \hline
 \multicolumn{2}{| c |}{Comienzo de la tabla}\\
 \hline
 Col 1 & Col 2\\
 \hline
 \endfirsthead % Hasta aqui encabezado pag. 1
 
 \hline
 \multicolumn{2}{|c|}{Continuación Tabla
 \ref{etiqueta}}\\
 \hline
  Col 1 & Col 2\\
 \hline
 \endhead %Hasta aqui encabezado pag. secundarias
 
 \hline
 \endfoot %Hasta aqui parte de abajo pag. secundarias
  
 \hline
 \multicolumn{2}{| c |}{Fin de la tabla.}\\
 \hline\hline
 \endlastfoot %Hasta aqui parte de abajo pag. final
 
 %Contenido
 Muchas lineas & como esta\\
 Muchas lineas & como esta\\
 Muchas lineas & como esta\\
 Muchas lineas & como esta\\
 ...
 Muchas lineas & como esta\\
 \end{longtable}
 \end{verbatim}
 \end{tiny}
 \end{multicols}
\end{frame}

\subsection{Combinar celdas.}
\begin{frame}{Contenido.}
\begin{multicols}{2}
  \tableofcontents[currentsubsection]
\end{multicols}
\end{frame}

\subsubsection*{\itshape multicols.}

\begin{frame}[fragile]{\itshape multicols.}
    La sintaxis es la siguiente:\\~\\
    \verb!\multicolumn{numColumnas}{Diseño}{Contenido}! \pause \\~\\
    \begin{minipage}{0.6\linewidth}
    \begin{small}
\begin{verbatim}
\begin{tabular}{|c|c|c|}\hline
     a & \multicolumn{2}{c}{b}\\ \hline
     c & d & e\\ \hline
     f & g & h\\ \hline
\end{tabular}
\end{verbatim}
    \end{small}
    \end{minipage}\pause
    \begin{minipage}{0.3\linewidth}
    \centering
\begin{tabular}{|c|c|c|}\hline
     a & \multicolumn{2}{c}{b}  \\ \hline
     c & d & e\\ \hline
     f & g & h\\ \hline
\end{tabular}
    \end{minipage}
    
\end{frame}

\subsubsection*{\itshape multirows.}

\begin{frame}[fragile]{\itshape multirows.}
    Para poder escribir combinar filas hay que cargar el paquete \verb!multirow!. La sintaxis es la siguiente:\\~\\ \verb!\multirow[opciones]{numFilas}{Ancho}{Contenido}!\pause\\~\\
    
    
\begin{minipage}{0.6\linewidth}
\begin{verbatim}
\begin{tabular}{ |c|c|c|c| } 
\hline
a & b & c \\
\hline
\multirow{3}{*}{d} & e & f \\ 
& g & h \\ 
& i & j \\ 
\hline
\end{tabular}
\end{verbatim}    
\end{minipage}\pause
\begin{minipage}{0.3\linewidth}
\begin{tabular}{ |c|c|c|c| } 
\hline
a & b & c \\
\hline
\multirow{3}{*}{d} & e & f \\ 
& g & h \\ 
& i & j \\ 
\hline
\end{tabular}
\end{minipage}
\end{frame}

\begin{frame}[fragile]{\itshape multirows.}
\begin{minipage}{0.6\linewidth}
\begin{verbatim}
\begin{tabular}{ |c|c|c|c| } 
\hline
a & b & c \\
\hline
\multirow[t]{3}{*}{d} & e & f \\ 
& g & h \\ 
& i & j \\ 
\hline
\end{tabular}
\end{verbatim}    
\end{minipage}
\begin{minipage}{0.3\linewidth}
\centering
\begin{tabular}{ |c|c|c|c| } 
\hline
a & b & c \\
\hline
\multirow[t]{3}{*}{d} & e & f \\ 
& g & h \\ 
& i & j \\ 
\hline
\end{tabular}
\end{minipage}
\end{frame}

\begin{frame}[fragile]{\itshape multirows.}
\begin{minipage}{0.6\linewidth}
\begin{verbatim}
\begin{tabular}{ |c|c|c|c| } 
\hline
a & b & c \\
\hline
\multirow[b]{3}{*}{d} & e & f \\ 
& g & h \\ 
& i & j \\ 
\hline
\end{tabular}
\end{verbatim}    
\end{minipage}
\begin{minipage}{0.3\linewidth}
\centering
\begin{tabular}{ |c|c|c|c| } 
\hline
a & b & c \\
\hline
\multirow[b]{3}{*}{d} & e & f \\ 
& g & h \\ 
& i & j \\ 
\hline
\end{tabular}
\end{minipage}
\end{frame}

\subsection{Booktabs}
\begin{frame}{Contenido.}
\begin{multicols}{2}
  \tableofcontents[currentsubsection]
\end{multicols}
\end{frame}

\begin{frame}{Reglas para el diseño de tablas\footnote{``Chicago Manual of Style''}.}

    \begin{enumerate}
        \item Evitar lineas verticales.
        \item Evitar encerrar todas las celdas.
        \item Evitar dobles lineas horizontales.
        \item Dejar espacio suficiente entre filas.
        \item Preferiblemente alinear a la izquierda.
    \end{enumerate}
\end{frame}

\begin{frame}[fragile]{Booktabs}
    El paquete \verb!booktabs! permite diseñar mejor las lineas de las tablas. Contiene los siguientes tipos de lineas:
    \begin{enumerate}
        \item \verb!\toprule[ancho]!: linea superior de la tabla.
        \item \verb!\midrule[ancho]!: lineas intermedias de la tabla.
        \item \verb!\cmidrule[ancho](trim){colInicial-colFinal}!: lineas intermedias que no van completas.
        \item \verb!\bottomrule[ancho]!: linea inferior de la tabla.
    \end{enumerate}
\end{frame}

\begin{frame}[fragile]{Booktabs}
\begin{center}
    \begin{tabular}{*4{c}}\toprule
     a & b & c & d  \\
     \cmidrule[2pt](r{1mm}){1-2}
     \cmidrule(l{1mm}){3-4}
     e & f & g & h  \\ \midrule
     i & j & k & l \\\bottomrule[1.5pt]
\end{tabular}
\end{center}

\begin{verbatim}
\begin{tabular}{*4{c}}\toprule
     a & b & c & d  \\
     \cmidrule[2pt](r{1mm}){1-2}
     \cmidrule(l{1mm}){3-4}
     e & f & g & h  \\ \midrule
     i & j & k & l \\\bottomrule[1.5pt]
\end{tabular}
\end{verbatim}

\end{frame}

\begin{frame}[fragile]{Ejercicio. (Actividad 2: nombre.tex y nombre.pdf)}
En un documento \verb!article! desarrolle lo siguiente:
\vspace*{-0.7cm}\begin{enumerate}
    \item Incluya la lista de tablas (etiquete todas las tablas).
    \item En un minipage de \verb!0.6\linewidth! escriba un texto.
    \item En un minipage junto al anterior ponga la siguiente tabla:
    \begin{center}
    \tiny
        \begin{tabular}{*2{c}*3{l}}\toprule
             \multicolumn{5}{c}{Este es el encabezado de la tabla}\\
             \multicolumn{2}{c}{Izquierda} & \multicolumn{3}{c}{Derecha}\\\midrule
             \multirow{3}{*}{Categoria} & 1 & \multirow{3}{*}{Categoria} & \multirow{2}{*}{1} & a \\\cmidrule{5-5}
             & 2 & & & b \\\cmidrule{4-5}
             & 3 & & \multicolumn{2}{c}{2}\\ \bottomrule 
        \end{tabular}
    \end{center}
    \item En un minipage de \verb!5cm! escale la tabla anterior para que se ajuste al minipage.
    \item En un minipage del ancho restante escriba la siguiente tabla: 
    \begin{center}
    \tiny
    \begin{tabular}{|*9{c|}}\hline
    cond. \textbackslash red & (1) & (2) & (3) & (4) & (5) & (6) & (7) & (8)\\ \hline
         Estable por Pares & SI & & & & SI & SI & SI & SI\\ \hline
         Eficiencia & & & & & SI & SI & SI & \\ \hline
         Eficiencia Pareto & & & & & SI & SI & SI & \\ \hline
    \end{tabular}
    \end{center}
    \item Rediseñe la tabla anterior con base en las reglas que mencionamos.
\end{enumerate}
\end{frame}
{
% all template changes are local to this group.
    \setbeamertemplate{navigation symbols}{}
    \setbeamercolor{background canvas}{bg=colorClase}
    \begin{frame}[plain, noframenumbering]
    \vfill
    \begin{center}
    \begin{Huge}
        %\textcolor{white}{Gracias!}
    \end{Huge}
    \end{center}
    \vfill
     \end{frame}
}


\end{document}