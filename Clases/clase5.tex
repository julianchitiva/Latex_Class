
\documentclass[dvipsnames,xcolor, handout]{beamer}
\usepackage[spanish]{babel}
\usepackage[utf8]{inputenc}
\usepackage[all]{xy}
\usepackage{afterpage}
\usepackage{tikz}
\usepackage{cancel}
\usepackage{verbatim}
\usepackage{tabu}
\usepackage{xfrac}
\usepackage{mathrsfs}
\usepackage{amsthm}
\usepackage{amssymb}
\usepackage{bbm}
\usepackage{enumerate}
\usepackage{booktabs}
\usepackage{relsize}
\usepackage{hyperref}
\usepackage{float}
\usepackage{longtable}
\usepackage{amsmath}
\usepackage{multirow}
\usepackage{multicol}
\usepackage{colortbl}
\usepackage{adjustbox}
\usepackage{xfrac}
\usepackage{bm}
\usepackage{keystroke}
\usepackage{wrapfig}

\newcommand{\hlc}[2][yellow]{ {\sethlcolor{#1} \hl{#2}} }

\newcommand*{\rom}[1]{\expandafter\romannumeral #1}
\newcommand{\Rom}[1]{\uppercase\expandafter{\romannumeral #1\relax}}

\newcommand{\Importante}[2]{{\color{#1}#2}}
\newcommand{\importante}[2]{{\color{#1}\underline{#2}}}

\renewcommand{\baselinestretch}{1}
\setlength{\parskip}{\baselineskip}


\def\Put(#1,#2)#3{\leavevmode\makebox(0,0){\put(#1,#2){#3}}}

 \usetheme{Boadilla}

%\usecolortheme{crane}
\definecolor{colorClase}{rgb}{0,0.55,0.082}
\usecolortheme[named=colorClase]{structure}
\usepackage{natbib}


\theoremstyle{plain}
  \newtheorem{teorema}{Teorema}
  \newtheorem{proposicion}{Proposición}
  \newtheorem{corolario}{Corolario}
  \newtheorem{lema}[teorema]{Lema}
\theoremstyle{definition}
  \newtheorem{definicion}{Definici\'on}
  \newtheorem{ejemplo}{Ejemplo}
  
  
\makeatletter
\setbeamertemplate{footline}
{
  \leavevmode%
  \hbox{%
  \begin{beamercolorbox}[wd=.4\paperwidth,ht=2.25ex,dp=1ex,center]{author in head/foot}%
    \usebeamerfont{author in head/foot}\insertshortauthor \hspace*{1em}(\insertshortinstitute)
  \end{beamercolorbox}%
  \begin{beamercolorbox}[wd=.5\paperwidth,ht=2.25ex,dp=1ex,center]{title in head/foot}%
    \usebeamerfont{title in head/foot}\insertsection 
  \end{beamercolorbox}%
  \begin{beamercolorbox}[wd=.1\paperwidth,ht=2.25ex,dp=1ex,center]{date in head/foot}%
    \usebeamerfont{date in head/foot}
    \insertframenumber{} / \inserttotalframenumber\hspace*{2ex} 
  \end{beamercolorbox}}%
  \vskip0pt%
}
\makeatother
\setbeamertemplate{caption}[numbered]
 
\title{Taller usos de \LaTeX \\ \small{Diseño de páginas} \vspace*{-0.2cm}}

\setbeamersize{text margin left=25pt,text margin right=25pt}

\author[Julián Chitiva Bocanegra]{Julián Enrique Chitiva Bocanegra}
\institute[Uniandes] 
{Universidad de los Andes\\ Facultad de Economía}
\titlegraphic{\includegraphics[width=0.8cm]{img/uniandes_logo.png}
}

\date{\today}

\subject{}
\usepackage[figurename=]{caption}
\begin{document}
\begin{frame}
  \titlepage
\end{frame}

\begin{frame}{Contenido.}
\begin{footnotesize}
\vspace*{-1cm}
\begin{multicols}{2}
  \tableofcontents
\end{multicols}
\end{footnotesize}
\end{frame}

\section{Múltiples columnas.}
\begin{frame}{Contenido.}
\begin{footnotesize}
\vspace*{-1cm}
\begin{multicols}{2}
  \tableofcontents[currentsection]
\end{multicols}
\end{footnotesize}
\end{frame}

\subsection{Columnas balanceadas.}
\begin{frame}{Contenido.}
 \begin{footnotesize}
\vspace*{-1cm}
\begin{multicols}{2}
  \tableofcontents[currentsubsection]
\end{multicols}
\end{footnotesize}
\end{frame}

\begin{frame}[fragile]{Columnas balanceadas.}
Para poder tener múltiples columnas balanceadas\footnote{Ambas columnas tienen la misma cantidad de lineas.} en un documento es necesario usar el paquete \verb!multicol!. La forma de usarlo es muy sencillo:
\begin{verbatim}
    \begin{multicols}{numColumnas}
    Contenido a partir en numColumnas columnas.
    \end{multicols}
\end{verbatim}
\end{frame}

\begin{frame}[fragile]{Columnas balanceadas: ejemplo.}
\begin{footnotesize}
\begin{verbatim}
\begin{multicols}{2}
Contenido a partir en dos columnas.  Este es un ejemplo de texto para
partir en dos columnas. Este es un ejemplo de texto para partir en dos
columnas. Este es un ejemplo de texto para partir en dos columnas. 
\end{multicols}
\end{verbatim}
\end{footnotesize}
\vfill\pause
\begin{footnotesize}
    \begin{multicols}{2}
    Contenido a partir en dos columnas.  Este es un ejemplo de texto para
partir en dos columnas. Este es un ejemplo de texto para partir en dos
columnas. Este es un ejemplo de texto para partir en dos columnas. 
    \end{multicols}
\end{footnotesize}

\end{frame}

\begin{frame}[fragile]{Columnas balanceadas: ejemplo.}
\begin{footnotesize}
\begin{verbatim}
\begin{multicols}{3}
Contenido a partir en tres columnas. Este es un ejemplo de texto para
partir en tres columnas. Este es un ejemplo de texto para partir en tres
columnas. Este es un ejemplo de texto para partir en tres columnas. 
\end{multicols}
\end{verbatim}
\end{footnotesize}
\vfill\pause
\begin{footnotesize}
    \begin{multicols}{3}
    Contenido a partir en tres columnas. Este es un ejemplo de
    texto para partir en tres columnas. Este es un ejemplo de
    texto para partir en tres columnas. Este es un ejemplo de
    texto para partir en tres columnas. 
    \end{multicols}
\end{footnotesize}
\end{frame}

\begin{frame}[fragile]{Columnas balanceadas: ejemplo.}
\begin{footnotesize}
\begin{verbatim}
\begin{multicols}{4}
Contenido a partir en cuatro columnas. Este es un ejemplo de
texto para partir en cuatro columnas. Este es un ejemplo de 
texto para partir en cuatro columnas. Este es un ejemplo de
texto para partir en cuatro columnas. 
\end{multicols}
\end{verbatim}
\end{footnotesize}
\vfill\pause
\begin{footnotesize}
    \begin{multicols}{4}
    Contenido a partir en cuatro columnas.  Este es un ejemplo de texto para
partir en cuatro columnas. Este es un ejemplo de texto para partir en cuatro
columnas. Este es un ejemplo de texto para partir en cuatro columnas. 
    \end{multicols}
\end{footnotesize}
\end{frame}

\subsection{Separación de las columnas.}
\begin{frame}{Contenido.}
 \begin{footnotesize}
\vspace*{-1cm}
\begin{multicols}{2}
  \tableofcontents[currentsubsection]
\end{multicols}
\end{footnotesize}
\end{frame}

\begin{frame}[fragile]{Separación de las columnas.}
Para poder cambiar el espacio de separación de las columnas predeterminado, en el preámbulo hay que poner:

\verb!\setlength{\columnsep}{3cm}!

\end{frame}

{
\setlength{\columnsep}{3cm}
\begin{frame}[fragile]{Separación de las columnas: ejemplo.}
\begin{footnotesize}
\begin{verbatim}
\begin{multicols}{2}
Contenido a partir en dos columnas.  Este es un ejemplo de texto para
partir en dos columnas. Este es un ejemplo de texto para partir en dos
columnas. Este es un ejemplo de texto para partir en dos columnas. 
\end{multicols}
\end{verbatim}
\end{footnotesize}
\vfill\pause
\begin{footnotesize}
    \begin{multicols}{2}
    Contenido a partir en dos columnas.  Este es un ejemplo de texto para
partir en dos columnas. Este es un ejemplo de texto para partir en dos
columnas. Este es un ejemplo de texto para partir en dos columnas. 
    \end{multicols}
\end{footnotesize}
\end{frame}
}

\subsection{Linea de separación entre columnas.}
\begin{frame}{Contenido.}
 \begin{footnotesize}
\vspace*{-1cm}
\begin{multicols}{2}
  \tableofcontents[currentsubsection]
\end{multicols}
\end{footnotesize}
\end{frame}

\begin{frame}[fragile]{Línea de separación entre columnas.}
Para poder fijar una línea de separación entre las columnas de manera predeterminado, en el preámbulo hay que poner:

\verb!\setlength{\columnseprule}{1mm}!

\end{frame}

{
\setlength{\columnseprule}{1mm}
\begin{frame}[fragile]{Línea de separación entre columnas: ejemplo.}
\begin{footnotesize}
\begin{verbatim}
\begin{multicols}{2}
Contenido a partir en dos columnas.  Este es un ejemplo de texto para
partir en dos columnas. Este es un ejemplo de texto para partir en dos
columnas. Este es un ejemplo de texto para partir en dos columnas. 
\end{multicols}
\end{verbatim}
\end{footnotesize}
\vfill\pause
\begin{footnotesize}
    \begin{multicols}{2}
    Contenido a partir en dos columnas.  Este es un ejemplo de texto para
partir en dos columnas. Este es un ejemplo de texto para partir en dos
columnas. Este es un ejemplo de texto para partir en dos columnas. 
    \end{multicols}
\end{footnotesize}
\end{frame}
}

\subsection{Encabezado de las columnas.}

\begin{frame}{Contenido.}
 \begin{footnotesize}
\vspace*{-1cm}
\begin{multicols}{2}
  \tableofcontents[currentsubsection]
\end{multicols}
\end{footnotesize}
\end{frame}

\begin{frame}[fragile]{Encabezado de las columnas.}
Para poder poner un encabezado a las columnas (saltarse la orden de multiples columnas) hay que poner el texto que queremos en una sola columna entre corchetes \verb![  ]!.\pause 
\begin{footnotesize}
\begin{verbatim}
\begin{multicols}{2}
[
Este es un encabezado para las columnas de este multicols
]
Contenido a partir en dos columnas.  Este es un ejemplo de texto para
partir en dos columnas. Este es un ejemplo de texto para partir en dos
columnas. Este es un ejemplo de texto para partir en dos columnas. 
\end{multicols}
\end{verbatim}
\end{footnotesize}
\pause
\begin{footnotesize}
\begin{multicols}{2}
[
Este es un encabezado para las columnas de este multicols
]
Contenido a partir en dos columnas.  Este es un ejemplo de texto para
partir en dos columnas. Este es un ejemplo de texto para partir en dos
columnas. Este es un ejemplo de texto para partir en dos columnas. 
    \end{multicols}
\end{footnotesize}

\end{frame}

\subsection{Columnas desbalanceadas.}
\begin{frame}{Contenido.}
 \begin{footnotesize}
\vspace*{-1cm}
\begin{multicols}{2}
  \tableofcontents[currentsubsection]
\end{multicols}
\end{footnotesize}
\end{frame}

\begin{frame}[fragile]{Columnas desbalanceadas.}
Para poder tener múltiples columnas desbalanceadas\footnote{Las columnas pueden tener diferente cantidad de lineas.} la sintaxis es la siguiente:
\begin{verbatim}
    \begin{multicols*}{numColumnas}
    Contenido a partir en numColumnas columnas.
    \end{multicols*}
\end{verbatim}
En este caso se puede forzar el salto de columna mediante 

\verb!\columnbreak!

Las demás características de las columnas balanceadas aplican para este caso.
\end{frame}

\begin{frame}[fragile]{Columnas desbalanceadas: ejemplo.}
\begin{footnotesize}
\begin{verbatim}
\begin{multicols*}{2}
Contenido a partir en dos columnas.  Este es un ejemplo de texto para
partir en dos columnas. Este es un ejemplo de texto 
\columnbreak 
para partir en dos columnas. Este es un ejemplo de 
texto para partir en dos columnas. 
\end{multicols*}
\end{verbatim}
\end{footnotesize}
\pause
\vspace*{0.6cm}
\begin{footnotesize}
    \begin{multicols*}{2}
    Contenido a partir en dos columnas.  Este es un ejemplo de texto para
partir en dos columnas. Este es un ejemplo de texto 
\columnbreak 
para partir en dos columnas. Este es un ejemplo de 
texto para partir en dos columnas. 
    \end{multicols*}
\end{footnotesize}

\end{frame}

\subsection{Tablas dentro de las columnas.}
\begin{frame}[fragile]{Tablas dentro de las columnas.}
Para poder insertar tablas dentro de las columnas, en el encabezado hay que poner \verb!\usepackage{wrapfig}!.

\begin{scriptsize}
\begin{verbatim}
\begin{multicols}{2}
Contenido a partir en dos columnas. 
  
\begin{wraptable}{l}{0.5\linewidth}
\centering
\begin{tabular}{|c|c|}
\hline
A & B \\
\hline
C & D \\
\hline
\end{tabular}
\end{wraptable}

Este es un ejemplo de texto para partir en dos columnas.Este es un ejemplo de
texto para partir en dos columnas.Este es un ejemplo de texto para partir en
dos columnas. Este es un ejemplo de texto para partir en dos columnas. Este es
un ejemplo de texto para partir en dos columnas.
\end{multicols}
\end{verbatim}
\end{scriptsize}

\end{frame}

\begin{frame}[fragile]{Tablas dentro de las columnas.}

\begin{multicols}{2}
Contenido a partir en dos columnas. 
  
\begin{wraptable}{l}{0.5\linewidth}
\centering
\begin{tabular}{|c|c|}
\hline
A & B \\
\hline
C & D \\
\hline
\end{tabular}
\end{wraptable}

Este es un ejemplo de texto para
partir en dos columnas.Este es un ejemplo de texto para
partir en dos columnas.Este es un ejemplo de texto para
partir en dos columnas.Este es un ejemplo de texto para
partir en dos columnas.
Este es un ejemplo de texto para
partir en dos columnas.
\end{multicols}

\end{frame}

\section{Espaciado horizontal.}
\begin{frame}{Contenido.}
 \begin{footnotesize}
\vspace*{-1cm}
\begin{multicols}{2}
  \tableofcontents[currentsection]
\end{multicols}
\end{footnotesize}
\end{frame}

\subsection{hspace}
\begin{frame}{Contenido.}
 \begin{footnotesize}
\vspace*{-1cm}
\begin{multicols}{2}
  \tableofcontents[currentsubsection]
\end{multicols}
\end{footnotesize}
\end{frame}

\begin{frame}[fragile]{\itshape hspace}
    Cuando queremos insertar espacios horizontales usamos la función \verb!\hspace{distancia}!
\begin{verbatim}
Este \hspace{2cm} es un ejemplo de hspace.
\end{verbatim}
Este \hspace{2cm} es un ejemplo de hspace.


A veces toca usar \verb!\hspace*{distancia}! para insertar obligatoriamente el espacio.

\end{frame}


\subsection{hfill}
\begin{frame}{Contenido.}
 \begin{footnotesize}
\vspace*{-1cm}
\begin{multicols}{2}
  \tableofcontents[currentsubsection]
\end{multicols}
\end{footnotesize}
\end{frame}

\begin{frame}[fragile]{\itshape hfill}
La función \verb!\hfill! sirve para llenar el espacio horizontal
    
\begin{verbatim}
Este es un \hfill ejemplo de hfill.    
\end{verbatim}

Este es un \hfill ejemplo de hfill.    \pause

\begin{verbatim}
Este es otro \hfill ejemplo de \hfill hfill.    
\end{verbatim}

Este es otro \hfill ejemplo de \hfill hfill.    

\end{frame}

\section{Espaciado vertical.}
\begin{frame}{Contenido.}
  \begin{footnotesize}
\vspace*{-1cm}
\begin{multicols}{2}
  \tableofcontents[currentsection]
\end{multicols}
\end{footnotesize}
\end{frame}

\subsection{vspace}
\begin{frame}{Contenido.}
 \begin{footnotesize}
\vspace*{-1cm}
\begin{multicols}{2}
  \tableofcontents[currentsubsection]
\end{multicols}
\end{footnotesize}
\end{frame}

\begin{frame}[fragile]{\itshape vspace}
    Cuando queremos insertar espacios horizontales usamos la función \verb!\vspace{distancia}!
\begin{multicols*}{2}
\begin{verbatim}
Este 

\vspace{0.5cm}

es un ejemplo de vspace.
\end{verbatim}
\columnbreak
Este 

\vspace{0.5cm} 

es un ejemplo de vspace.
\end{multicols*}


A veces toca usar \verb!\vspace*{distancia}! para insertar obligatoriamente el espacio.

\end{frame}

\subsection{vfill}
\begin{frame}{Contenido.}
 \begin{footnotesize}
\vspace*{-1cm}
\begin{multicols}{2}
  \tableofcontents[currentsubsection]
\end{multicols}
\end{footnotesize}
\end{frame}

\begin{frame}[fragile]{\itshape vfill}
La función \verb!\vfill! sirve para llenar el espacio horizontal
    

\begin{verbatim}
Este es un \vfill ejemplo de vfill.    
\end{verbatim}

Este es un \vfill  ejemplo de vfill.        

\end{frame}

\section{Numeración de páginas.}
\begin{frame}{Contenido.}
  \begin{footnotesize}
\vspace*{-1cm}
\begin{multicols}{2}
  \tableofcontents[currentsection]
\end{multicols}
\end{footnotesize}
\end{frame}

\begin{frame}[fragile]{Numeración de páginas.}
Para cambiar la numeración de páginas hay que usar el comando \verb!\pagenumbering{num_style}! en el preámbulo\footnote{El cambio de estilo en la numeración se puede hacer en cualquier parte del documento.}.\pause

Los estilos de numeración disponibles en \LaTeX\ son:
\begin{itemize}
    \item \verb!arabic!: números arabicos.
    \item \verb!roman!: números romanos en minúscula.
    \item \verb!Roman!: números romanos en mayúscula.
    \item \verb!alph!: letras del alfabeto en minúscula.
    \item \verb!Alph!: letras del alfabeto en mayúscula.
\end{itemize}\pause

Recuerden que también pueden usar los contadores. \verb!\setcounter{page}{3}!

\end{frame}

\section{Salto de página.}
\begin{frame}{Contenido.}
  \begin{footnotesize}
\vspace*{-1cm}
\begin{multicols}{2}
  \tableofcontents[currentsection]
\end{multicols}
\end{footnotesize}
\end{frame}

\begin{frame}[fragile]{Salto de página.}
    Hay 3 formas de insertar saltos de línea:
    \begin{itemize}
        \item \verb!\clearpage!: inserta una nueva página y si tiene tablas o imagenes sin poner las pone antes de insertar una nueva página.
        \item \verb!\newpage!: inserta una nueva página dejando como espacio en blanco el espacio restante de la página.
        \item \verb!pagebreak!: inserta una nueva página y distribuye el espacio en blanco restante entre los elementos de la página. 
    \end{itemize}
\end{frame}

\section{Páginas horizontales.}
\begin{frame}{Contenido.}
  \begin{footnotesize}
\vspace*{-1cm}
\begin{multicols}{2}
  \tableofcontents[currentsection]
\end{multicols}
\end{footnotesize}
\end{frame}

\begin{frame}[fragile]{Páginas horizontales.}
    Para poder insertar páginas horizontales es necesario:
    \begin{enumerate}
        \item En el preámbulo: \verb!\usepackage{pdflscape}!
        \item La parte del documento que se quiere en horizontal:
\begin{verbatim}
    \begin{landscape}
    Contenido de la página horizontal.
    \end{landscape}
\end{verbatim}
    \end{enumerate}
\end{frame}
 
\section{Actividad} 
\begin{frame}{Contenido.}
  \begin{footnotesize}
\vspace*{-1cm}
\begin{multicols}{2}
  \tableofcontents[currentsection]
\end{multicols}
\end{footnotesize}
\end{frame}
\begin{frame}[fragile]{Ejercicio. (Actividad 4: nombre.tex y nombre.pdf)}
En un documento \verb!article! desarrolle lo siguiente:

\begin{enumerate}
    \item En la primera página: inserte cuatro tablas (las que quiera) una en cada esquina (pista: use \verb!\hfill! y \verb!\vfill!)
    \item En la segunda página: Inserte tres columnas de texto (El que quiera) balanceadas, con lineas de separación de 2pt de grosor y espacio de separación de 4cm. 
    \item En la tercera página: cambie la numeración a números romanos y que esa sea la página 50.
\end{enumerate}

\end{frame}
{
% all template changes are local to this group.
    \setbeamertemplate{navigation symbols}{}
    \setbeamercolor{background canvas}{bg=colorClase}
    \begin{frame}[plain, noframenumbering]
    \vfill
    \begin{center}
    \begin{Huge}
        %\textcolor{white}{Gracias!}
    \end{Huge}
    \end{center}
    \vfill
     \end{frame}
}
            
\end{document}