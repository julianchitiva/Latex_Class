\documentclass[dvipsnames,xcolor, handout]{beamer}
\usepackage[spanish]{babel}
\usepackage[utf8]{inputenc}
\usepackage[all]{xy}
\usepackage{afterpage}
\usepackage{tikz}
\usepackage{cancel}
\usepackage{verbatim}
\usepackage{tabu}
\usepackage{xfrac}
\usepackage{mathrsfs}
\usepackage{amsthm}
\usepackage{amssymb}
\usepackage{bbm}
\usepackage{enumerate}
\usepackage{booktabs}
\usepackage{relsize}
\usepackage{hyperref}
\usepackage{float}
\usepackage{longtable}
\usepackage{amsmath}
\usepackage{multirow}
\usepackage{multicol}
\usepackage{colortbl}
\usepackage{adjustbox}
\usepackage{xfrac}
\usepackage{bm}
\usepackage{keystroke}

\newcommand{\hlc}[2][yellow]{ {\sethlcolor{#1} \hl{#2}} }

\newcommand*{\rom}[1]{\expandafter\romannumeral #1}
\newcommand{\Rom}[1]{\uppercase\expandafter{\romannumeral #1\relax}}

\newcommand{\Importante}[2]{{\color{#1}#2}}
\newcommand{\importante}[2]{{\color{#1}\underline{#2}}}

\renewcommand{\baselinestretch}{1}
\setlength{\parskip}{\baselineskip}


\def\Put(#1,#2)#3{\leavevmode\makebox(0,0){\put(#1,#2){#3}}}

 \usetheme{Boadilla}

%\usecolortheme{crane}
\definecolor{colorClase}{rgb}{0.682,0.055,0.0208}
\usecolortheme[named=colorClase]{structure}
\usepackage{natbib}


\theoremstyle{plain}
  \newtheorem{teorema}{Teorema}
  \newtheorem{proposicion}{Proposición}
  \newtheorem{corolario}{Corolario}
  \newtheorem{lema}[teorema]{Lema}
\theoremstyle{definition}
  \newtheorem{definicion}{Definici\'on}
  \newtheorem{ejemplo}{Ejemplo}
  
  
\makeatletter
\setbeamertemplate{footline}
{
  \leavevmode%
  \hbox{%
  \begin{beamercolorbox}[wd=.4\paperwidth,ht=2.25ex,dp=1ex,center]{author in head/foot}%
    \usebeamerfont{author in head/foot}\insertshortauthor \hspace*{1em}(\insertshortinstitute)
  \end{beamercolorbox}%
  \begin{beamercolorbox}[wd=.5\paperwidth,ht=2.25ex,dp=1ex,center]{title in head/foot}%
    \usebeamerfont{title in head/foot}\insertsection 
  \end{beamercolorbox}%
  \begin{beamercolorbox}[wd=.1\paperwidth,ht=2.25ex,dp=1ex,center]{date in head/foot}%
    \usebeamerfont{date in head/foot}
    \insertframenumber{} / \inserttotalframenumber\hspace*{2ex} 
  \end{beamercolorbox}}%
  \vskip0pt%
}
\makeatother
\setbeamertemplate{caption}[numbered]
 
\title{Taller usos de \LaTeX \\ \small{Listas} \vspace*{-0.2cm}}

\setbeamersize{text margin left=25pt,text margin right=25pt}

\author[Julián Chitiva Bocanegra]{Julián Enrique Chitiva Bocanegra}
\institute[Uniandes] 
{Universidad de los Andes\\ Facultad de Economía}
\titlegraphic{\includegraphics[width=0.8cm]{img/uniandes_logo.png}
}

\date{\today}

\subject{}
\usepackage[figurename=]{caption}
\begin{document}
\begin{frame}
  \titlepage
\end{frame}

\begin{frame}{Contenido.}
  \tableofcontents
\end{frame}

\section{Listas no ordenadas.}
\begin{frame}{Contenido.}
  \tableofcontents[currentsection]
\end{frame}

\begin{frame}[fragile]{Listas no ordenadas.}
    En \LaTeX\ las listas son muy (muy) fáciles de crear. Para crear listas no ordenadas usamos el ambiente \verb!itemize!.
    
    \begin{minipage}{0.5\linewidth}
\begin{verbatim}
\begin{itemize}
    \item Un elemento.
    \item Otro elemento.
\end{itemize}    
\end{verbatim}
    \end{minipage}\pause
    \begin{minipage}{0.5\linewidth}
    \begin{itemize}
    \item Un elemento.
    \item Otro elemento.
\end{itemize}    
    \end{minipage}
\end{frame}

\begin{frame}[fragile]{Características de listas no ordenadas.}
    \begin{itemize}
        \item Cada entrada de una lista no ordenada es indicada (por \textit{default}) con un punto negro ``$\bullet$'', llamado bullet.
        \item El texto en cada una de las entradas puede ser de cualquier longitud. 
    \end{itemize}
\end{frame}

\section{Listas ordenadas.}
\begin{frame}{Contenido.}
  \tableofcontents[currentsection]
\end{frame}

\begin{frame}[fragile]{Listas ordenadas.}
La forma de escribir estas listas es igual a las no ordenadas, sin embargo, se debe usar el ambiente \verb!enumerate! \\~\\ \pause
\begin{minipage}{0.5\linewidth}
\begin{verbatim}
\begin{enumerate}
    \item Un elemento.
    \item Otro elemento.
\end{enumerate}    
\end{verbatim}
    \end{minipage}\pause
    \begin{minipage}{0.5\linewidth}
    \begin{enumerate}
    \item Un elemento.
    \item Otro elemento.
\end{enumerate}    
    \end{minipage}

\end{frame}

\begin{frame}[fragile]{Características de listas ordenadas.}
    \begin{enumerate}
        \item Las etiquetas consisten de números que siguen un orden secuencial.
        \item Los números empiezan con 1. cada vez que se llama el ambiente \verb!enumerate!.
    \end{enumerate}
\end{frame}

\section{Listas anidadas.}
\begin{frame}{Contenido.}
  \tableofcontents[currentsection]
\end{frame}

\begin{frame}[fragile]{Listas anidadas.}
Una lista anidada es una lista que contiene sublistas. La forma de escribir estas listas es igual, dependiendo si queremos anidar listas ordenadas, no ordenadas o mezcladas. 
\end{frame}

\begin{frame}[fragile]{Listas anidadas: \itshape itemize.}
\begin{minipage}{0.55\linewidth}
\begin{verbatim}
\begin{itemize}
    \item Nivel 1.
    \begin{itemize}
        \item Nivel 2.
        \begin{itemize}
            \item Nivel 3.
        \end{itemize}
    \end{itemize}
\end{itemize}    
\end{verbatim}
    \end{minipage}\pause
    \begin{minipage}{0.45\linewidth}
\begin{itemize}
\item Nivel 1.
\begin{itemize}
    \item Nivel 2.
    \begin{itemize}
        \item Nivel 3.
    \end{itemize}
\end{itemize}
\end{itemize}    
    \end{minipage}

\end{frame}

\begin{frame}[fragile]{Listas anidadas: \itshape enumerate.}
\begin{minipage}{0.55\linewidth}
\begin{verbatim}
\begin{enumerate}
    \item Nivel 1.
    \begin{enumerate}
        \item Nivel 2.
        \begin{enumerate}
            \item Nivel 3.
        \end{enumerate}
    \end{enumerate}
\end{enumerate}    
\end{verbatim}
    \end{minipage}\pause
    \begin{minipage}{0.45\linewidth}
\begin{enumerate}
    \item Nivel 1.
    \begin{enumerate}
        \item Nivel 2.
        \begin{enumerate}
            \item Nivel 3.
        \end{enumerate}
    \end{enumerate}
\end{enumerate}   
\end{minipage}
\end{frame}

\begin{frame}[fragile]{Listas anidadas: mixta.}
\begin{minipage}{0.55\linewidth}
\begin{verbatim}
\begin{enumerate}
    \item Nivel 1.
    \begin{itemize}
        \item Nivel 2.
        \begin{enumerate}
            \item Nivel 3.
        \end{enumerate}
    \end{itemize}
\end{enumerate}    
\end{verbatim}
    \end{minipage}\pause
    \begin{minipage}{0.45\linewidth}
\begin{enumerate}
    \item Nivel 1.
    \begin{itemize}
        \item Nivel 2.
        \begin{enumerate}
            \item Nivel 3.
        \end{enumerate}
    \end{itemize}
\end{enumerate}    
\end{minipage}
\end{frame}

\begin{frame}[fragile]{Características de las listas anidadas.}
En \verb!documentclass! diferentes a \verb!beamer!:

\begin{itemize}
    \item Las listas anidadas permiten hasta 4 niveles de profundidad, en \verb!beamer! solo permite 3. 
    \item Cuando se están anidando listas ordenadas (\verb!enumerate!), las etiquetas de cada nivel son distintas automáticamente.
\end{itemize}
\end{frame}

\section{Diseño de listas.}
\begin{frame}{Contenido.}
  \tableofcontents[currentsection]
\end{frame}
\subsection{Cambio de etiquetas.}
\begin{frame}{Contenido.}
  \tableofcontents[currentsubsection]
\end{frame}
\begin{frame}[fragile]{Cambio de etiquetas.}
Hay tres formas de cambiar las etiquetas de las listas ordenadas y no ordenadas. 
\begin{enumerate}
    \item Cambiando las etiquetas dependiendo del nivel de profundidad.
    \item Definiendo las etiquetas para cada entrada manualmente.
    \item Creando etiquetas personalizadas para cada \verb!itemize! o \verb!enumerate!. 
\end{enumerate}
\end{frame}

\begin{frame}[fragile]{Etiquetas dependiendo del nivel de profundidad: \itshape itemize.}
En el preámbulo escribimos:
\begin{verbatim}
\renewcommand{\labelitemi}{$\blacksquare$}
\renewcommand{\labelitemii}{$\square$}
\renewcommand{\labelitemiii}{\textquestiondown}    
\renewcommand{\labelitemiv}{\#}    
\end{verbatim}

\begin{itemize}
    \item[$\blacksquare$] Nivel 1.
    \begin{itemize}
        \item[$\square$] Nivel 2.
        \begin{itemize}
            \item[\textquestiondown] Nivel 3.
        \end{itemize}
    \end{itemize}
\end{itemize}   
\end{frame}

\begin{frame}[fragile]{Etiquetas dependiendo del nivel de profundidad: \itshape enumerate.}
En el preámbulo escribimos:
\begin{verbatim}
\usepackage{enumerate}
\renewcommand{\labelenumi}{\Roman{enumi}.}
\renewcommand{\labelenumii}{\alph{enumii}.}
\renewcommand{\labelenumiii}{\roman{enumiii}.}    
\renewcommand{\labelenumiv}{\Alph{enumiv}.}
\end{verbatim}

\begin{enumerate}
    \item[\Rom{1}.] Nivel 1.
    \begin{enumerate}
        \item[a.] Nivel 2.
        \begin{enumerate}
            \item[\rom{1}.] Nivel 3.
        \end{enumerate}
    \end{enumerate}
\end{enumerate}
\end{frame}

\begin{frame}[fragile]{Definiendo etiquetas manualmente.}
\begin{verbatim}
\begin{itemize}
    \item[$\cdots$] Nivel 1.
    \begin{enumerate}
        \item[$\sim$] Nivel 2.
        \begin{itemize}
            \item[$\alpha$] Nivel 3.
        \end{itemize}
    \end{enumerate}
\end{itemize}   
\end{verbatim}

\begin{itemize}
    \item[$\cdots$] Nivel 1.
    \begin{enumerate}
        \item[$\sim$] Nivel 2.
        \begin{itemize}
            \item[$\alpha$] Nivel 3.
        \end{itemize}
    \end{enumerate}
\end{itemize}   
\end{frame}

\begin{frame}[fragile]{Etiquetas personalizadas}
Para poder personalizar etiquetas tenemos que importar el paquete \verb!enumitem!.
\begin{verbatim}
\begin{enumerate}[start=1,
    label={\large\bfseries Módulo\ \arabic*:}]
\item Nivel 1
\item Nivel 1.1
\end{enumerate}
\end{verbatim}
\begin{enumerate}
\item[\large\bfseries Módulo 1:] Nivel 1
\item[\large\bfseries Módulo 2:] Nivel 1.1
\end{enumerate}

Las formas de personalizar una lista son muchisimas, recomiendo este \href{http://mirrors.ucr.ac.cr/CTAN/macros/latex/contrib/enumitem/enumitem.pdf}{\textcolor{colorClase}{documento}} si quieren profundizar.

\end{frame}
    
\subsection{Identación.}
\begin{frame}{Contenido.}
  \tableofcontents[currentsubsection]
\end{frame}
\begin{frame}[fragile]{Etiquetas personalizadas}
Para poder personalizar etiquetas tenemos que importar el paquete \verb!enumitem!.
\begin{verbatim}
\begin{enumerate}[start=1,
label={\large\bfseries Módulo\ \arabic*:},
leftmargin=2cm]
\item Nivel 1
\item Nivel 1.1
\end{enumerate}
\end{verbatim}
\begin{enumerate}[leftmargin=2cm]
\item[\large\bfseries Módulo 1:] Nivel 1
\item[\large\bfseries Módulo 2:] Nivel 1.1
\end{enumerate}
\end{frame}

\subsection{Contadores.}
\begin{frame}{Contenido.}
  \tableofcontents[currentsubsection]
\end{frame}
\begin{frame}[fragile]{Contadores}
\begin{verbatim}
\begin{enumerate}
\setcounter{enumi}{3}
    \item Este es el primer elemento de la lista.
\end{enumerate}    
\end{verbatim}
    \begin{enumerate}
    \setcounter{enumi}{3}
        \item Este es el primer elemento de la lista.
    \end{enumerate}
Los contadores se pueden tomar valores entre $-2^{31}$ y $2^{31}-1$
\end{frame}

\begin{frame}[fragile]{Contadores.}
Tambien podemos crear nuestros propios contadores. La sintaxis es:
\begin{small}
\begin{verbatim}
\newcounter{contadorPrueba}
\newcounter{contadorPrueba}[otroContador]
\end{verbatim}
\end{small}
Por \textit{default} arranca en 0. Para definir que arranque en otro valor usamos el comando de antes:
\begin{small}
\begin{verbatim}
\setcounter{contadorPrueba}{valor}
\end{verbatim}
\end{small}

\end{frame}

\begin{frame}[fragile]{Contadores.}

Cada vez que lo usemos tenemos que incrementar manualmente mediante:
\begin{small}
\begin{verbatim}
\stepcounter{contadorPrueba}
\end{verbatim}
\end{small}

Podemos obtener el valor del contador mediante 
\begin{small}
\begin{verbatim}
\value{contadorPrueba}
\end{verbatim}
\end{small}

Para profundizar en contadores sugiero revisar las páginas web \href{https://en.wikibooks.org/wiki/LaTeX/Counters}{\textcolor{colorClase}{1}} y \href{https://www.overleaf.com/learn/latex/Counters}{\textcolor{colorClase}{2}}.
\end{frame}

\begin{frame}[fragile]{Uso de contadores}

La forma más fácil de usar los contadores es mediante el comando
\begin{small}
\begin{verbatim}
    \renewcommand{\labelenumi}{contadorPrueba}
\end{verbatim}
\end{small}

O usando el valor de \verb!contadorPrueba! en las demás funciones de \verb!enumerate!
    
\end{frame}



\section{Ejercicio}
\begin{frame}{Contenido.}
  \tableofcontents[currentsection]
\end{frame}
\begin{frame}[fragile]{Ejercicio. (Actividad 3: nombre.tex y nombre.pdf)}
En un documento \verb!article! desarrolle lo siguiente:

\begin{enumerate}
    \item Cree una lista ordenada con una etiqueta de la forma ``\textbf{Tarea} \Rom{1}.'', ``\textbf{Tarea} \Rom{2}.'', $\dots$
    
    \item Cree una lista ordenada con una etiqueta de la forma ``\textbf{\textit{Tarea}} \rom{1}.'', ``\textbf{\textit{Tarea}} \rom{2}.'', $\dots$
    
    \item Cree una lista ordenada con una etiqueta de la forma ``\textit{\textbf{Tarea}} \textbf{E}.'', ``\textbf{\textit{Tarea}} \textbf{F}.'',$\dots$
    
    \item Cree un contador que arranque en 10
    \item Cree una lista ordenada, con etiquetas numéricas, que arranque en este contador e increméntelo con cada item.
    \item Cree otra lista ordenada, con etiquetas en letras minúsculas, que arranque en este contador e increméntelo con cada item.
\end{enumerate}

\end{frame}
{
% all template changes are local to this group.
    \setbeamertemplate{navigation symbols}{}
    \setbeamercolor{background canvas}{bg=colorClase}
    \begin{frame}[plain, noframenumbering]
    \vfill
    \begin{center}
    \begin{Huge}
        %\textcolor{white}{Gracias!}
    \end{Huge}
    \end{center}
    \vfill
     \end{frame}
}


\end{document}