% Options for packages loaded elsewhere
\PassOptionsToPackage{unicode}{hyperref}
\PassOptionsToPackage{hyphens}{url}
%
\documentclass[
]{article}
\usepackage{lmodern}
\usepackage{amssymb,amsmath}
\usepackage{ifxetex,ifluatex}
\ifnum 0\ifxetex 1\fi\ifluatex 1\fi=0 % if pdftex
  \usepackage[T1]{fontenc}
  \usepackage[utf8]{inputenc}
  \usepackage{textcomp} % provide euro and other symbols
\else % if luatex or xetex
  \usepackage{unicode-math}
  \defaultfontfeatures{Scale=MatchLowercase}
  \defaultfontfeatures[\rmfamily]{Ligatures=TeX,Scale=1}
\fi
% Use upquote if available, for straight quotes in verbatim environments
\IfFileExists{upquote.sty}{\usepackage{upquote}}{}
\IfFileExists{microtype.sty}{% use microtype if available
  \usepackage[]{microtype}
  \UseMicrotypeSet[protrusion]{basicmath} % disable protrusion for tt fonts
}{}
\makeatletter
\@ifundefined{KOMAClassName}{% if non-KOMA class
  \IfFileExists{parskip.sty}{%
    \usepackage{parskip}
  }{% else
    \setlength{\parindent}{0pt}
    \setlength{\parskip}{6pt plus 2pt minus 1pt}}
}{% if KOMA class
  \KOMAoptions{parskip=half}}
\makeatother
\usepackage{xcolor}
\IfFileExists{xurl.sty}{\usepackage{xurl}}{} % add URL line breaks if available
\IfFileExists{bookmark.sty}{\usepackage{bookmark}}{\usepackage{hyperref}}
\hypersetup{
  pdftitle={Preparación clase 1 de marzo de 2016},
  pdfauthor={Julián Enrique Chitiva Bocanegra},
  hidelinks,
  pdfcreator={LaTeX via pandoc}}
\urlstyle{same} % disable monospaced font for URLs
\setlength{\emergencystretch}{3em} % prevent overfull lines
\providecommand{\tightlist}{%
  \setlength{\itemsep}{0pt}\setlength{\parskip}{0pt}}
\setcounter{secnumdepth}{-\maxdimen} % remove section numbering

\title{Preparación clase 1 de marzo de 2016}
\author{Julián Enrique Chitiva Bocanegra}
\date{}

\begin{document}
\maketitle

Tema: Propiedades de los determinantes, Determinantes e Inversas,
Espacios Vectoriales (Introducción, definición y propiedades básicas

\begin{enumerate}
\def\labelenumi{\arabic{enumi}.}
\item
  Sean \(A\) y \(B\) dos matrices \(n \times n\). Entonces
  \(det\ AB = det\ A\ det\ B\)
\item
  Sea \(A\) una matriz \(n \times n\) . Entonces
  \(det\ A\prime = det\ A\)
\item
  Sea
\end{enumerate}

\[A = \begin{pmatrix}
a_{11} & a_{12} & \ldots & a_{1n} \\
a_{21} & a_{22} & \ldots & a_{2n} \\
 \vdots & \vdots & \ddots & \vdots \\
a_{n1} & a_{n2} & \ldots & a_{nn} \\
\end{pmatrix}\]

\begin{enumerate}
\def\labelenumi{\arabic{enumi}.}
\setcounter{enumi}{3}
\item
  una matriz de \(n \times n\). Entonces
  \(det\ A = a_{i1}A_{i1} + a_{i2}A_{i2} + \ldots + a_{in}A_{in} = \sum_{k = 1}^{n}a_{ik}A_{ik},\)
  para \(i = 1,2,\ldots,n\). Es decir, se puede calcular \(det\ A\)
  expandiendo por cofactores en cualquier renglón de \(A\), también se
  puede calcular expandiendo por cofactores en cualquier columna de
  \(A\).
\item
  Si cualquier renglón o columna de \(A\) es el vector cero, entonces
  \(det\ A = 0\)
\item
  Si el renglón (columna) \(i\) de \(A\) se multiplica por un escalar
  \(c\), entonces \(det\ A\) se multiplica por \(c\)
\end{enumerate}

\[\begin{vmatrix}
a_{11} & a_{12} & \ldots & a_{1n} \\
a_{21} & a_{22} & \ldots & a_{2n} \\
 \vdots & \vdots & \ddots & \vdots \\
ca_{i1} & ca_{i2} & \ldots & ca_{in} \\
 \vdots & \vdots & \ddots & \vdots \\
a_{n1} & a_{n2} & \ldots & a_{nn} \\
\end{vmatrix} = c\begin{vmatrix}
a_{11} & a_{12} & \ldots & a_{1n} \\
a_{21} & a_{22} & \ldots & a_{2n} \\
 \vdots & \vdots & \ddots & \vdots \\
a_{i1} & a_{i2} & \ldots & a_{in} \\
 \vdots & \vdots & \ddots & \vdots \\
a_{n1} & a_{n2} & \ldots & a_{nn} \\
\end{vmatrix}\]

\begin{enumerate}
\def\labelenumi{\arabic{enumi}.}
\setcounter{enumi}{6}
\item
  Intercambiar cualesquiera dos renglones o columnas tiene el efecto de
  multiplicar el determinante por \(- 1\)
\item
  Si A tiene dos renglones o columnas iguales, entonces \(det\ A = 0\)
\item
  Si un renglón (columna) de \(A\) es múltiplo escalar de otro renglón
  (columna) entonces \(det\ A = 0\)
\item
  Si se suma un múltiplo escalar de un renglón (columna) de \(A\) a otro
  renglón (columna) de \(A\), entonces el determinante no cambia.
\item
  Si \(A\) es invertible, entonces \(det\ A \neq 0\) y
  \(det\ A^{- 1} = \frac{1}{det\ A}\)
\item
  Sea \(A\) una matriz \(n \times n\). Entonces
  \((A)(adj\ A) = \begin{pmatrix}
  det\ A & 0 & \ldots & 0 \\
  0 & det\ A & \ldots & 0 \\
   \vdots & \ldots & \ddots & \vdots \\
  0 & 0 & \ldots & det\ A \\
  \end{pmatrix} = (det\ A)I\)
\item
  Sea \(A\) una matriz de \(n \times n\). Entonces \(A\) es invertible
  si y solo si \(det\ A \neq 0\). Si \(det\ A \neq 0\), entonces
  \(A^{- 1} = \frac{1}{det\ A}adj\ A\)
\end{enumerate}

En los temas anteriores se aprendio cómo operar los objetos (vectores)
de un espacio vectorial particular ``\(\mathbb{R}^{n}\)'', ahora
generalizaremos estas operaciones a cualquier espacio vectorial; por
ejemplo, el espacio de funciones, polinomios, matrices, etc. De esta
manera, el segundo de los temas introduce el concepto de espacio
vectorial real. Un espacio vectorial real V es un conjunto de objetos,
llamados vectores, junto con dos operaciones llamadas suma y
multiplicación por escalar que satisfacen los siguientes axiomas:

\begin{enumerate}
\def\labelenumi{\arabic{enumi}.}
\item
  Si \(x \in V\) y \(y \in V\), entonces \(x + y \in V\) (cerrado bajo
  la suma)
\item
  \(\forall x,y,z \in V,\ (x + y) + z = x + (y + z)\) (ley asociativa de
  la suma de vectores)
\item
  \(\exists 0 \in V\) tal que \(\forall x \in V,\ x + 0 = 0 + x = x\)
\item
  Si \(x \in V,\ \exists - x \in V\) tal que \(x + ( - x) = 0\) (\(- x\)
  se llama el inverso aditivo)
\item
  Si \(x,y \in V\), entonces \(x + y = y + x\) (ley conmutativa de la
  suma de vectores)
\item
  Si \(x \in V\) y \(\alpha \in \mathbb{R}\), entonces
  \(\alpha x \in V\) (cerrado bajo multiplicación por escalar)
\item
  Si \(x,y \in V\) y \(alpha \in \mathbb{R}\), entonces
  \(\alpha(x + y) = \alpha x + \alpha y\) (primera ley distributiva)
\item
  Si \(x \in V\) y \(\alpha,\beta \in \mathbb{R}\), entonces
  \((\alpha + \beta)x = \alpha x + \beta x\) (segunda ley distributiva)
\item
  Si \(x \in V\) y \(\alpha,\beta \in \mathbb{R}\), entonces
  \(\alpha(\beta x) = (\alpha\beta)x\) (ley asociativa de la
  multiplicación por escalares)
\item
  \(\forall x \in V,\ 1x = x\)
\end{enumerate}

Estos dos temas sirven como base para la construccion de conjuntos de
vectores linealmente independientes, bases de espacios vectoriales, etc.

\hypertarget{desarrollo-de-los-ejercicios-propuestos}{%
\subsection{Desarrollo de los ejercicios
Propuestos}\label{desarrollo-de-los-ejercicios-propuestos}}

\hypertarget{propiedades-de-los-determinantes}{%
\subsubsection{2.2 Propiedades de los
determinantes}\label{propiedades-de-los-determinantes}}

\begin{itemize}
\item
  11. Evalúe el determinante usando los métodos de esta sección
\end{itemize}

\[\begin{matrix}
\begin{vmatrix}
1 & 1 & - 1 & 0 \\
 - 3 & 4 & 6 & 0 \\
2 & 5 & - 1 & 3 \\
4 & 0 & 3 & 0 \\
\end{vmatrix} & = & \begin{vmatrix}
1 & 1 & - 1 & 0 \\
0 & 7 & 3 & 0 \\
0 & 3 & 1 & 3 \\
0 & - 4 & 7 & 0 \\
\end{vmatrix}\begin{pmatrix}
R_{2} = R_{2} + 3R_{1} \\
R_{3} = R_{3} - 2R_{1} \\
R_{4} = R_{4} - 4R_{1} \\
\end{pmatrix} \\
\end{matrix}\]

\[\begin{matrix}
\begin{vmatrix}
1 & 1 & - 1 & 0 \\
0 & 7 & 3 & 0 \\
0 & 3 & 1 & 3 \\
0 & - 4 & 7 & 0 \\
\end{vmatrix} & = & \begin{vmatrix}
1 & 1 & - 1 & 0 \\
0 & 3 & 10 & 0 \\
0 & 3 & 1 & 3 \\
0 & - 1 & 8 & 3 \\
\end{vmatrix}\begin{pmatrix}
R_{2} = R_{2} + R_{4} \\
R_{4} = R_{4} + R_{3} \\
\end{pmatrix} \\
 & = & \begin{vmatrix}
1 & 1 & - 1 & 0 \\
0 & 0 & 34 & 9 \\
0 & 0 & 25 & 12 \\
0 & - 1 & 8 & 3 \\
\end{vmatrix}\begin{pmatrix}
R_{2} = R_{2} + 3R_{4} \\
R_{4} = R_{4} + 3R_{4} \\
\end{pmatrix} \\
 & = & ( - 1)^{2}\begin{vmatrix}
1 & 1 & - 1 & 0 \\
0 & - 1 & 8 & 3 \\
0 & 0 & 40 & 9 \\
0 & 0 & 34 & 12 \\
\end{vmatrix}\begin{pmatrix}
R_{2} \leftrightarrow R_{4} \\
R_{3} \leftrightarrow R_{4} \\
\end{pmatrix} \\
 & = & 1( - 1)(34*12 - 25*9) \\
 & = & - 183 \\
\end{matrix}\]

\begin{itemize}
\item
  27. Evalúe el determinante suponiendo
\end{itemize}

\[\begin{vmatrix}
a_{11} & a_{12} & a_{13} \\
a_{31} & a_{32} & a_{33} \\
a_{21} & a_{22} & a_{23} \\
\end{vmatrix} = 8\]

\[\begin{matrix}
\begin{vmatrix}
2a_{11} - 3a_{21} & 2a_{12} - 3a_{22} & 2a_{13} - 3a_{23} \\
a_{31} & a_{32} & a_{33} \\
a_{21} & a_{22} & a_{23} \\
\end{vmatrix} & = & \begin{vmatrix}
2a_{11} & 2a_{12} & 2a_{13} \\
a_{31} & a_{32} & a_{33} \\
a_{21} & a_{22} & a_{23} \\
\end{vmatrix}\begin{pmatrix}
R_{1} = R_{1} + 3R_{3} \\
\end{pmatrix} \\
 & = & 2\begin{vmatrix}
a_{11} & a_{12} & a_{13} \\
a_{31} & a_{32} & a_{33} \\
a_{21} & a_{22} & a_{23} \\
\end{vmatrix}\begin{pmatrix}
R_{1} = \frac{R_{1}}{2} \\
\end{pmatrix} \\
 & = & ( - 1)*2\begin{vmatrix}
a_{11} & a_{12} & a_{13} \\
a_{21} & a_{22} & a_{23} \\
a_{31} & a_{32} & a_{33} \\
\end{vmatrix}\begin{pmatrix}
R_{2} \leftrightarrow R_{3} \\
\end{pmatrix} \\
 & = & - 2*8 = 16 \\
\end{matrix}\]

\begin{itemize}
\item
  28. Usando la propiedad 2, demuestre que si \(\alpha\) es un escalar y
  \(A\) es una matriz \(n \times n\), entonces
  \(det(\alpha A) = \alpha^{n}det(A)\)
\end{itemize}

\[\begin{matrix}
det(\alpha A) & = & \begin{vmatrix}
\alpha a_{11} & \alpha a_{12} & \ldots & \alpha a_{1n} \\
\alpha a_{21} & \alpha a_{22} & \ldots & \alpha a_{2n} \\
 \vdots & \vdots & \ddots & \vdots \\
\alpha a_{n1} & \alpha a_{n2} & \ldots & \alpha a_{nn} \\
\end{vmatrix} = \alpha\begin{vmatrix}
a_{11} & a_{12} & \ldots & a_{1n} \\
\alpha a_{21} & \alpha a_{22} & \ldots & \alpha a_{2n} \\
 \vdots & \vdots & \ddots & \vdots \\
\alpha a_{n1} & \alpha a_{n2} & \ldots & \alpha a_{nn} \\
\end{vmatrix} \\
 & = & \alpha^{2}\begin{vmatrix}
a_{11} & a_{12} & \ldots & a_{1n} \\
a_{21} & a_{22} & \ldots & a_{2n} \\
 \vdots & \vdots & \ddots & \vdots \\
\alpha a_{n1} & \alpha a_{n2} & \ldots & \alpha a_{nn} \\
\end{vmatrix} = \ldots = \alpha^{n}\begin{vmatrix}
a_{11} & a_{12} & \ldots & a_{1n} \\
a_{21} & a_{22} & \ldots & a_{2n} \\
 \vdots & \vdots & \ddots & \vdots \\
a_{n1} & a_{n2} & \ldots & a_{nn} \\
\end{vmatrix} \\
 & = & \alpha^{n}det(A) \\
\end{matrix}\]

\begin{itemize}
\item
  29. Demuestre que
\end{itemize}

\[\begin{vmatrix}
1 + x_{1} & x_{2} & \ldots & x_{n} \\
x_{1} & 1 + x_{2} & \ldots & x_{n} \\
 \vdots & \vdots & \ddots & \vdots \\
x_{1} & x_{2} & \ldots & 1 + x_{n} \\
\end{vmatrix} = 1 + x_{1} + x_{2} + \ldots + x_{n}\]

\begin{itemize}
\item
  Vamos a proceder por inducción en \(n\).

  Para n=2,
\end{itemize}

\[\begin{vmatrix}
1 + x_{1} & x_{2} \\
x_{1} & 1 + x_{2} \\
\end{vmatrix} = (1 + x_{1})(1 + x_{2}) - x_{1}x_{2} = 1 + x_{1} + x_{2} + x_{1}x_{2} - x_{1}x_{2} = 1 + x_{1} + x_{2}\]

\begin{itemize}
\item
  Suponga que se cumple para n-1, vamos a ver que también se cumple para
  n:
\end{itemize}

\[\begin{matrix}
\begin{vmatrix}
1 + x_{1} & x_{2} & \ldots & x_{n} \\
x_{1} & 1 + x_{2} & \ldots & x_{n} \\
 \vdots & \vdots & \ddots & \vdots \\
x_{1} & x_{2} & \ldots & 1 + x_{n} \\
\end{vmatrix} & = & \begin{vmatrix}
1 & - 1 & \ldots & 0 \\
x_{1} & 1 + x_{2} & \ldots & x_{n} \\
 \vdots & \vdots & \ddots & \vdots \\
x_{1} & x_{2} & \ldots & 1 + x_{n} \\
\end{vmatrix}\begin{pmatrix}
R_{1} = R_{1} - R_{2} \\
\end{pmatrix} \\
 & = & \begin{vmatrix}
1 & 0 & \ldots & 0 \\
x_{1} & 1 + x_{2} + x_{1} & \ldots & x_{n} \\
 \vdots & \vdots & \ddots & \vdots \\
x_{1} & x_{2} + x_{1} & \ldots & 1 + x_{n} \\
\end{vmatrix}\begin{pmatrix}
C_{2} = C_{2} + C_{1} \\
\end{pmatrix} \\
 & = & 1*\begin{vmatrix}
1 + x_{2} + x_{1} & \ldots & x_{n} \\
 \vdots & \ddots & \vdots \\
x_{2} + x_{1} & \ldots & 1 + x_{n} \\
\end{vmatrix} \\
 & = & 1 + x_{1} + x_{2} + \ldots + x_{n}\ (\mathrm{\ hipótesis\ de\ inducción}) \\
\end{matrix}\]

\begin{itemize}
\item
  32. Una matriz \(A\) se llama \textbf{ortogonal} si \(A\) es
  invertible y \(A^{- 1} = A^{t}\). Demuestre que si \(A\) es ortogonal,
  entonces \(det\ A = \pm 1\)
\end{itemize}

\begin{itemize}
\item
  Sabemos que \(det\ A = det\ A^{t} = det\ A^{- 1}\). Ademas,
\end{itemize}

\[\begin{matrix}
det\ AA^{- 1} & = & det\ I = 1 \\
det\ A*det\ A^{- 1} & = & 1 \\
(det\ A)^{2} & = & 1 \\
det\ A & = & \pm 1 \\
\end{matrix}\]

\begin{itemize}
\item
  36.
\end{itemize}

\[D_{4} = \begin{vmatrix}
1 & 1 & 1 & 1 \\
a_{1} & a_{2} & a_{3} & a_{4} \\
a_{1}^{2} & a_{2}^{2} & a_{3}^{2} & a_{4}^{2} \\
a_{1}^{3} & a_{2}^{3} & a_{3}^{3} & a_{4}^{3} \\
\end{vmatrix}\]

\begin{itemize}
\item
  es el determinante de Vandermode de \(4 \times 4\)
\end{itemize}

\begin{itemize}
\item
  Demuestre que
  \(D_{4} = (a_{2} - a_{1})(a_{3} - a_{1})(a_{4} - a_{1})(a_{3} - a_{2})(a_{4} - a_{2})(a_{4} - a_{3})\)
\end{itemize}

\[\begin{matrix}
D_{4} & = & \begin{vmatrix}
1 & 1 & 1 & 1 \\
a_{1} & a_{2} & a_{3} & a_{4} \\
a_{1}^{2} & a_{2}^{2} & a_{3}^{2} & a_{4}^{2} \\
a_{1}^{3} & a_{2}^{3} & a_{3}^{3} & a_{4}^{3} \\
\end{vmatrix} = \begin{vmatrix}
1 & 1 & 1 & 1 \\
0 & a_{2} - a_{1} & a_{3} - a_{1} & a_{4} - a_{1} \\
0 & a_{2}^{2} - a_{1}a_{2} & a_{3}^{2} - a_{1}a_{3} & a_{4}^{2} - a_{1}a_{4} \\
0 & a_{2}^{3} - a_{1}a_{2}^{2} & a_{3}^{3} - a_{1}a_{3}^{2} & a_{4}^{3} - a_{1}a_{4}^{2} \\
\end{vmatrix} \\
 & = & (a_{2} - a_{1})(a_{3} - a_{1})(a_{4} - a_{1})\begin{vmatrix}
1 & 1 & 1 \\
a_{2} & a_{3} & a_{4} \\
a_{2}^{2} & a_{3}^{2} & a_{4}^{2} \\
\end{vmatrix} \\
 & = & (a_{2} - a_{1})(a_{3} - a_{1})(a_{4} - a_{1})\begin{vmatrix}
1 & 1 & 1 \\
0 & a_{3} - a_{2} & a_{4} - a_{2} \\
0 & a_{3}^{2} - a_{2}a_{3} & a_{4}^{2} - a_{2}a_{4} \\
\end{vmatrix} \\
 & = & (a_{2} - a_{1})(a_{3} - a_{1})(a_{4} - a_{1})(a_{3} - a_{2})(a_{4} - a_{2})\begin{vmatrix}
1 & 1 \\
a_{3} & a_{4} \\
\end{vmatrix} \\
 & = & (a_{2} - a_{1})(a_{3} - a_{1})(a_{4} - a_{1})(a_{3} - a_{2})(a_{4} - a_{2})(a_{4} - a_{3}) \\
\end{matrix}\]

\begin{itemize}
\item
  38. Sea \(A = \begin{pmatrix}
  a_{11} & a_{12} \\
  a_{21} & a_{22} \\
  \end{pmatrix}\) y \(B = \begin{pmatrix}
  b_{11} & b_{12} \\
  b_{21} & b_{22} \\
  \end{pmatrix}\)

  \begin{enumerate}
  \def\labelenumi{\alph{enumi}.}
  \item
    Escriba el producto \(AB\)
  \end{enumerate}
\end{itemize}

\[\begin{matrix}
AB & = & \begin{pmatrix}
a_{11} & a_{12} \\
a_{21} & a_{22} \\
\end{pmatrix}\begin{pmatrix}
b_{11} & b_{12} \\
b_{21} & b_{22} \\
\end{pmatrix} \\
 & = & \begin{pmatrix}
a_{11}b_{11} + a_{12}b_{21} & a_{11}b_{12} + a_{12}b_{22} \\
a_{21}b_{11} + a_{22}b_{21} & a_{21}b_{12} + a_{22}b_{22} \\
\end{pmatrix} \\
\end{matrix}\]

\begin{enumerate}
\def\labelenumi{\alph{enumi}.}
\setcounter{enumi}{1}
\item
  Calcule \(det\ A,\ det\ B\) y \(det\ AB\)
\end{enumerate}

\[\begin{matrix}
det\ A & = & \begin{vmatrix}
a_{11} & a_{12} \\
a_{21} & a_{22} \\
\end{vmatrix} \\
 & = & a_{11}a_{22} - a_{12}a_{21} \\
\end{matrix}\]

\[\begin{matrix}
det\ B & = & \begin{vmatrix}
b_{11} & b_{12} \\
b_{21} & b_{22} \\
\end{vmatrix} \\
 & = & b_{11}b_{22} - b_{12}b_{21} \\
\end{matrix}\]

\[\begin{matrix}
det\ AB & = & \begin{vmatrix}
a_{11}b_{11} + a_{12}b_{21} & a_{11}b_{12} + a_{12}b_{22} \\
a_{21}b_{11} + a_{22}b_{21} & a_{21}b_{12} + a_{22}b_{22} \\
\end{vmatrix} \\
 & = & (a_{11}b_{11} + a_{12}b_{21})(a_{21}b_{12} + a_{22}b_{22}) - (a_{11}b_{12} + a_{12}b_{22})(a_{21}b_{11} + a_{22}b_{21}) \\
 & = & a_{11}a_{22}b_{11}b_{22} - a_{11}a_{22}b_{12}b_{21} + a_{12}a_{21}b_{12}b_{21} - a_{12}a_{21}b_{11}b_{22} \\
 & = & a_{11}a_{22}(b_{11}b_{22} - b_{12}b_{21}) - a_{12}a_{21}(b_{11}b_{22} - b_{12}b_{21}) \\
 & = & (a_{11}a_{22} - a_{12}a_{21})(b_{11}b_{22} - b_{12}b_{21}) \\
\end{matrix}\]

\begin{itemize}
\item
  41. La matriz \(A\) se llama \textbf{idempotente} si\(A^{2} = A\).
  Cuáles son los posibles valores para \(det\ A\) si \(A\) es
  idempotente?
\end{itemize}

\[\begin{matrix}
det\ A^{2} & = & det\ A \\
(det\ A)^{2} & = & det\ A \\
det\ A = 1 \\
\end{matrix}\]

\hypertarget{determinantes-e-inversas}{%
\subsubsection{2.4 Determinantes e
Inversas}\label{determinantes-e-inversas}}

\begin{itemize}
\item
  9. Utilice los métodos de esta sección para determinar si la matriz es
  invertible. si lo es, calcule la inversa
\end{itemize}

\[\begin{pmatrix}
2 & - 1 & 4 \\
 - 1 & 0 & 5 \\
19 & - 7 & 3 \\
\end{pmatrix}\]

\[\begin{matrix}
\begin{vmatrix}
2 & - 1 & 4 \\
 - 1 & 0 & 5 \\
19 & - 7 & 3 \\
\end{vmatrix} & = & 2(35) + ( - 3 - 95) + 4(7) \\
 & = & 70 - 98 + 28 \\
 & = & 0 \Rightarrow \mathrm{\text{\ La\ matriz\ no\ es\ invertible}} \\
\end{matrix}\]

\begin{itemize}
\item
  13. Use determinates para demostrar que una matriz \(A\) de
  \(n \times n\) es invertible si y solo si \(A^{t}\) es invertible.
  Como \(det\ A = det\ A^{t}\). Suponga que \(A\) es invertible
  \(\Rightarrow det\ A \neq 0 \Rightarrow det\ A^{t} \neq 0\). Ahora,
  suponga que \(A^{t}\) es invertible
  \(\Rightarrow det\ A^{t} \neq 0 \Rightarrow det\ A \neq 0\)
\item
  14. Para \(A = \begin{pmatrix}
  1 & 1 \\
  2 & 5 \\
  \end{pmatrix}\), verifique que \(det\ A^{- 1} = \frac{1}{det\ A}\)
  \(det\ A = 5 - 2 = 3\). Calculamos \(A^{- 1}\) por el método de la
  matriz adjunta, entonces, \(A^{- 1} = \frac{1}{3}\begin{pmatrix}
  5 & - 1 \\
   - 2 & 1 \\
  \end{pmatrix}\) y ahora, calculamos
  \(det\ A^{- 1} = \frac{5}{9} - \frac{2}{9} = \frac{3}{9} = \frac{1}{3}\)
\item
  Para qué valores de \(\alpha\) la matriz \(\begin{pmatrix}
  \alpha & - 3 \\
  4 & 1 - \alpha \\
  \end{pmatrix}\) es no invertible?
\end{itemize}

\begin{itemize}
\item
  \(\begin{vmatrix}
  \alpha & - 3 \\
  4 & 1 - \alpha \\
  \end{vmatrix} = \alpha(1 - \alpha) + 12\). Para que sea no invertible,
  su determinante tiene que ser cero. Entonces,
\end{itemize}

\[\begin{matrix}
\alpha(1 - \alpha) + 12 & = & 0 \\
\alpha^{2} - \alpha - 12 & = & 0\  \Rightarrow \\
\alpha & = & \frac{1 \pm \sqrt{1 + 48}}{2} \\
\alpha & = & \frac{1 \pm 7}{2} \\
 & & \vdots \\
\alpha_{1} & = & 4 \\
\alpha_{2} & = & - 3 \\
\end{matrix}\]

\begin{itemize}
\item
  19. Sea \(\theta \in \mathbb{R}\). Demuestre que \(\begin{pmatrix}
  \mathrm{\cos}\theta & \mathrm{\sin}\theta \\
   - \mathrm{\sin}\theta & \mathrm{\cos}\theta \\
  \end{pmatrix}\) es invertible y encuentre su inversa.
\end{itemize}

\[\begin{vmatrix}
\mathrm{\cos}\theta & \mathrm{\sin}\theta \\
 - \mathrm{\sin}\theta & \mathrm{\cos}\theta \\
\end{vmatrix} = \mathrm{\cos}^{2}\theta + \mathrm{\sin}^{2}\theta = 1 \neq 0\]

\begin{itemize}
\item
  Calculamos la inversa por el método de la adjunta:
\end{itemize}

\[\begin{pmatrix}
\mathrm{\cos}\theta & \mathrm{\sin}\theta \\
 - \mathrm{\sin}\theta & \mathrm{\cos}\theta \\
\end{pmatrix}^{- 1} = \begin{pmatrix}
\mathrm{\cos}\theta & - \mathrm{\sin}\theta \\
\mathrm{\sin}\theta & \mathrm{\cos}\theta \\
\end{pmatrix}\]

\hypertarget{espacios-vectoriales-introducciuxf3n-definiciuxf3n-y-propiedades-buxe1sicas}{%
\subsubsection{4.2 Espacios vectoriales (Introducción, definición y
propiedades
básicas)}\label{espacios-vectoriales-introducciuxf3n-definiciuxf3n-y-propiedades-buxe1sicas}}

\begin{itemize}
\item
  3.\(V = \{(x,y);y \leq 0;\ x,y \in \mathbb{R}\}\) con la suma de
  vectores y la multiplicación por escalar usuales.
\end{itemize}

\begin{itemize}
\item
  No. Note que para \((x,y) \in V,\)
  \(y < 0\  \Rightarrow ( - x, - y) \notin V,\) porque \(- y > 0\)
\end{itemize}

\begin{itemize}
\item
  4. Los vectores en el plano que están en el primer cuadrante.
\end{itemize}

\begin{itemize}
\item
  No. Por un argumento similiar al anterior si \((x,y)\) está en el
  primer cuadrante, entonces, \(( - x, - y)\) está en el tercer
  cuadrante y no pertenece.
\end{itemize}

\begin{itemize}
\item
  7. El conjunto de las matrices simétricas \(n \times n\) bajo la suma
  y la multiplicación por escalar usuales.
\end{itemize}

\begin{itemize}
\item
  Sí. Sean \(A,B\) matrices simétricas y \(\alpha \in \mathbb{R}\). Es
  decir, \(A\prime = A\) y \(B\prime = B\). Note que
  \((A + \alpha B)\prime = A\prime + (\alpha B\prime) = A + \alpha B\),
  lo cual implica que es cerrado bajo la suma y la multiplicación por
  escalar. Los demás axiomas los cumple trivialmente.
\end{itemize}

\begin{itemize}
\item
  8. El conjunto de matrices \(2 \times 2\) de la forma
  \(\begin{pmatrix}
  0 & a \\
  b & 0 \\
  \end{pmatrix}\) bajo la suma y multiplicación por un escalar usuales.
\end{itemize}

\[\begin{pmatrix}
0 & a \\
b & 0 \\
\end{pmatrix} + \begin{pmatrix}
0 & c \\
c & 0 \\
\end{pmatrix} = \begin{pmatrix}
0 & a + c \\
b + c & 0 \\
\end{pmatrix} \in V\]

\[\begin{pmatrix}
0 & 0 \\
0 & 0 \\
\end{pmatrix} \in V\]

\[\alpha\begin{pmatrix}
0 & a \\
b & 0 \\
\end{pmatrix} = \begin{pmatrix}
0 & \alpha a \\
\alpha b & 0 \\
\end{pmatrix} \in V\]

\begin{itemize}
\item
  El resto de axiomas los cumple trivialmente.
\end{itemize}

\begin{itemize}
\item
  9. El conjunto de matrices \(2 \times 2\) de la forma
  \(\begin{pmatrix}
  1 & a \\
  b & 1 \\
  \end{pmatrix}\) bajo la suma y multiplicación por un escalar usuales.
\end{itemize}

\[\begin{pmatrix}
1 & a \\
b & 1 \\
\end{pmatrix} + \begin{pmatrix}
1 & c \\
c & 1 \\
\end{pmatrix} = \begin{pmatrix}
2 & a + c \\
b + c & 2 \\
\end{pmatrix} \notin V\]

\begin{itemize}
\item
  11. El conjunto de polinomios de grado \(\leq n\) con término
  constante 0.
\end{itemize}

\begin{itemize}
\item
  Sí. Note que la suma de dos polinomios con término constante 0 va a
  tener su término constante 0. \(0 \in V\).
  \(p(x) \in V \Rightarrow - p(x) \in V,\) porque \(- 0 = 0\).
  \(\alpha p(x) \in V,\) porque \(\alpha*0 = 0\). El resto de axiomas
  los cumple trivialmente.
\end{itemize}

\begin{itemize}
\item
  14. El conjunto de puntos en \(\mathbb{R}^{3}\) que están sobre la
  recta que pasa por el origen
\end{itemize}

\begin{itemize}
\item
  Sí. \(t_{1}(a,b,c) + t_{2}(a,b,c) = (t_{1} + t_{2})(a,b,c) \in V\).
  \((0,0,0) \in V\). Si
  \((a,b,c) = t(d,e,f) \in V \Rightarrow ( - a, - b, - c) = ( - t)(d,e,f) \in V\).
  \(\alpha(t(a,b,c)) = (\alpha t)(a,b,c)\forall\alpha \in \mathbb{R}\).
  Los demás axiomas se cumplen trivialmente.
\end{itemize}

\end{document}
