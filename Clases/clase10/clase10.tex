\documentclass[dvipsnames,xcolor=x11names]{beamer}
\usepackage[spanish]{babel}
\usepackage[utf8]{inputenc}
\usepackage[all]{xy}
\usepackage{afterpage}
\usepackage{tikz}
\usepackage{cancel}
\usepackage{verbatim}
\usepackage{tabu}
\usepackage{xfrac}
\usepackage{mathrsfs}
\usepackage{amsthm}
\usepackage{amssymb}
\usepackage{bbm}
\usepackage{enumerate}
\usepackage{booktabs}
\usepackage{relsize}
\usepackage{hyperref}
\usepackage{float}
\usepackage{longtable}
\usepackage{amsmath}
\usepackage{multirow}
\usepackage{multicol}
\usepackage{colortbl}
\usepackage{adjustbox}
\usepackage{xfrac}
\usepackage{bm}
\usepackage{keystroke}
\usepackage{wrapfig}
\usepackage{graphicx}
\usepackage{csvsimple}
\usepackage{otros/pgf-pie}
\usetikzlibrary{decorations.pathmorphing, patterns,shapes}
\usetikzlibrary{positioning}
\usepackage{pgfplots}
\pgfplotsset{compat=1.12}
\usepackage{pgfplotstable}


\PassOptionsToPackage{demo}{graphicx}

\newcommand{\hlc}[2][yellow]{ {\sethlcolor{#1} \hl{#2}} }

\newcommand*{\rom}[1]{\expandafter\romannumeral #1}
\newcommand{\Rom}[1]{\uppercase\expandafter{\romannumeral #1\relax}}

\newcommand{\Importante}[2]{{\color{#1}#2}}
\newcommand{\importante}[2]{{\color{#1}\underline{#2}}}

\renewcommand{\baselinestretch}{1}
\setlength{\parskip}{\baselineskip}

\usetheme{Boadilla}
\definecolor{colorClase}{RGB}{0,108,91}
\usecolortheme[named=colorClase]{structure}
\usepackage{natbib}


\theoremstyle{plain}
  \newtheorem{teorema}{Teorema}
  \newtheorem{proposicion}{Proposición}
  \newtheorem{corolario}{Corolario}
  \newtheorem{lema}[teorema]{Lema}
\theoremstyle{definition}
  \newtheorem{definicion}{Definici\'on}
  \newtheorem{ejemplo}{Ejemplo}
  
  
\setbeamertemplate{caption}[numbered]
 
\title{Taller usos de \LaTeX}
\subtitle{Conexión con otros software.}
\setbeamersize{text margin left=25pt,text margin right=25pt}

\author[Julián Chitiva Bocanegra]{Julián Enrique Chitiva Bocanegra}
\institute[Uniandes] 
{Universidad de los Andes\\ Facultad de Economía}
\titlegraphic{\includegraphics[width=0.8cm]{Clases/img/uniandes_logo.png}
}

\pgfdeclareimage[height=.8cm]{university-logo}{img/uniandes_logo.png}
\logo{\pgfuseimage{university-logo}}

\date{\today}
\subject{}
\usepackage{caption}
\usepackage{subcaption}

\begin{document}

\begin{frame}
  \titlepage
\end{frame}

\begin{frame}{Contenido.}
  \tableofcontents\footnote{\href{http://www-math.mit.edu/~psh/exam/examdoc.pdf}{Aquí la documentación de exam}}
\end{frame}

\section{Excel.}
\begin{frame}{Contenido.}
  \tableofcontents[currentsection]
\end{frame}
\begin{frame}{Excel.}
\begin{itemize}[<+->]
    \item La conexión con Excel es muy sencilla.
    \item Nos permite exportar tablas de excel a código de \LaTeX\ usando un complemento.
    \item El complemento lo pueden descargar \href{https://ctan.org/pkg/excel2latex?lang=en}{\textcolor{colorClase}{aquí}}.
\end{itemize}
    
\end{frame}
\section{Stata.}
\begin{frame}{Contenido.}
  \tableofcontents[currentsection]
\end{frame}
\begin{frame}[fragile]{Stata.}
    \begin{itemize}[<+->]
        \item La conexión con Stata es un tris más complicada. 
        \item Requiere los paquetes \texttt{Stata,pagedims,sj} de \LaTeX\ (están en la clase de hoy)
        \item Requiere instalar el paquete \texttt{texdoc} \verb!ssc install texdoc!
        \item Requiere correr el siguiente comando 
    \begin{verbatim}
    net from http://www.stata-journal.com/production
    net install sjlatex
    \end{verbatim}
    \item Requiere escribir el \texttt{dofile} con los comandos de \LaTeX.
    \item Finalmente compilar este \texttt{dofile} mediante \texttt{texdoc}.
    \end{itemize}
\end{frame}
\section{R.}
\begin{frame}{Contenido.}
  \tableofcontents[currentsection]
\end{frame}
\begin{frame}{R.}
    \begin{itemize}[<+->]
        \item El script es un poco más fácil de escribir que para el caso de Stata.
        \item Hay que tener instalado \TeX\ en el computador.
        \item Es como escribir cualquier archivo \textit{.tex}.
        \item Hay que tener el paquete \texttt{Sweave} de R.
    \end{itemize}
\end{frame}
\section{Pandoc.}
\begin{frame}{Contenido.}
  \tableofcontents[currentsection]
\end{frame}
\begin{frame}{Pandoc.}
    \begin{itemize}
        \item Este es el más complicado de usar.
        \item Toda la información la pueden encontrar en \href{https://pandoc.org}{\textcolor{colorClase}{este link}}.
        \item Se puede transformar de \textit{.tex} a otros formatos
    \end{itemize}
\end{frame}
\subsection{Word.}
\begin{frame}{Contenido.}
  \tableofcontents[currentsection, currentsubsection]
\end{frame}
\begin{frame}[fragile]{Word.}
    \begin{itemize}
        \item De \LaTeX a Word (Cualquier formato) funciona bastante bien. La estructura es:
        \begin{itemize}
            \item \verb!pandoc -s pandoc.tex -o pandoc.docx!
        \end{itemize}
        
        \item De Word (Cualquier formato) a \LaTeX\ no funciona tan bien (pero funciona)
        
        \begin{itemize}
            \item \verb!pandoc -s pandoc.docx -t latex -o pandoc2.tex!
        \end{itemize}
    \end{itemize}
\end{frame}

\end{document}