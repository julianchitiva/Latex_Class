\documentclass[12pt]{article}
\usepackage{otros/stata, graphicx}
\usepackage[utf8]{inputenc}
\usepackage[spanish]{babel}
\title{Este es un ejemplo de un \textit{.tex} desde STATA}
\author{Julián Enrique Chitiva}
\begin{document}
\maketitle
\section{Esta es la primera sección}
Aquí vamos a poner las estadísticas descriptivas
\begin{center}
\begin{stlog}\input{Clases/clase10/Stata/prueba_1.log.tex}\end{stlog}
\end{center}
\begin{enumerate}
\item Este es un item de una lista
\item Ahora vamos a hacer una regresión
\begin{center}
\begin{stlog}. reg price mpg length weight foreign
{\smallskip}
      Source {\VBAR}       SS           df       MS      Number of obs   =        74
\HLI{13}{\PLUS}\HLI{34}   F(4, 69)        =     21.01
       Model {\VBAR}   348708940         4    87177235   Prob > F        =    0.0000
    Residual {\VBAR}   286356456        69  4150093.57   R-squared       =    0.5491
\HLI{13}{\PLUS}\HLI{34}   Adj R-squared   =    0.5230
       Total {\VBAR}   635065396        73  8699525.97   Root MSE        =    2037.2
{\smallskip}
\HLI{13}{\TOPT}\HLI{64}
       price {\VBAR}      Coef.   Std. Err.      t    P>|t|     [95\% Conf. Interval]
\HLI{13}{\PLUS}\HLI{64}
         mpg {\VBAR}  -13.40719   72.10761    -0.19   0.853    -157.2579    130.4436
      length {\VBAR}  -92.48018    33.5912    -2.75   0.008    -159.4928   -25.46758
      weight {\VBAR}   5.716181   1.016095     5.63   0.000     3.689127    7.743235
     foreign {\VBAR}   3550.194   655.4564     5.42   0.000     2242.594    4857.793
       _cons {\VBAR}    5515.58   5241.941     1.05   0.296    -4941.807    15972.97
\HLI{13}{\BOTT}\HLI{64}
{\smallskip}
\end{stlog}
\end{center}
\item Ahora vamos a hacer otra regresión
\begin{center}
\begin{stlog}\input{Clases/clase10/Stata/prueba_3.log.tex}\end{stlog}
\end{center}
\end{enumerate}

\begin{tabular}{lc} \hline
 & (1) \\
VARIABLES & price \\ \hline
 &  \\
mpg & -13.41 \\
 & (72.11) \\
longitud & -36.99*** \\
 & (13.44) \\
weight & 5.716*** \\
 & (1.016) \\
foreign & 3,550*** \\
 & (655.5) \\
Constant & 5,516 \\
 & (5,242) \\
 &  \\
Observations & 74 \\
 R-squared & 0.549 \\ \hline
\multicolumn{2}{c}{ Standard errors in parentheses} \\
\multicolumn{2}{c}{ *** p$<$0.01, ** p$<$0.05, * p$<$0.1} \\
\end{tabular}

\end{document}
