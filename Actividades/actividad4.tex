\documentclass[dvipsnames,xcolor=x11names, handout]{beamer}
\usepackage[spanish]{babel}
\usepackage[utf8]{inputenc}
\usepackage[all]{xy}
\usepackage{afterpage}
\usepackage{tikz}
\usepackage{cancel}
\usepackage{verbatim}
\usepackage{tabu}
\usepackage{xfrac}
\usepackage{mathrsfs}
\usepackage{amsthm}
\usepackage{amssymb}
\usepackage{bbm}
\usepackage{enumerate}
\usepackage{booktabs}
\usepackage{relsize}
\usepackage{hyperref}
\usepackage{float}
\usepackage{longtable}
\usepackage{amsmath}
\usepackage{multirow}
\usepackage{multicol}
\usepackage{colortbl}
\usepackage{adjustbox}
\usepackage{xfrac}
\usepackage{bm}
\usepackage{keystroke}
\usepackage{wrapfig}
\usepackage{graphicx}
\usepackage{csvsimple}
\usepackage{otros/pgf-pie}
\usetikzlibrary{decorations.pathmorphing, patterns,shapes}
\usetikzlibrary{positioning}
\usepackage{pgfplots}
\pgfplotsset{compat=1.12}
\usepackage{pgfplotstable}


\PassOptionsToPackage{demo}{graphicx}

\newcommand{\hlc}[2][yellow]{ {\sethlcolor{#1} \hl{#2}} }

\newcommand*{\rom}[1]{\expandafter\romannumeral #1}
\newcommand{\Rom}[1]{\uppercase\expandafter{\romannumeral #1\relax}}

\newcommand{\Importante}[2]{{\color{#1}#2}}
\newcommand{\importante}[2]{{\color{#1}\underline{#2}}}

\renewcommand{\baselinestretch}{1}
\setlength{\parskip}{\baselineskip}

\usetheme{Boadilla}

\definecolor{colorClase}{RGB}{0,108,91}
\usecolortheme[named=colorClase]{structure}
\usepackage{natbib}


\theoremstyle{plain}
  \newtheorem{teorema}{Teorema}
  \newtheorem{proposicion}{Proposición}
  \newtheorem{corolario}{Corolario}
  \newtheorem{lema}[teorema]{Lema}
\theoremstyle{definition}
  \newtheorem{definicion}{Definici\'on}
  \newtheorem{ejemplo}{Ejemplo}
  
  
\setbeamertemplate{caption}[numbered]
 
\title{Taller usos de \LaTeX}
\subtitle{Beamer}
\setbeamersize{text margin left=25pt,text margin right=25pt}

\author[Julián Chitiva Bocanegra]{Julián Enrique Chitiva Bocanegra}
\institute[Uniandes] 
{Universidad de los Andes\\ Facultad de Economía}
\titlegraphic{\includegraphics[width=0.8cm]{img/uniandes_logo.png}
}
\renewcommand{\insertframenumber}{\roman{framenumber}}
\date{\today}
\subject{}
\usepackage{caption}
\usepackage{subcaption}

\begin{document}

\begin{frame}
  \titlepage
\end{frame}

\begin{frame}{Algunos dibujos bonitos}

\begin{multicols*}{4}
\uncover<1,2,10>{
\begin{tikzpicture}
[domain=-1:1] 
    \draw[-](0,-1)--(0,1);
    \draw[-](-1,0)--(1,0);
    \draw[-](0,1)--(0,1) node[circle, fill ,black, inner sep=1.5pt]{};
    \draw[-](0,-1)--(0,-1) node[circle, fill ,black, inner sep=1.5pt]{};
    \draw[-](1,0)--(1,0) node[circle, fill ,black, inner sep=1.5pt]{};
    \draw[-](-1,0)--(-1,0) node[circle, fill ,black, inner sep=1.5pt]{};
    \draw[-](0,0)--(0,0) node[circle, fill ,black, inner sep=1.5pt]{};
\end{tikzpicture}}

\columnbreak 

\uncover<2,4,6,8>{
\begin{tikzpicture}
[domain=-1:1] 
    \draw[-](0,0.5)--(0,1);
    \filldraw[fill=gray!20](0,0.5)--(-0.5,-0.5)--(0.5,-0.5)--(0,0.5);
    \draw[-](0.5,-0.5)--(1,-1);
    \draw[-](-0.5,-0.5)--(-1,-1);
    \draw[-](0,0.5)--(0,0.5)node[circle, fill ,black, inner sep=1.5pt]{};
    \draw[-](-0.5,-0.5)--(-0.5,-0.5)node[circle, fill ,black, inner sep=1.5pt]{};
    \draw[-](0.5,-0.5)--(0.5,-0.5)node[circle, fill ,black, inner sep=1.5pt]{};
    \draw[-](0,1)--(0,1) node[circle, fill ,black, inner sep=1.5pt]{};
    \draw[-](-1,-1)--(-1,-1) node[circle, fill ,black, inner sep=1.5pt]{};
    \draw[-](1,-1)--(1,-1) node[circle, fill ,black, inner sep=1.5pt]{};
\end{tikzpicture}}

\vspace*{0.5cm}

\uncover<3,5,7>{
\begin{tikzpicture}
[domain=-1:1] 
    \filldraw[fill=gray!20](0,0.5)--(-0.5,0)--(0,-0.5)--(0.5,0)--(0,0.5);
    \draw[-](-0.5,0)--(-1,0);
    \draw[-](0.5,0)--(1,0);
    \draw[-](1,0)--(1,0) node[circle, fill ,black, inner sep=1.5pt]{};
    \draw[-](-1,0)--(-1,0) node[circle, fill ,black, inner sep=1.5pt]{};
    \draw[-](0,0.5)--(0,0.5) node[circle, fill ,black, inner sep=1.5pt]{};
    \draw[-](0.5,0)--(0.5,0) node[circle, fill ,black, inner sep=1.5pt]{};
    \draw[-](0,-0.5)--(0,-0.5) node[circle, fill ,black, inner sep=1.5pt]{};
    \draw[-](-0.5,0)--(-0.5,0) node[circle, fill ,black, inner sep=1.5pt]{};
\end{tikzpicture}}

\columnbreak 

\uncover<1,4,8>{
\begin{tikzpicture}
[domain=-1:1] 
    \filldraw[fill=gray!20](0,0.5)--(-0.5,0)--(0,-0.5)--(0.5,0)--(0,0.5);
    \draw[-](-0.5,0)--(-1,0);
    \draw[-](0.5,0)--(1,0);
    \draw[-](1,0)--(1,0) node[circle, fill ,black, inner sep=1.5pt]{};
    \draw[-](-1,0)--(-1,0) node[circle, fill ,black, inner sep=1.5pt]{};
    \draw[-](0,0.5)--(0,0.5) node[circle, fill ,black, inner sep=1.5pt]{};
    \draw[-](0.5,0)--(0.5,0) node[circle, fill ,black, inner sep=1.5pt]{};
    \draw[-](0,-0.5)--(0,-0.5) node[circle, fill ,black, inner sep=1.5pt]{};
    \draw[-](-0.5,0)--(-0.5,0) node[circle, fill ,black, inner sep=1.5pt]{};
\end{tikzpicture}}

\vspace*{0.5cm}

\uncover<5,6,8>{
\begin{tikzpicture}
[domain=-1:1] 
    \filldraw[fill=gray!20](0,0.5)--(-0.5,0)--(0,-0.5)--(0.5,0)--(0,0.5);
    \draw[-](0,0.5)--(0,1);
    \draw[-](0.5,0)--(1,0);
    \draw[-](1,0)--(1,0) node[circle, fill ,black, inner sep=1.5pt]{};
    \draw[-](0,1)--(0,1) node[circle, fill ,black, inner sep=1.5pt]{};
    \draw[-](0,0.5)--(0,0.5) node[circle, fill ,black, inner sep=1.5pt]{};
    \draw[-](0.5,0)--(0.5,0) node[circle, fill ,black, inner sep=1.5pt]{};
    \draw[-](0,-0.5)--(0,-0.5) node[circle, fill ,black, inner sep=1.5pt]{};
    \draw[-](-0.5,0)--(-0.5,0) node[circle, fill ,black, inner sep=1.5pt]{};
    \draw[-](-1,0)--(-1,0) node {};
\end{tikzpicture}}

\columnbreak 

\uncover<2,5,7>{
\begin{tikzpicture}
[domain=-1:1] 
    \filldraw[fill=gray!20](0,0.5)--(-0.47,0.15)--(-0.3,-0.4)--(0.3,-0.4)--(0.47,0.15)--(0,0.5);
    \draw[-](0.47,0.15)--(1,0.3);
    \draw[-](0,0.5)--(0,0.5) node[circle, fill ,black, inner sep=1.5pt]{};
    \draw[-](-0.47,0.15)--(-0.47,0.15) node[circle, fill ,black, inner sep=1.5pt]{};
    \draw[-](-0.3,-0.4)--(-0.3,-0.4) node[circle, fill ,black, inner sep=1.5pt]{};
    \draw[-](0.3,-0.4)--(0.3,-0.4) node[circle, fill ,black, inner sep=1.5pt]{};
    \draw[-](0.47,0.15)--(0.47,0.15) node[circle, fill ,black, inner sep=1.5pt]{};
    \draw[-](1,0.3)--(1,0.3) node[circle, fill ,black, inner sep=1.5pt]{};
    
\end{tikzpicture}}

\vspace*{0.8cm}
\uncover<7-10>{
\begin{tikzpicture}
[domain=-1:1] 
    \filldraw[fill=gray!20](0.5,0)--(0.25,0.44)--(-0.25,0.44)--(-0.5,0)--(-0.25,-0.44)--(0.25,-0.44)--(0.5,0);
    \draw[-](0.5,0)--(0.5,0) node[circle, fill ,black, inner sep=1.5pt]{};
    \draw[-](0.25,0.44)--(0.25,0.44) node[circle, fill ,black, inner sep=1.5pt]{};
    \draw[-](-0.25,0.44)--(-0.25,0.44) node[circle, fill ,black, inner sep=1.5pt]{};
    \draw[-](-0.5,0)--(-0.5,0) node[circle, fill ,black, inner sep=1.5pt]{};
    \draw[-](-0.25,-0.44)--(-0.25,-0.44) node[circle, fill ,black, inner sep=1.5pt]{};
    \draw[-](0.25,-0.44)--(0.25,-0.44) node[circle, fill ,black, inner sep=1.5pt]{};
\end{tikzpicture}}

\end{multicols*}
\end{frame}
\end{document}
