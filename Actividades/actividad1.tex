\documentclass[12pt]{article}
\usepackage{amsfonts, amssymb, amsmath, mathrsfs}
\usepackage[utf8]{inputenc}
\usepackage{booktabs}
\usepackage{multirow}
\usepackage[spanish]{babel}
\usepackage{graphicx}
\usepackage{float}
\usepackage{anysize}
\usepackage{xfrac}
\renewcommand\spanishtablename{Tablita}
\begin{document}



\begin{table}[h!]
    \centering
    \caption{Esta es la Tablita \ref{tab:1} del Documento}
    \begin{tabular}{c *{3}{c} c *{3}{c} c *{3}{c}}\cmidrule[1pt]{2-4}\cmidrule[1pt]{6-8}\cmidrule[1pt]{10-12}
          \multicolumn{1}{c}{3} & \multicolumn{3}{c}{M}&\multirow{4}{*}{}& \multicolumn{3}{c}{N} &\multirow{4}{*}{}& \multicolumn{3}{c}{Q}\\\cmidrule{2-4}\cmidrule{6-8}\cmidrule{10-12}
          & 1\textbackslash 2 & A & B && 1\textbackslash 2 & A & B && 1\textbackslash 2 & A & B \\\cmidrule{2-4}\cmidrule{6-8}\cmidrule{10-12}
          & X & 0,0,3 & 0,0,0 & & X & 2,2,2 & 0,0,0 & &X & 0,0,0 & 1,1,1 \\  
          & Y & 1,1,1 & 0,0,0 & & Y & 0,0,0 & 2,2,2 & & Y & 0,0,0 & 0,0,3 \\ \cmidrule[1pt]{2-4}\cmidrule[1pt]{6-8}\cmidrule[1pt]{10-12} 
    \end{tabular} 
    \label{tab:1}
\end{table}
 
 \begin{minipage}{0.4\linewidth}
 En el lado derecho de la página encontrán una ecuación numerada. En este caso es la ecuacion \ref{eq:1}
 \end{minipage}
 \begin{minipage}{0.6\linewidth}
 \begin{equation}
 \gamma = \begin{cases} (X,A,N) & \text{con probabilidad }\sfrac{1}{2} \\
 (Y,B,N) & \text{con probabilidad }\frac{1}{2} \\
 \text{Los demás} & \text{con probabilidad } 0 \\ \end{cases}
 \label{eq:1}
 \end{equation}
 \end{minipage}
 
 \begin{table}[h!]
 \centering
     \begin{tabular}{|c|c|c|c|}\hline
      1\textbackslash 2 & A & B & C \\ \hline
      A & $2+2\delta,2+2\delta$ & $0+2\delta,0+2\delta$ & $6+2\delta,1+2\delta$ \\ \hline
      B & $0+2\delta,0+2\delta$ & $4+2\delta,4+2\delta$ & $0+2\delta,0+2\delta$ \\ \hline
      C & $1+2\delta,6+2\delta$ & $0+2\delta,0+2\delta$ & $5+4\delta,5+4\delta$ \\ \hline
    \end{tabular}
     \caption{Esta es una Tablita que contiene símbolos matemáticos}
     \label{tab:2}
 \end{table}
 
 
 \begin{minipage}{0.3\linewidth}
 \begin{align}
    5+4\delta&\geq 6+2\delta \label{eq:2}\\
    2\delta&\geq 1\nonumber\\
    \delta&\geq \frac{1}{2} \label{eq:3}
\end{align}
 \end{minipage}
 \begin{minipage}{0.2\linewidth}
\hfill
 \end{minipage}
 \begin{minipage}{0.5\linewidth}
 La solución anterior implica que el primer paso es la ecuación \ref{eq:2} y el último paso es la ecuación \ref{eq:3}.
 \end{minipage}
 
 
 \listoftables
 \end{document}