\documentclass[dvipsnames,xcolor=x11names]{beamer}
\usepackage[spanish]{babel}
\usepackage[utf8]{inputenc}
\usepackage[all]{xy}
\usepackage{afterpage}
\usepackage{tikz}
\usepackage{cancel}
\usepackage{verbatim}
\usepackage{tabu}
\usepackage{xfrac}
\usepackage{mathrsfs}
\usepackage{amsthm}
\usepackage{amssymb}
\usepackage{bbm}
\usepackage{enumerate}
\usepackage{booktabs}
\usepackage{relsize}
\usepackage{hyperref}
\usepackage{float}
\usepackage{longtable}
\usepackage{amsmath}
\usepackage{multirow}
\usepackage{multicol}
\usepackage{colortbl}
\usepackage{adjustbox}
\usepackage{xfrac}
\usepackage{bm}
\usepackage{keystroke}
\usepackage{wrapfig}
\usepackage{graphicx}
\usepackage{csvsimple}
\usepackage{otros/pgf-pie}
\usetikzlibrary{decorations.pathmorphing, patterns,shapes}
\usetikzlibrary{positioning}
\usepackage{pgfplots}
\pgfplotsset{compat=1.12}
\usepackage{pgfplotstable}


\PassOptionsToPackage{demo}{graphicx}

\newcommand{\hlc}[2][yellow]{ {\sethlcolor{#1} \hl{#2}} }

\newcommand*{\rom}[1]{\expandafter\romannumeral #1}
\newcommand{\Rom}[1]{\uppercase\expandafter{\romannumeral #1\relax}}

\newcommand{\Importante}[2]{{\color{#1}#2}}
\newcommand{\importante}[2]{{\color{#1}\underline{#2}}}

\renewcommand{\baselinestretch}{1}
\setlength{\parskip}{\baselineskip}

\usetheme{Boadilla}

\definecolor{colorClase}{RGB}{112,39,61}
\usecolortheme[named=colorClase]{structure}
\usepackage{natbib}


\theoremstyle{plain}
  \newtheorem{teorema}{Teorema}
  \newtheorem{proposicion}{Proposición}
  \newtheorem{corolario}{Corolario}
  \newtheorem{lema}[teorema]{Lema}
\theoremstyle{definition}
  \newtheorem{definicion}{Definici\'on}
  \newtheorem{ejemplo}{Ejemplo}
  
  
\setbeamertemplate{caption}[numbered]
 
\title{Taller usos de \LaTeX}
\subtitle{Beamer}
\setbeamersize{text margin left=25pt,text margin right=25pt}

\author[Julián Chitiva Bocanegra]{Julián Enrique Chitiva Bocanegra}
\institute[Uniandes] 
{Universidad de los Andes\\ Facultad de Economía}
\titlegraphic{\includegraphics[width=0.8cm]{img/uniandes_logo.png}
}

\date{\today}
\subject{}
\usepackage{caption}
\usepackage{subcaption}

\begin{document}

\begin{frame}
  \titlepage
\end{frame}

\begin{frame}{\texttt{\textbackslash only, \textbackslash uncover }}
   \uncover<2->
   {a partir de la diapositiva 2\\}
   \uncover<3-4>
   {desde la diapositiva 3 hasta la 4\\}
   \invisible<3>{invisible solo en la diapositiva 3}\\
   \uncover<3->{desde la diapositiva 3 hasta el final\\}
   \only<1>{Solo en la primera diapositiva\\}
   \alt<1,2,5>{Solo en las diapositivas 1,2 y 5\\}{Solo en las diapositivas 3 y 4}
\end{frame}

\begin{frame}{Tabla}
    \begin{tabular}{|l|l|l|}\hline
      \only<2>{\cellcolor{blue}}\color<2>{white}A & \only<3>{\cellcolor{blue}}\color<3>{white}B & \only<4>{\cellcolor{blue}}\color<4>{white}C  \\\hline
      \only<5>{\cellcolor{blue}}\color<5>{white}D & \alt<2->{\alt<1,3,5,7,9>{\cellcolor{blue}}{\cellcolor{red}}}{}\color<2->{white}E & \only<7>{\cellcolor{blue}}\color<7>{white}F \\ \hline
      \only<8>{\cellcolor{blue}}\color<8>{white}G & \only<9>{\cellcolor{blue}}\color<9>{white}H & \only<10>{\cellcolor{blue}}\color<10>{white}J\\ \hline
   \end{tabular}
\end{frame}
\end{document}